\chapter*{Agradecimentos}

Quando pensei em fazer o doutorado não fazia ideia do quanto minha vida se transformaria. A boa notícia é que mudou para melhor!

Não poderia deixar de fazer alguns agradecimentos aos envolvidos direta ou indiretamente na minha pesquisa e que possibilitaram trilhar essa jornada. 

Agradeço a Deus, pela sua misericórdia e providência em minha vida.

Agradeço do fundo do meu coração e com todas as forças à minha mãe, Tânia Andrade, que me educou, me ensinou a ler e escrever, sempre orou por mim, lutou para que eu sempre tivesse uma boa educação, me ajudou financeiramente quando eu precisei e sempre me incentivou a estudar e dar o melhor de mim. Apesar de uma origem humilde essa mulher pelejou para que eu pudesse concretizar o sonho do doutorado. Obrigado mãe.

À minha irmã, Aristiane Dias, por estar presente na minha vida e pelo incentivo não apenas na minha vida acadêmica, mas principalmente no âmbito pessoal.

Agradeço aos meus amigos e familiares, principalmente à minha esposa, Juliana Pontes, pela compreensão com minha falta de atenção e pela minha ausência durante este período doutoral.

Aos professores que tive na cidade de Brejinho de Nazaré que contribuíram para minha formação básica; aos professores que tive em Palmas, durante a graduação, que foram responsáveis pela minha formação superior; e finalmente aos professores que tive no mestrado e no doutorado por todo o conhecimento compartilhado no período de pós-graduação.

Aos inúmeros amigos que fiz no LAC, Laboratório de Algoritmos e Combinatória, e no PPGI, Programa de Pós-graduação em Informática, com os quais pude aprender muito e comungar de momentos de estudo e descontração.

Aos meus orientadores, Jayme, Claudson e Uéverton, por serem luz, sobriedade, ajuda, professores e amigos ao longo do tempo em que trabalhamos juntos.

Aos demais membros da banca, professoras Márcia Cerioli, Maria Pía e Liliana Alcón por avaliarem e contribuírem com este trabalho.

Não poderia deixar de reconhecer com gratidão o estágio doutoral feito na Universidade Nacional de La Plata - UNLP, Argentina. Agradeço à acolhida que tive na Argentina e na UNLP personificados nas pessoas das professoras Maria Pía e Liliana Alcón.


Agradeço ao colegiado do curso de Ciência da Computação, e demais instâncias da Universidade Federal do Tocantins que colaboraram para meu afastamento para qualificação doutoral.



Também é justo colocar um agradecimento  à rede de cafés Starbucks onde muitas vezes me retirei para escrever alguns artigos. 


À Coordenação de Aperfeiçoamento 
de Pessoal de Nível Superior - Brasil (CAPES) pelo financiamento parcial dessa pesquisa.