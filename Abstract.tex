\begin{abstract}
Golumbic, Lipshteyn e Stern definiram em 2009 a classe de grafos EPG, uma classe de grafos de intersecção baseada na intersecção de arestas em caminhos sobre uma grade. Um grafo EPG $G$ é um grafo que admite uma representação onde seus vértices correspondem a caminhos em uma grade $Q$, tal que dois vértices de $G$ são adjacentes se e somente se os caminhos correspondentes em $Q$ tem pelo menos uma aresta comum. Se os caminhos na representação tem no máximo $k$ mudanças de direção (dobras), dizemos que  essa é uma representação $B_k$-EPG. Uma coleção $C$ de conjuntos satisfaz a propriedade Helly quando toda subcoleção de $C$ que é mutuamente intersectante possui no mínimo um elemento comum. Neste trabalho mostramos que  o problema de reconhecimento de grafos  $B_k$-EPG-Helly %, de um grafo   $G=(V,E)$, cujas aresta-intersecções de caminhos em uma grade satisfaz a propriedade Helly, então chamados grafos  $B_k$-EPG-Helly, 
 está em  $\mathcal{NP}$, para todo $k$ limitado por uma função polinomial de $|V(G)|$. Além disto, mostramos que o reconhecimento de grafos $B_1$-EPG-Helly é $NP$-completo, e ele permanece   $NP$-completo mesmo quando restrito aos grafos   2-apex e 3-degenerado.


Palavras-chave: Aresta-intersecção de caminhos sobre uma grade, Propriedade Helly, Grafos de Intersecção, $NP$-completude, Dobra simples.
\end{abstract}