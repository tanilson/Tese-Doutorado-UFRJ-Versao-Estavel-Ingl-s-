\documentclass[english,dsc,numbers]{coppe}
% qualificacao - dscexam
% tese - dsc

\usepackage[english]{babel}

\usepackage{natbib}

\usepackage{amsmath,amssymb, amsthm}
\usepackage{hyperref}
\usepackage[utf8]{inputenc}
\usepackage{graphicx}
\usepackage[caption=false]{subfig}
%\usepackage{ifthen}
\usepackage{color}
\usepackage{array}
\usepackage{colortbl}
\usepackage[algoruled,lined,boxed]{algorithm2e}
\usepackage{lineno,setspace}
\usepackage{eqparbox}
\usepackage{indentfirst}
\usepackage{braket}
\usepackage{amsfonts}
\usepackage{fancyhdr}
\usepackage{comment}
\usepackage{float}
\usepackage{pdfpages}
\usepackage{enumitem}
\usepackage{lscape}
\usepackage[export]{adjustbox}
\usepackage{footnote} %para notas de rodape em tabelas
%\usepackage{cite}
%\usepackage{subfigure}

%\usepackage{caption}    
%\usepackage{subcaption} 

\usepackage{makeidx} % sugestão... caso você queira fazer um índice

\newtheorem{theorem}{Teorema}[chapter]
\newtheorem{example}[theorem]{Exemplo}
\newtheorem{conjecture}[theorem]{Conjectura}
\newtheorem{observation}[theorem]{Observa\c{c}\~ao}
\newtheorem{lemma}[theorem]{Lema}
\newtheorem{proposition}[theorem]{Proposi\c{c}\~ao}
\newtheorem{fac}[theorem]{Fato}
\newtheorem{corollary}[theorem]{Corol\'ario}
\newtheorem{lema}[theorem]{Lema}
\newtheorem{definition}[theorem]{Defini\c{c}\~ao}
%\newenvironment{proof}[1][Demonstra\c{c}\~ao]{\emph{#1.} }{\ \hfill$\square$}

\newenvironment{proofidea}{\par\noindent\textit{Justificativa.}}{\hfill$\square$}


\newcommand\floor[1]{\left\lfloor #1 \right\rfloor}
\newcommand\toricclass[1]{#1_\circ^\circ}
\newcommand{\toric}[1]{\left[#1\right]^\circ_\circ}
\renewcommand\mod[1]{\!\!\!\!\!\pmod{#1}}

%\usepackage{xparse}% http://ctan.org/pkg/xparse
%\NewDocumentCommand{\ceil}{s O{} m}{%
%  \IfBooleanTF{#1} % starred
%    {\left\lceil#3\right\rceil} % \ceil*[..]{..}
%    {#2\lceil#3#2\rceil} % \ceil[..]{..}
%}%Para função matematica piso/teto

\makelosymbols
\makeloabbreviations

\makeindex  % Se não tiver indice remissivo pode remover

\begin{document}
\title{Estudo de Grafos de Intersecção de Caminhos}
  \foreigntitle{Some Remarks About Intersection Graphs of Paths}
  \author{Tanilson Dias dos}{Santos}
  \advisor{Prof.}{Jayme Luiz}{Szwarcfiter}{Ph.D.}  
  \advisor{Prof.}{Uéverton dos Santos}{Souza}{D.Sc.}
\advisor{Prof.}{Claudson Ferreira}{Bornstein}{Ph.D.}

 \examiner{Prof.}{Jayme Luiz Szwarcfiter, Ph.D.}{Ph.D.}
    \examiner{Prof.}{Uéverton dos Santos Souza, D.Sc.}{D.Sc.}
 \examiner{Prof.}{Claudson Ferreira Bornstein, Ph.D.}{Ph.D.}
 \examiner{Profª.}{Liliana Alcón, D.Sc.}{D.Sc.}
 \examiner{Profª.}{María Pía Mazzoleni, D.Sc.}{D.Sc.}
 \examiner{Profª.}{Márcia Rosana Cerioli, D.Sc.}{D.Sc.}
  
  %\examiner{Prof.}{Nome do Quinto Examinador Sobrenome}{Ph.D.}
  \department{PESC}
  \date{09}{2020}

  \keyword{Edge Path}
  \keyword{Grid Path}
  \keyword{Intersections}
  \keyword{search}

  \maketitle

  \frontmatter
  \dedication{Dedico à minha filha, Ana Flor de Lis, e à minha esposa, Juliana Pontes.
  Dedico também aos meus avós maternos, Teobaldo Ferreira Dias (\textit{in memoriam}) e Lídia Andrade Ferreira (\textit{in memoriam}), e paternos,
  Armando Bento de Oliveira e Josefa Ericino de Oliveira (\textit{in memoriam}).
  }

  \chapter*{Agradecimentos}

Quando pensei em fazer o doutorado não fazia ideia do quanto minha vida se transformaria. A boa notícia é que mudou para melhor!

Não poderia deixar de fazer alguns agradecimentos aos envolvidos direta ou indiretamente na minha pesquisa e que possibilitaram trilhar essa jornada. 

Agradeço a Deus, pela sua misericórdia e providência em minha vida.

Agradeço do fundo do meu coração e com todas as forças à minha mãe, Tânia Andrade, que me educou, me ensinou a ler e escrever, sempre orou por mim, lutou para que eu sempre tivesse uma boa educação, me ajudou financeiramente quando eu precisei e sempre me incentivou a estudar e dar o melhor de mim. Apesar de uma origem humilde essa mulher pelejou para que eu pudesse concretizar o sonho do doutorado. Obrigado mãe.

À minha irmã, Aristiane Dias, por estar presente na minha vida e pelo incentivo não apenas na minha vida acadêmica, mas principalmente no âmbito pessoal.

Agradeço aos meus amigos e familiares, principalmente à minha esposa, Juliana Pontes, pela compreensão com minha falta de atenção e pela minha ausência durante este período doutoral.

Aos professores que tive na cidade de Brejinho de Nazaré que contribuíram para minha formação básica; aos professores que tive em Palmas, durante a graduação, que foram responsáveis pela minha formação superior; e finalmente aos professores que tive no mestrado e no doutorado por todo o conhecimento compartilhado no período de pós-graduação.

Aos inúmeros amigos que fiz no LAC, Laboratório de Algoritmos e Combinatória, e no PPGI, Programa de Pós-graduação em Informática, com os quais pude aprender muito e comungar de momentos de estudo e descontração.

Aos meus orientadores, Jayme, Claudson e Uéverton, por serem luz, sobriedade, ajuda, professores e amigos ao longo do tempo em que trabalhamos juntos.

Aos demais membros da banca, professoras Márcia Cerioli, Maria Pía e Liliana Alcón por avaliarem e contribuírem com este trabalho.

Não poderia deixar de reconhecer com gratidão o estágio doutoral feito na Universidade Nacional de La Plata - UNLP, Argentina. Agradeço à acolhida que tive na Argentina e na UNLP personificados nas pessoas das professoras Maria Pía e Liliana Alcón.


Agradeço ao colegiado do curso de Ciência da Computação, e demais instâncias da Universidade Federal do Tocantins que colaboraram para meu afastamento para qualificação doutoral.



Também é justo colocar um agradecimento  à rede de cafés Starbucks onde muitas vezes me retirei para escrever alguns artigos. 


À Coordenação de Aperfeiçoamento 
de Pessoal de Nível Superior - Brasil (CAPES) pelo financiamento parcial dessa pesquisa.

  \begin{abstract}
Um grafo EPG é um grafo de aresta-interseção de caminhos sobre uma grade. Nesta tese de doutorado exploraremos principalmente os grafos EPG, em particular os grafos $B_1$-EPG.  Entretanto, outras classes de grafos de interseção serão estudadas, como as classes de grafos VPG, EPT e VPT, além dos parâmetros número de Helly e número de Helly forte nos grafos EPG e VPG. Apresentaremos uma prova de $ NP$-completude  para o problema de reconhecimento de grafos  $B_1$-EPG-Helly. Investigamos os parâmetros número de Helly e o número de Helly forte nessas duas classes de grafos, EPG e VPG, a fim de determinar limites inferiores e superiores para esses parâmetros.  Resolvemos completamente o problema de determinar o número de Helly e o número de Helly  forte para os grafos $ B_k$-EPG e $ B_k$-VPG, para cada valor $ k$.

Em seguida, apresentamos o resultado de que todo grafo  $ B_1$-EPG Chordal está simultaneamente nas classes de grafos VPT e EPT. Em particular, descrevemos estruturas que ocorrem em grafos $ B_1$-EPG que não suportam uma representação $B_1$-EPG-Helly e assim definimos alguns conjuntos de subgrafos que delimitam subfamílias Helly.
 Além disso, também são apresentadas características de algumas famílias de grafos não triviais que estão propriamente contidas em $ B_1$-EPG-Helly.

Palavras-chave: EPG, EPT, Grafos de Interseção, $NP$-completude, Propriedade Helly, VPG, VPT.
\end{abstract}

  \begin{foreignabstract}
Golumbic, Lipshteyn and Stern defined in 2009 the class of EPG graphs, an intersection graph class  based on edge intersection of paths on a grid. An EPG graph $G$ is a graph that admits a representation where its vertices correspond to paths in a grid $Q$, such that two vertices of $G$ are adjacent if and only if the corresponding paths in $Q$ have a common edge. If the paths in the representation have at most $k$ changes of direction  (bends), we say that this is a  $B_k$-EPG representation. A collection $C$ of sets satisfies the Helly property when every sub-collection of $C$ that is pairwise intersecting has at least a common element. In this paper we show that the problem of  recognizing $B_k$-EPG-Helly graphs  
% $G=(V,E)$ whose edge-intersections of paths in a grid satisfy the Helly property, so-called $B_k$-EPG-Helly graphs, 
 is in $\mathcal{NP}$, for every $k$ bounded by a polynomial function of $|V(G)|$. In addition, we show that recognizing $B_1$-EPG-Helly graphs is $NP$-complete, and it remains $NP$-complete even when restricted to 2-apex and 3-degenerate graphs.



Keywords: Edge-intersection of paths on a grid, Helly property, Intersection graphs, $NP$-completeness, Single bend paths.

\end{foreignabstract}



  \tableofcontents
  \listoffigures
  %\listoftables
  \printlosymbols
  \printloabbreviations

  \mainmatter
  \chapter{Introduction}

\begin{flushright}
\begin{minipage}[t][0cm][b]{0.47\textwidth}
\emph{Se não podes entender, crê para que entendas. A fé precede, o intelecto segue.}
\end{minipage}

\rule[0cm]{7cm}{0.03cm}%{largura}{espessura}

Santo Agostinho
\end{flushright}

The Graph Theory is a branch of the math that is used by the Computer Science to describe and model several real and theoretical problems.  This doctoral thesis is dedicated to solving some problems of Graph Theory. In particular, in this chapter you will find a brief description of the related problems, the motivation of the study and a resume on the organization of the text.

\section{Motivation, Goals and Thesis Structure}


\emph{Graph Theory} is based on relations of points that we called vertices interconnected (by elements denoted as edges) in a network. In this context we define a graph $G=(V,E)$, where $V(G)$ denote the vertex set of $G$ and $E(G)$ its edge set. The graph is the object that we use to model the relationship among  elements of a set.

An \emph{intersection graph} is a graph that represents the pattern of intersections of a family of sets. In this doctoral thesis, we are interested in the study of intersection graphs. Issues related to intersection graphs have been attracting the attention of researchers since the 1960, e.g. \cite{erdos1966representation}, and to the present day, see~\cite{petito2002grafos,jose2018}.

 First, we know that every graph is an intersection graph, i.e. any graph can be represented by some intersection model, \cite{szpilrajn1945translation, erdos1966representation}. \citet{scheinerman1985characterizing} presents a research that is exclusively dedicated to characterization of classes of intersection graphs, also providing necessary and sufficient conditions for the existence of intersection representations for some specific graph class.


Many important graph families can be described as intersection graphs, we can cite among them interval, circular-arc, permutation, trapezoid, chordal,
disk, circle graphs are more important or at leas the most studied in the literature in general. 

For us, in this study some among these classes are interesting. The class of Interval graphs, i.e. the intersection graph of set of segments on a line, and the class of Chordal graphs, that are graphs where each cycle $C_n, n\geq 3$ has a chord. Interval graphs have been extensively studied by~\cite{lekkeikerker1962representation}. About Chordal graphs, \citet{gavril1974intersection} show that this class corresponds exactly to the intersection graph of subtrees on a tree. In this thesis, we will study intersection graphs of paths on a grid and on trees.

\citet{golumbic2009} defined the edge intersection graphs of paths on a grid (EPG graphs). Similarly, \cite{asinowski2011string, asinowski2012} defined the vertex intersection graphs of paths on a grid (VPG graphs).  Both intersection models have some practical importance, since they can be used to generalize naturally the context of
circuit layout problems and layout optimization~\cite{sinden1966topology} where a layout is modeled
as paths (wires) on a grid. Thus, they are problems that arise directly from this modeling: reducing the number of times that each path can to bend in order to minimize the cost or difficulty of production~\cite{bandy1990,molitor1991}; or  other times layouts may consist of several layers where the paths on each layer are not allowed to intersect, this can be understood as a colouring problem. These are the main applications that instigate research on the EPG and VPG representations of some graph families. Other applications and details on circuit layout problems can be found in~\cite{bandy1990, molitor1991, sinden1966topology}.

Some particular questions related to intersection graphs aroused our research interest. Among these, we can quote: What is the complexity of recognizing a class of path intersection graphs on a grid if we restrict the number of bends in each path individually and considering the fact of each set of intersections have a common element? Will it be possible to solve the problem of calculating some parameter in the class of paths intersection graphs on a grid  even when the entire paths bend $k$ times? Is there any relationship among the classes of intersection graphs when we change the tree host to a grid host? The answers to these and other questions are diluted in the next chapters of this thesis.


The text of this thesis is distributed over the next 5 chapters as follows.


Chapter~\ref{Notions} contains the definitions and concepts needed to fully understand this work.  In addition, we provide a short overview of the problems studied and a brief literature review on the main subjects covered in the text.

Chapter~\ref{cap:capiii} will be dedicated to the study of the Helly property and EPG graphs. In particular, the chapter presents an analysis of some basic EPG representations, a comparison of $L$-shaped paths and $B_1$-EPG graph classes, as well as a proof of the $NP$-completeness of the Helly-$B_1$ EPG  graph recognition problem~\cite{dmtcs:6506}. The following writings are publications resulting from this research:

\begin{enumerate}
    \item BORNSTEIN, C. F.; SANTOS, T. D.; SOUZA, U. S.; SZWARCFITER, J. L. A Complexidade do Reconhecimento de Grafos B1-EPG-Helly. In: 50º SBPO - Simpósio Brasileiro de Pesquisa Operacional, 2018, Rio de Janeiro. Cidades Inteligentes: Planejamento Urbano, Fontes Renováveis e Distribuição de Recursos, 2018.

     \item BORNSTEIN, C. F.; SANTOS, T. D.; SOUZA, U. S.; SZWARCFITER, J. L. Sobre a Dificuldade de Reconhecimento de Grafos B1-EPG-Helly. In: XXXVIII Congresso da Sociedade Brasileira de Computação, 2018, Natal - RN. Computação e Sustentabilidade, 2018. p. 113-116.

     
     \item BORNSTEIN, C. F.; SANTOS, T. D.; SOUZA, U. S.; SZWARCFITER, J. L. The complexity of B1-EPG-Helly graph recognition. In: VIII Latin American Workshop On Cliques in Graphs (LAWCG), ICM 2018 Satellite Event, 2018, Rio de Janeiro. Program and Abstracts, 2018. p. 69.

     
     \item BORNSTEIN, C. F.; GOLUMBIC, M.C.; SANTOS, T. D.; SOUZA, U. S.; SZWARCFITER, J. L.  The Complexity of Helly-B1-EPG graph Recognition . In:Discrete Mathematics \& Theoretical Computer Science (DMTCS), Source: oai:arXiv.org:1906.11185, June 4, 2020,  vol. 22 no. 1. 
\end{enumerate}

The last section of the Chapter~\ref{cap:capiii} contains a full copy of the paper of~\citet{dmtcs:6506}, which was published by the journal Discrete Mathematics \& Theoretical Computer Science (DMTCS), and which contains a full version of the demonstrations that were omitted in the chapter.


In Chapter~\ref{cap:iv}, the parameters Helly number and strong Helly number will be studied for  $ B_k$-EPG and $ B_k$-VPG graphs. We used the strategy of determining tight lower and upper bounds to show the value of the Helly and strong Helly number parameters in each class and for each value of $k$. The following writing is a paper resulting from the research carried out in the Chapter~\ref{cap:iv} and which can be found in a full copy in the last section of chapter.

\begin{enumerate}
    \item BORNSTEIN, C. F.; MORGENSTERN, G.; SANTOS, T. D.; SOUZA, U. S.; SZWARCFITER, J. L.  Helly and Strong Helly Numbers of $B_k$-EPG and $B_k$-VPG Graphs. Submitted to journal Discussiones Mathematicae Graph Theory (DMGT) in  May 16, 2020. 
\end{enumerate}



The Chapter~\ref{cap:v} presents relationship among Chordal $B_1$-EPG, VPT and EPT graphs. The reader will find a set of induced subgraphs that appear in any graph that does not belong to Helly-$B_1$ EPG but belongs to $B_1$-EPG and non-trivial graph classes that are included by Helly-$B_1$ EPG, namely Bipartite, Blocks, Cactus and Line of Bipartite. The main result of this chapter is the prove that every Chordal $B_1$-EPG graph is simultaneously in the VPT and EPT classes.  The writing of this chapter and corresponding research was done while the author of this doctoral thesis (Tanilson) was a doctoral research (sandwich doctoral by 1 year) fellow at National University of La Plata - UNLP, Math Department. 
 The following are  papers resulting from the research:
 
 
 
\begin{enumerate}

     \item ALCON, L.; MAZZOLENI, M. P.  SANTOS, T. D. Identifying Subclasses of Helly-$B_1$-EPG Graphs. Submited to: 52nd  Brazilian Operational Research Symposium (SBPO), 2020.
     
     \item ALCON, L.; MAZZOLENI, M. P.  SANTOS, T. D. Some Results for Paths in Trees and $B_1$-EPG Graphs. Submitted to: (???).

\end{enumerate}

 
The last section of the Chapter~\ref{cap:v} contains a full copy of the paper of~???, which was published/submitted by the journal ??? (???), and which contains a full version of the demonstrations that were omitted in the chapter.


Chapter~\ref{conclusao} is dedicated to discuss the results of this research and it includes the concluding remarks of this thesis with suggestions for future work and prospects for continuing this work.


The results obtained in this thesis can be found at the final of each
chapter in its last section, as well as all set of the complete proofs. In the
text we have chosen to hide some proofs for simplicity, thus the reader will find in each chapter only  proofs that we think to be more relevant. %  in structure of this thesis. %but the complete proofs are in each paper at the end corresponding chapter. 

Next, we present the basic concepts. Good reading.
  \chapter{Intersection graphs of paths on grid and trees}\label{Notions}


\begin{flushright}
\begin{minipage}[t][0cm][b]{0.47\textwidth}
\emph{Se você conhece o inimigo e conhece a si mesmo, não precisa temer o resultado de cem batalhas. }
\end{minipage}

\rule[0cm]{7cm}{0.03cm}%{largura}{espessura}

Sun Tzu
\end{flushright}

In this chapter we will present some concepts that will facilitate the understanding of the studied problems. In particular, we describe the notations and we will illustrate with examples only those concepts and definitions that are outside the basic scope of graph theory. As a basic bibliography on graphs, we suggest reading~\cite{bondy1976graph} and~\cite{jayme2018} .

\section{Preliminaries}

In this thesis we will consider finite graphs, connected and simple, i.e. graphs without loops (edge connecting a vertex in itself) or more than one edge connecting two vertices. Thus, when we talk about graphs we will consider a simple, finite and connected graph unless something different is explicitly said.



The follow we describe the terminology and notation used in this work.

A \emph{graph} $ G $ is a structure composed of two finite subsets: $ V(G) $ is the subset whose elements are called \emph{vertices}, and $ E(G) $ is a subset of unordered pairs of elements taken from $ V(G) $, which are called \emph{edges}. An edge $ e=(u, v) \in E (G) $ is formed by the pair of vertices $ u, v \in V(G) $, in this case $ u $ and $ v $ are said to be vertices \emph{adjacent}. We also say that $ e $ is \emph{incident edge} to $ u $ and $ v $. We denote the \emph{cardinality} of $ |V(G)| = n $ and $ |E(G)| = m $.

Given a vertex $v\in V(G)$,  $N(v)$ and $N[v]$ represent the \emph{open} and the  \emph{close neighborhood} of $v$ in $G$, respectively. 
For a subset $S \subseteq V(G)$,  $G[S]$ is the subgraph of $G$ induced by $S$.
 If $\mathcal{F}$ is any family of graphs, we say that  $G$ is  \emph{$\mathcal{F}$-free} if $G$ has no induced subgraph isomorphic to a member of $\mathcal{F}$.


Let $u, v$  be vertices of $G$, if $N(u) = N(v)$ then $u$ and $v$ are said \emph{false twins}, on the other hand, if $N[u] = N[v]$, then $u$ and $v$ are said \emph{true twins}. The \emph{degree of a  vertex} $v$ is denoted  by $d(v)$ and corresponds to the number of vertices adjacent to $v$, i.e., the cardinality of $|N(v)|$. The \emph{maximum degree} of a graph $G$ is denoted by $\Delta(G) = max\{d(v) | v \in V(G)\}$. Similarly, the  \emph{minimum degree} is denoted by  $\delta(G) = min\{d(v) | v \in V(G)\}$.

Given a graph $G$, and a vertex $v \in V(G)$, the graph $G\backslash \{v\}$ is obtained from $G$ by removing the vertex $v$ from its vertex set, and also removing all edges of $E(G)$ incidents at $v$. Similarly, given an edge $e \in E(G)$, the graph $G\backslash \{e\}$ is obtained from $G$ removing the edge $e$ from $E(G)$.

We say that $G'(V',E')$ is a \emph{subgraph} of a graph $G(V,E)$ when $V'\subseteq V$ and $E'\subseteq E$. When the subgraph $G'$ contains all edges of $E$ whose ends are contained in $V'$, then  $G'$ is the \emph{induced subgraph} of $G$ by $V'$.  

A graph  $G$ is a \emph{cycle}, denoted by $C_n$, if it is a sequence of vertices   $v_1, \dots, v_n, v_1$ distinct, where $v_i \neq v_j$ for $i\neq j$ and $(v_i, v_{i+1})\in E(G)$,  such that $n\geq 3$. For a cycle $C_k$, we say that it is an
\emph{even cycle} if $k$ is even and an \emph{odd cycle}, otherwise. A \emph{chord} is an edge that is between two non-consecutive vertices in a sequence of vertices of a cycle. An \emph{induced cycle}  or \emph{chordless cycle} is a cycle that has no chord. A graph that has no cycles is called \emph{acyclic}. A  graph $ G $ is \emph{connected} if there is a path between any pair of  vertices of $ G $. A graph is a \emph{tree} when it is acyclic and connected. A connected subgraph of a tree is called \emph{subtree}.

A set $\mathcal{S}$ is \emph{maximal} in relation to a particular property $P$ if $\mathcal{S}$ satisfies $P$, and each set $S'$ containing properly $\mathcal{S}$ does not satisfy $P$. In a similar way, a set $\mathcal{S}$ is \emph{minimal} in relation to a particular property $P$ if $\mathcal{S}$ satisfies $P$, and each subset  $S'$ that is properly contained in $\mathcal{S}$ does not satisfy $P$.

A graph $G$ is a \emph{intersection graph} of a family of subsets of a set $\mathcal{S}$, when it is possible to associate each vertex $v \in V(G)$ to a subset $S_v \subseteq \mathcal{S}$, such that $S_u \cap S_v \neq \emptyset$ if and only if $(u,v)\in E(G)$.  In this thesis, in particular, we will study four families of intersection graphs: the VPG, EPG, VPT and EPT graphs.


The term \emph{grid} is used to denote the Euclidean space of entire orthogonal coordinates. Each pair of entire \emph{coordinates} corresponds to a point or \emph{vertex of the grid} (which by the context is not to be confused with the vertex of the graph). The term \emph{grid edge} (which is also not to be confused with the edge of the graph), will be used to denote a pair of vertices that are at distance one in the grid. Two edges $ e_1 $ and $ e_2 $ are \emph{consecutive edges} when they share exactly one point on the grid. A grid is the \emph{host} on which we accommodate the VPG and EPG representations. When we refer to the VPT and EPT graphs, we implicitly know that the host of their representations is a tree.



 A \emph{path in the grid} is distinguished by two contexts, in the first we study families of subsets $\cal{F}$ of edge of the grid. In this context a path in the grid is defined as a finite sequence of consecutive edges  $e_1 = (v_1, v_{2}), e_2 = (v_2, v_{3}), \dots, e_i = (v_i, v_{i+1}), \dots, e_m = (v_{m}, v_{m+1})$,  where   $v_i \neq v_j$ for $i \neq j$.  We call a collection of such paths an {\it EPG representation}, i.e., a collection of paths that represent a graph via its intersection graph (considering edge intersections). {\it EPG graphs} are the class of graphs that admit an EPG representation.
  In the second context, for vertex paths, we study families of subsets $\cal{F}$ of vertex of the grid, and a path consists of a sequence of consecutive vertices of the grid  $v_1, v_2, \dots , v_k$ such that $(v_i, v_{i+1})$ is an edge of the grid, for all $i \in {1, \dots,  k - 1}$, where   $v_i \neq v_j$ for $i \neq j$, and a collection of these paths forms a {\it VPG representation} and corresponds to a {\it VPG graph}. 

 
 The first and last edges of a path are called \emph{extremities edges}.
The \emph{direction of an edge} is \emph{vertical} when the first coordinate of its vertices is equal, and is \emph{horizontal} when the second coordinate is equal. A \emph {bend} in a path is a pair of consecutive edges $ e_1, e_2 $ of the path, such that the directions of $ e_1 $ and $ e_2 $ are different. When two edges $ e_1 $ and $ e_2 $ form a bend, they are called \emph{bend edges}. A \emph{segment} is a path without bend.
 
 In context of EPG graphs, we say that two paths are
 \emph{edge-intersecting}, or simply  \emph{intersecting}, if these share at least one edge (of the grid).
 
 
 EPG graphs are a class of  intersection graphs of paths on a grid~\cite{golumbic2009}. Shortly after came the VPG graphs, this class was introduced in 2011 \cite{asinowski2011string} and \cite{asinowski2012}. 
 These classes consist of graphs whose vertices can be represented by paths of a grid $ Q$, such that two vertices of $ G $ are adjacent if and only if the corresponding paths intersect (in edges, if EPG graphs or in vertex, if VPG graphs). If every path in a representation can be represented with a maximum of $ k $ bends, we say that this graph $ G $ has a representation \emph{$ B_k$-EPG} (resp. \emph{$ B_k$-VPG}). When $ k = 1 $ we say that this is a representation of \emph{single bend}.

 Let $P$ be a family of paths on a host tree $T$ . Two types of intersection graphs from the pair $<P,T>$ are defined, namely VPT and EPT graphs.
The \textit{edge intersection graph} of $P$, EPT(P), has vertices which correspond to the members of $P$, and two vertices are adjacent in EPT(P) if and only if the corresponding paths in $P$ share at least one edge in T. Similarly, the \textit{vertex intersection graph} of $P$, VPT(P), has vertices which correspond to the members of $P$, and two vertices are adjacent in VPT(P) if and only if the corresponding paths in $P$ share at least one vertex in $T$.
%
VPT and EPT graphs are incomparable families of graphs. However, when the maximum degree of the host tree is restricted to three the family of VPT graphs coincides with the family of EPT graphs \cite{golumbic1985edge}. Also it is known that any Chordal EPT graph is VPT (see~\cite{syslo1985triangulated}). Recall that it was shown that Chordal graphs are the vertex intersection graphs of subtrees of a tree \cite{gavril1974intersection}.
% \cite{alcon2010necessary

We say that a family  $\mathcal{F}$ of sets is \emph{$k$-intersecting} if for each $F_1, F_2, \dots, F_k$ subsets of $\mathcal{F}$, we have that $F_1\cap F_2 \cap \dots \cap F_k \neq \emptyset$. We also say that a family $\mathcal{F}$  of sets is \emph{$k$-Helly}, when every subfamily $k$-intersecting $F'$ of  $\mathcal{F}$ has at least one common element.
 In particular, we say that a family of sets is \emph{pairwise intersecting}, i.e. two by two intersecting, if any two sets in the family intersect. A collection $ C $ of non-empty sets satisfies the Helly property, i.e. it is $ 2$-Helly, when every subcollection pairwise intersecting $ S $  of $ C $ has at least one element that is in every subset of $ S$.

For simplicity of notation, in this thesis when we refer to a family of sets as a Helly family it is understood that this family is $ 2$-Helly.

In Boolean algebra, a \emph{clause} is a disjunction or conjunction of literals. We say that a \emph{formula} $ F $ is in the \emph{Conjunctive Normal Form} (CNF) if $ F $ is a conjunction of clauses, where a clause is a disjunction of literals.

\section{The State-of-the-Art}

In this section, we will present the main known results on the related study topics in this work, namely Helly property, EPG, VPG, EPT and VPT graphs.

\subsection{The Helly property}


The Helly property is named in honor of the Austrian mathematician Eduard Helly, who in 1923 proposed his famous theorem about the relationship of intersecting sets. Such a theorem can be  stated as follows: given a collection of  sets $ C$, not empty, we say that this collection satisfies the Helly property when every subcollection of $C$ that is pairwise intersecting has at least one element in common.


We can note that the Helly property is a topic that has instigated scientific research since it appeared, moreover we can also mention recent works in the area of Graph Theory, see~\cite{berge1973, bergeDuchet1975, DOURADO2008, golumbic2013, dourado2006computational, teles2016, jose2018}.
The study of the Helly property proves to be useful in the most diverse areas of science, of which one can enumerate applications in semantics, code theory, computational biology, database, image processing, graph theory, optimization, in problems of location and linear programming, \cite{teles2016}. In particular, in the area of Graph Theory, the Helly property has motivated the study of several graph classes, for example we can cite the  Clique-Helly graphs~\cite{DOURADO2008}, Helly Circular-arc~\cite{safe2016essential}, Helly EPT~\cite{alcon2017helly}, Disk-Helly~\cite{lin2007faster} and Helly Hypergraphs~\cite{mulder1979median}.


In addition to the applications mentioned above, the Helly property can be studied on  $ B_k$-EPG representations, where each path is considered as a set of edges. A  graph $ G $ has a Helly $B_k$-EPG representation if there is a $ B_k$-EPG representation of $ G $ where each path has at most $ k $ bends and the representation satisfies the Helly property.
We will use the  notation $ P_{v_i} $ to indicate the path corresponding to the  vertex $ v_i$.
Figure~\ref{fig:envelopeRepresentacoes}(a) depicts two representations $ B_1$-EPG of a graph with 5 vertices. Figure~\ref{fig:envelopeRepresentacoes}(b) depicts pairwise intersecting paths ($ P_{v_1}, P_{v_2}, P_{v_5} $), containing a common edge, so this is a representation Helly $B_1$-EPG. In Figure~\ref{fig:envelopeRepresentacoes}(c), although the 3 paths are pairwise intersecting, there is no edge common to the 3 paths simultaneously, and thus they do not satisfy the Helly property.

\begin{figure}[h]
  \centering
  \begin{tabular}{ p{4cm} p{5cm} p{5cm} }
    \centering \includegraphics[width=3cm]{./img/envelope.png} & \includegraphics[width=4cm]{./img/envelopeHellyGradeTransparente.png} & \includegraphics[width=3.8cm]{./img/envelopeNaoHellyGrade.png}
    \\
    \footnotesize \centering (a) A graph with  5 vertices. & \footnotesize(b) A $B_1$-EPG representation that satisfies the Helly property & \footnotesize (c) A $B_1$-EPG representation that does not satisfy the Helly property  \\

  \end{tabular}
\caption{A graph with 5 vertices and two of its single bend representation, the first is Helly and the second does not satisfy the Helly property.} \label{fig:envelopeRepresentacoes}
\end{figure}



\subsection{An approach in EPG graphs}

A problem related to the study of EPG graphs is the problem of edge-intersection graphs of paths in a tree, well known in the literature as EPT (Edge-intersection Graphs of Paths in a Tree), see by instance~\cite{gavril1974intersection, golumbic2004recognition}. For EPT graphs, in particular, the value of the parameters Helly number, which is 2, and the Helly strong number, which is 3, are known results, also in~\cite{golumbic2004recognition}.

Regarding the hardness of the $B_k$-EPG graph recognition, only the hardness recognition of few of these graph subclasses was determined. We can name the following graphs subclasses: $ B_0$-EPG can be recognized in polynomial time, since these correspond to the interval graphs, see~\cite{booth1976}. In contrast, the $ B_1$-EPG and $ B_2$-EPG graphs recognition are $ NP$-complete problems, see~\cite{heldt2014, martin2017}, and the $ B_1$-EPG graph recognition problem remains $ NP$-complete even  for $ L$-shaped paths on a grid, see~\cite{cameron2016edge}. Moreover, in this doctoral thesis you will also find an $ NP$-completeness proof  for the  Helly-$B_1$-EPG graphs recognition in Chapter~\ref{cap:capiii}, and same chapter we further studied the subsets of $L$-shapes and its relationship with $B_1$-EPG and Helly-$B_1$-EPG graphs.

In this work we are going to study graphs that have a Helly-EPG representation and related subjects.  The Helly property related to EPG graph representations was studied by~\cite{golumbic2009} and~\cite{golumbic2013}. In particular, they determined the parameter strong Helly number of graphs $ B_1$-EPG. We determine two parameters to every class of EPG graphs, the Helly number and strong Helly number, this results are presents in Chapter~\ref{cap:iv}.

Research about graphs of edge-intersection of  paths on a grid is a relatively new topic in the area of Graph Theory. The first formal definitions of problems and applications were presented by Golumbic in 2009~\cite{golumbic2009}. Since then, several researches have been conducted by the scientific community. These questions often discuss the path representations, restrictions on the bend number in a representation, among others. 
A survey that summarizes the state-of-the-art for the topic of EPG graphs can be found at~\cite{chung201950}. 

Next, we will present some works that were successful in to delimit the \emph{bend number} for some classes of graphs.

In their study, \citeauthor{alcon2016}~\cite{alcon2016}, the authors show that 3 bends are enough to represent all graphs in the class of circular-arc graphs, i.e. they are in $ B_3$-EPG. Additionally, they also show that there are circular-arc graphs that cannot be represented with 2 bends. Using the fact that we can to represent any circular-arc graph using only a rectangle of a grid of any size, the work defines the class of EPR graphs and classifies the normal circular-arc graphs  as being $ B_2$-EPR, they also show that there are normal circular-arc graphs that are not $ B_1$-EPR. Finally, the work gives a characterization of $ B_1$-EPR graphs by a minimal family of forbidden induced subgraphs and shows that this subfamily corresponds to a subclass of normal Helly circular-arc graphs.


In paper of ~\citeauthor{biedl2010}~\cite{biedl2010}, the authors show that 5 bends are enough to represent all planar graphs and that 3 bends are enough to represent all outerplanar graphs. These results are further improved by~\cite{daniel2014b}. In addition to these results, the work shows that every bipartite planar graph has a $B_2$-EPG representation and that every Line graph has a $B_2$-EPG representation. In this thesis, we demonstrate that every Line of Bipartite graph is in Helly-$B_1$-EPG, these results are in Chapter~\ref{cap:v}.

\citeauthor{daniel2014b} in~\cite{daniel2014b} showed that 4 bends are enough to represent all planar graphs and present a linear algorithm to find this representation with 4 bends. However, the authors still comment that for some planar graphs, 3 bends are often enough to construct the representation. In fact, it is not that simply the majority of planar graphs could be constructed with 4 bends, in fact there are no known planar graphs that cannot be drawn using 3 bends. This leaves the question: if 4 bends are always enough to represent any planar graph, then are 4 bends really needed to represent any planar graph? That question is still open. The authors still conjecture that there is a graph where for any of its EPG representations there is always at least one path that needs to use the 4 bends.


The Table~\ref{tab:limitesBenNumber} presents the main known bounds for the \emph{bend number}, denoted by $b(G)$, of some graph classes.


\begin{table}[h]
\caption{Some graph classes and  known bounds to their \textit{bend number}.}
\label{tab:limitesBenNumber}
\begin{center}
\begin{tabular}{|c|c|c|}
\hline 
Graph Class & b(G) & Reference \\ 
\hline \hline  
Interval graphs & 0 & \cite{golumbic2009} \\ 
\hline 
Forests, Cycles & 1 & \cite{golumbic2013} \\ 
\hline 
Outerplanar &  2 & \cite{daniel2014b} \\ 
% \hline 
% Planar & 5 e dim $(n-1)\times(2n-3)$ & \cite{biedl2010} 2010\\ 
\hline 
Planar & $\in [3, 4]$ & \cite{daniel2014b}\\ 
\hline  
  Bipartite Planar & 2 & \cite{biedl2010} \\ 
\hline 
Line Graph & 2 & \cite{biedl2010} \\ 
\hline 
dgn(G)~\footnotemark %\footnote{Degeneracy}
$\leq k$ & $2k-1$ & \cite{daniel2014b} \\ 
\hline 
tw(G)~\footnotemark%\footnote{Treewidth} 
$\leq k$ & $2k-2$ & \cite{daniel2014b} \\ 
\hline 
Degree $\leq \Delta$ & $ \in [	\lceil \frac{\Delta}{2}\rceil, \Delta ] $ & \cite{daniel2014b} \\ 
\hline 
Circular-arc & 3 & \cite{alcon2016} \\ 
\hline 
Normal Circular-arc & 2 & \cite{alcon2016} \\ 
\hline 
Halin graphs & 2 & \cite{mathew2016}  \\ 
\hline 
\end{tabular} 
\end{center}
\end{table}



\footnotetext[1]{Degeneracy}
\footnotetext[2]{Treewidth}
% \footnotetext{Degeneracy}
% \footnotetext{Treewidth}

In addition to the results cited for bounds on the bend number of some classes of graphs, there are many works that characterize other types of graphs not mentioned in this table, such that the work of~\citeauthor{ries2009} in~\citep{ries2009} that characterizes the Chordal graphs  claw-free, bull-free and diamond-fee that have a $ B_{1}$-EPG representation. In that same article there is also a characterization of some Split graphs, with restriction on the size of the independent set or  clique, by forbidden subgraphs. The work still has an interesting result that shows that the neighborhood of every vertex of a graph $ B_1$-EPG induces a graph that is Weakly Chordal. Implicitly this paper delimits a set of Helly-$B_1$-EPG graphs, the bull-free graphs. Based on this fact in this thesis we extend the results to delimit another Helly-$B_1$ EPG subfamily, the diamond-free subfamily. This result can be found in Chapter~\ref{cap:v}.

Although it is possible to find several researches on EPG graphs investigating the bend number, the interests of studies in this class of graphs extend to other classic problems, which we can mention to follow.


In~\citet{cohen2014} a linear time recognition algorithm is presented for $ B_{1}$-EPG Cographs by a family of forbidden induced subgraphs. The algorithm that the paper presents uses the Cotree of the Cograph in the recognition process.

Approximation Algorithms for colorize $ B_1$-EPG graphs  were studied in~\cite{epstein2013approximation}. The work cited shows that the coloring problem and the maximum independent set problem are both $ NP$-complete for graphs $ B_1$-EPG even when the EPG representation is given. The authors present a 4-approximate algorithm that solves both problems, assuming that the EPG representation is given. The work still shows that the maximum clique can be found efficiently in graphs $ B_1$-EPG even when the representation is not given.

Clique coloring problems in $B_1$-EPG graphs were studied by~\cite{bonomo2017clique}. The authors consider the clique coloring problem and show that $B_1$-EPG graphs are 4-clique-colorables and present a linear time algorithm to solve the problem. Moreover, given a $B_1$-EPG representation of a graph, the paper provides a linear time algorithm that constructs a 4-clique coloring of it.
 
We can also mention as an often research with respect to EPG graphs the study of $NP$-hardness~\cite{daniel2014b, martin2017}, area of the grid necessary to represent a graph whose maximum degree is $\Delta(G)$~\cite{Asinowski2009}, and many others. The hardness of recognizing few classes of EPG graphs is known, and even for small $ k $ values only. Research with EPG graphs whose representations satisfy the Helly property is sparse. Thus, these topics and other similar topics prove to be interesting branches of research from a scientific point of view.

Finally, we mention that the $B_k$-EPG hierarchy is proper, i.e.,

$B_0$-EPG $\subset$ $B_1$-EPG $\subset$ $B_2$-EPG $\subset \dots$ $B_{k-1}$-EPG $\subset$ $B_k$-EPG $\subset$ $B_{k+1}$-EPG 

this result is demonstrated by~\citet{biedl2010} for even $k$ and  \citet{heldt2014} complete the result for all $k$.
A correlated result is presented by~\citet{Asinowski2009} that proved that for any $k$, only a small fraction of all labeled graphs on $n$ vertices are $B_k$-EPG.

\subsection{A brief approach on VPG graphs}

VPG representations arise naturally when studying circuit layout problems
and layout optimization where layouts are modelled as paths (wires) on
grids. One approach to minimize the cost or difficulty of production involves minimizing the number of times that each path bend, see~\cite{bandy1990, molitor1991, sinden1966topology}.
Other times layout may consist of
several layers where the paths on each layer are not allowed to intersect. This is naturally modelled as the coloring problem on the corresponding intersection graph, see~\cite{Alcn2017VertexIG}.


A graph is a VPG if it is the vertex intersection graph of paths in a grid. A graph is called
$B_k$-VPG if it has a $B_k$-VPG representation, i.e. if there is a representation where each path in this representation has at most $k$ bends. VPG graphs was introduced in 2011 by~\citet{asinowski2011string} and~\citet{asinowski2012}. They prove that VPG and String are
the same graph class. However, it is known that  recognizing String graphs is an $NP$-complete problem, by the results of~\cite{schaefer2003recognizing, kratochvil1991string}.

\citet{asinowski2012} study $B_0$-VPG graphs and observe
that horizontal and vertical segments have strong Helly number 2 and that the clique problem has polynomial-time complexity, given the path representation. Among other results, they present a proof that the recognition and coloring problems for $B_0$-VPG graphs are $NP$-complete. Moreover, they  give a 2-approximation algorithm for coloring $B_0$-VPG graphs. Furthermore, they prove that triangle-free $B_0$-VPG graphs are 4-colorable, and this is best  possible. In addition, they present a hierarchy of VPG graphs relating them to other known families of graphs, see Figure~\ref{fig:hierarquiaVPG}. The grid intersection graphs are shown to be equivalent to the bipartite $B_0$-VPG graphs and the circle graphs are strictly contained in $B_1$-VPG. They still prove the strict containment of $B_0$-VPG into $B_1$-VPG, and conjecture that, in general, this strict containment continues for all values of $k$. Finally, they present
a graph which is not in $B_1$-VPG. 

\begin{figure}[htb]	
\center%6.3
\includegraphics[width=6cm]{./img/hierarquiaVPG.png}
\caption{Relations between $B_k$-VPG graphs and well known graph classes \cite{asinowski2012}. }
\label{fig:hierarquiaVPG}
\end{figure}


It is known that all planar graphs are $B_2$-VPG, see~\cite{chaplick2012planar}. This paper also show that the 4-connected planar graphs constitute a subclass
of the intersection graphs of $Z$-shapes (i.e., a special case of $B_2$-VPG). Additionally, they demonstrate that a $B_2$-VPG representation of a planar
graph can be constructed in polynomial time. They further show that the triangle-free planar graphs are contact graphs of L-shapes, $\Gamma$-shapes, vertical segments and horizontal segments (i.e., a special case of contact $B_1$-VPG). 

Approximation algorithms for the maximum independent set problem over the class of $B_1$-VPG graphs are presented by~\citet{lahiri2015maximum}. Also, the NP-completeness of the decision version restricted to unit length equilateral   $B_1$-VPG graphs was established by them.

\citet{cohen2016posets} investigate the VPG graphs, and specifically the relationship between the bend number of a Cocomparability graph and the poset dimension of its complement. They show that the bend number of a Cocomparability graph $G$ is at most the poset dimension of the complement of $G$ minus one. Then, via Ramsey type arguments, they show that their upper bound is best  possible.


In \citet{felsner2016intersection}, the authors research the L-shapes representations for $B_k$-VPG graphs. The paper investigates several known subclasses of segment graphs (SEG-graphs), motivated mainly by research \cite{middendorf1992max} that states that every $[\llcorner, \ulcorner]$-shape is a SEG-graph.
They show that this subclasses of SEG-graphs belong to $[\llcorner]$-shapes, also that all Planar 3-trees, all Line graphs of Planar graphs, and all full subdivisions of Planar graphs are $[\llcorner]$-shapes. Furthermore \cite{felsner2016intersection} showed that complements of Planar graphs are $B_{17}$-VPG graphs and complements of full subdivisions are $B_2$-VPG graphs. 

In paper of \citet{golumbic2013intersection} certain subclasses of $B_0$-VPG graphs have been characterized and showed to admit polynomial time recognition. We can list these classes as Split, Chordal claw-free, and Chordal bull-free $B_0$-VPG graphs.
The $B_0$-VPG Split graphs were characterized by  a set of forbidden induced subgraph. The same is true for another listed subclasses.


In \citet{chaplick2011recognizing}, they investigate  $B_0$-VPG graphs. Their paper describe polynomial time
algorithms for recognizing chordal $B_0$-VPG graphs, and for recognizing $B_0$-VPG graphs that have a representation on a grid with 2 rows and arbitrary number of columns.

\citet{chaplick2012bend} show  that for every fixed $k$, $B_k$-VPG $\subsetneq$ $B_{k+1}$-VPG and that recognition of graphs from $B_k$-VPG is $NP$-complete even when the input graph is given by a $B_{k+1}$-VPG representation. 


$B_0$-VPG graphs restricted to Block graphs were studied by \citet{Alcn2017VertexIG}. Their research given a characterization by an infinite family of minimal forbidden induced subgraph  for $B_0$-VPG Block graphs. Fhurthemore, the work provides an alternative recognition and representation algorithm for $B_0$-VPG graphs also in the class of Block graphs.





\subsection{The EPT and VPT graphs}

Models based on paths intersection  may consider  intersections by vertices or   intersections by edges.  Cases where the paths are hosted on a tree  appear first in the literature, see for instance \cite{gavril1978recognition, golumbic1985edge, golumbic1985}.  Representations using paths on a grid were considered later, see  \cite{golumbic2009,golumbic2013, golumbic2013intersection}.

EPT and VPT graphs have applications in communication networks, see~\cite{boyaci2013graphs} in~\cite{brandstadt2013graph}. Assume
that we model a communication network as a tree $T$ and the message routes to be delivered in this communication network as paths on $T$ . Two paths conflict if they both require to use the same link (vertex). This conflict model is equivalent to an EPT (a VPT) graph. Suppose we try to find a schedule for the messages such that no two messages sharing a link (vertex) are scheduled in the same time interval. Then a vertex coloring of the EPT (VPT) graph corresponds to a feasible schedule on this network, ~\cite{boyaci2013graphs} in~\cite{brandstadt2013graph}.

 Let $P$ be a family of paths on a host tree $T$ . Two types of intersection graphs from the pair <$P,T$> are defined, namely VPT and EPT graphs.
The \textit{edge intersection graph} of $P$, EPT(P), has vertices which correspond to the members of $P$, and two vertices are adjacent in EPT(P) if and only if the corresponding paths in $P$ share at least one edge in T. Similarly, the \textit{vertex intersection graph} of $P$, VPT(P), has vertices which correspond to the members of $P$, and two vertices are adjacent in VPT(P) if and only if the corresponding paths in $P$ share at least one vertex in $T$.

VPT and EPT graphs are incomparable families of graphs. However, when the maximum degree of the host tree is restricted to three the family of
VPT graphs coincides with the family of EPT graphs \cite{golumbic1985edge}. Also it is known that any Chordal EPT graph is VPT, see~\cite{syslo1985triangulated}. Recall that it was shown that Chordal graphs are the vertex intersection graphs of subtrees of a tree \cite{gavril1974intersection}.

Next, we list some research involving EPT and VPT graphs.

%EPT and VPT graphs have been extensively studied in the literature. 
Although VPT graphs can be characterized by a fixed number of forbidden subgraphs, see~\cite{leveque2009characterizing}, it is show that EPT graphs recognition is an NP-complete problem, see~\cite{golumbic1985}. Main optimization and decision problems such as recognition~\cite{gavril1978recognition}, the maximum clique~\cite{gavril2000maximum} , the minimum vertex coloring~\cite{golumbic2004algorithmic}  and the maximum stable set problems~\cite{spinrad1995algorithms} are polynomial-time solvable in VPT whereas recognition and minimum vertex coloring problems remain NP-complete in EPT graphs~\cite{golumbic1985edge}. In contrast, we can solve in polynomial time the maximum clique,
see~\cite{golumbic1985}, and the maximum stable set, see~\cite{tarjan1985decomposition},  problems in EPT graphs. 

In~\citet{alcon2010necessary} we find a short paper that deals with EPT graphs. The paper defines the concept of satellite of a clique and we give us a necessary condition for structure of cliques in EPT graphs based on satellites of cliques. In addition, the paper presents a finite family of minimal forbidden subgraphs for the EPT class.

Next we will present the notation $[h,s,t]$ so that we can talk about some equivalences known in the literature.


The class of graphs that have an $[h,s,t]$-representation is denoted by $[h,s,t]$. A graph $G$ has an $[h,s,t]$-representation  when $h,s,$ and $t$ are positive integers such that $h \geq s$, there is a host tree $T$ with maximum degree $\Delta(T) \leq h$, there is a family of subtrees $S = \{S_u \subseteq T / u\in V(G) \}$ with $\Delta(S_u)\leq s$, and there is an edge $uv \in E(G)$ if and only if $|S_u \cap S_v|\geq t$.    

In~\citet{alcon2015characterizing} was studied a set of minimal forbidden induced subgraphs from VPT and their $[h,s,t]$-representations.
When there is no restriction on the maximum degree of $T$ or on the maximum degree of the subtrees is used the notation $h=\infty $  and $s=\infty$,  respectively. Therefore, $[\infty, \infty, 1]$ is the class of Chordal graphs and $[2, 2, 1]$ is the class of interval graphs. The classes $[\infty, 2, 1]$ and $[\infty, 2, 2]$ correspond to VPT and EPT respectively in~\cite{golumbic1985edge}; and UV and UE, respectively in~\cite{monma1986intersection}.
By taking $h=3$ they obtain a characterization by minimal forbidden induced subgraphs of the class VPT $\cap$ EPT = EPT $\cap$ Chordal = $[3,2,2] = [3,2,1]$, see~\citet{golumbic1985edge}. The paper also proved that the problem of deciding whether a given VPT graph belongs to $[h,2,1]$ is NP-complete even when restricted to the class VPT $ \cap $ Split without dominated stable vertices, among other minor results.


 It is known that when the EPT graphs are restricted to host trees of vertex degree 3 this class corresponds precisely to the Chordal EPT graphs. In~\citet{golumbic2008representing} was  proved an analogous result that Weakly Chordal EPT graphs are precisely the EPT graphs whose host tree restricted to degree 4. Moreover, they provide an algorithm to reduce a given EPT representation of a Weakly Chordal EPT graph to an EPT representation on a degree 4 tree. In resume, their proof state that  $[4, 2, 2]$ graphs are equivalent to Weakly Chordal $[\infty, 2, 2]$ graphs. In addition, we know that when the maximum degree of the host tree T is 3, the coloring problem is polynomial, by
~\cite{golumbic1985}. The paper of~\cite{golumbic2008representing} also shows the analogous polynomial result for a degree 4 host tree, thus the coloring problem on EPT graphs restricted to a host tree of vertex degree 4 is polynomial.

In~\citet{golumbic2008equivalences}, the research presents equivalences and the complete hierarchy of intersection graphs of paths in a tree, this including VPT and EPT grpahs, in particular orthodox-$[h,s,t]$ graphs with $s=2$ and considering variations of $h,t$. For more information about orthodox-$[h,s,t]$ graphs we recommend reading~\citet{jamison2005constant} and \citet{jose2018}.

Other researches still focus on variations of the EPT representations, such as \cite{boyaci2013graphs} and \cite{boyaci2016graphs}. These two articles represent a same research divided into two parts. Given a set of paths $P$, they define the graph ENPT($P$) of edge intersecting non-splitting paths of a tree, denoted by ENPT graph, as the graph having a vertex for each path in $P$, and an edge between every pair of vertices representing two paths that are both edge-intersecting and non-splitting. A graph $G$ is an ENPT graph if there is a tree $T$ and a set of paths $P$ of $T$ such that $G$ = ENPT($P$). The papers investigate basic properties of this class and proof that some graph classes belong to ENPT, such that Trees, Holes, Complete graphs, etc. Among the results they show that the problem of finding such a representation is $NP$-Hard in general also for this class.

As we can see, EPT and VPT graphs have been extensively studied in the literature. With approaches that study from classic problems in these classes of graphs to variations of constructions and representations in those same classes. In this thesis, in particular, we will study the relationship of the VPT and EPT graphs with the EPG graphs.

In the next section we present a table with the main notations used in the text.

%\subsection{The VPT graphs}

\section{Terminology of text}

Table~\ref{tab:terminologyTable} resumes the basic symbols and their meanings about graph theory.
More specific definitions will be given in the next chapters as necessary.

\rowcolors{2}{gray!25}{white}
\begin{table}[h]
\caption{Terms and basic symbols of Graph Theory used in this thesis.}
\label{tab:terminologyTable}
\begin{center}
\begin{tabular}{|c|p{10cm}|}
 \rowcolor{gray!50}
\hline 
Symbol & Description \\ 
\hline \hline  
$G=(V,E)$& Graph $G$ with vertex set $V(G)$ and edge set $E(G)$. \\
\hline 
$V(G)$ & Vertex set of $G$. \\
\hline 
$E(G)$ & Edge set of $G$.\\
\hline 
$n(G)$& Number of vertices in $G$. \\
\hline 
$m(G)$& Number of edges in $G$.\\
\hline 
$v_i$& Vertex $v_i$. \\
\hline 
$P_{v_i}$& Path corresponding to the vertex $v_i$. \\
\hline 
$e=(v_i,v_j)$& Edge $e$ with endpoints  $v_i$ and $v_j$. \\
\hline 
$d(v)$& Degree of vertex $v$. \\
\hline 
$\delta(G)$& Minimum degree of a vertex in $G$. \\
\hline 
$\Delta(G)$ & Maximum degree of a vertex in $G$.  \\
\hline 
$N(v)$ & Opened neighborhood of the vertex $v$. \\
\hline 
$N[v]$& Closed neighborhood of the vertex $v$. \\
\hline 
$G[S]$ & Induced subgraph in $G$ by subset of vertices $S$. \\
\hline 
$|S|$ & Cardinality of set $S$. \\
\hline 
$G\backslash \{v\}$ & Subgraph obtained of $G$ by removing the vertex $v$. \\
\hline 
$C_n$ & Induced Cycle with $n$ vertices. \\
\hline 
$W_n$ & Wheel graph with $n$ vertices. \\
\hline
$K_{r,s}$  &  Complete Bipartite graph with parts os size $r$ and $s$.   \\
\hline
  $K_n$ & Complete graph or clique with $n$ vertices.   \\
 \hline 
 $B_k$-representation & Representation where each path has at most $k$ bends.  \\
 \hline
  $<P,T>$ & Set of paths $P$ on a tree $T$.  \\
  $[h,s,t]$-representation & Representation on a host tree of degree at most $h$ of subtrees of degree at most $s$ and intersection of lenght at least $t$.  \\
\hline 
\end{tabular} 
\end{center}
\end{table}



In the following chapters we will dedicate ourselves to expose the main results obtained by researching this thesis.
  \chapter{The Helly property and EPG graphs}\label{cap:capiii}

\begin{flushright}
\begin{minipage}[t][0cm][b]{0.47\textwidth}
\emph{
Talento é 1\% inspiração e 99\% transpiração. }
\end{minipage}

\rule[0cm]{7cm}{0.03cm}%{largura}{espessura}

Thomas Edison
\end{flushright}

% Neste capítulo examinaremos as relações hierárquicas entre algumas classes EPG e EPG-Helly. Ademais, abordaremos representações $B_1$-EPG de alguns grafos que serão utilizados posteriormente. Primeiro, vamos observar como as classes $B_0$-EPG, $B_0$-EPG-Helly, $B_1$-EPG e $B_1$-EPG-Helly se relacionam, em seguida consideramos as representações $B_1$-EPG de $C_4$'s e do grafo octaedro. Por último, apresentaremos a prova de $NP$-completude do problema de reconhecimento de grafos $B_1$-EPG-Helly.

In this chapter we will examine the hierarchical relationships between some EPG and Helly-EPG classes. In addition, we will approach $ B_1$-EPG representations of some graphs that will be used later. First, let is focus our attention to understand how the classes $B_0$-EPG, $B_1$-EPG,  Helly-$B_1$ EPG and  $L$-shaped paths are related, then we consider the $ B_1$-EPG representations of graphs $C_4 $ and the Octahedral graph. Next, we will present the proof of $NP$-completeness to Helly-$B_1$-EPG graph recognition problem. Finally, at the end of this chapter the reader can find a section with a complete version of the paper published in the journal DMTCS that contains the set of proofs that have been omitted from the text.


\section{Introduction}
An EPG graph $G$ is a graph that admits a representation in which its vertices are represented by paths of a grid $Q$, such that two vertices of $G$ are adjacent if and only if the corresponding paths have at least one common edge.

The study of EPG graphs has motivation related to the problem of VLSI design that combines the notion of edge intersection graphs of paths in a  tree with a  VLSI  grid layout model, see~\cite{golumbic2009}. The number of bends in an integrated circuit may increase the layout area, and consequently, increase the cost of chip manufacturing.
This is one of the main applications that instigate research on the EPG representations of some graph families when there are constraints on the number of bends in the paths used in the representation.
Other applications and details on circuit layout problems can be found in~\cite{bandy1990, molitor1991}.

A graph is a $ B_k$-EPG graph if it admits a representation in which each path has at most $k$ bends. As an example, Figure~\ref{fig:trianguloepgRepresentacao}(a) shows a $C_3$, Figure~\ref{fig:trianguloepgRepresentacao}(b) shows an EPG representation where the paths have no bends and Figure~\ref{fig:trianguloepgRepresentacao}(c) shows a representation with at most one bend per path.   
Consequently, $C_3$ is a $B_0$-EPG graph. More generally, $B_0$-EPG graphs coincide with interval graphs.


\begin{figure}[h]
  \centering
  \begin{tabular}{ p{3cm} p{0.7cm} p{4cm} p{0.7cm} p{4cm} }
    \includegraphics[width=2.3cm]{./img/trianguloabc} && \includegraphics[width=3.9cm]{./img/b0epgTransparenciaGrade2} & &
    \includegraphics[width=3.5cm]{./img/b1EpgTransparenteGrade2}
    \\
    \footnotesize
    (a) The  graph $C_3$. && \footnotesize(b) $B_0$-EPG representation of $C_3$ (edge-clique).&& \footnotesize(c) $B_1$-EPG representation of $C_3$ (claw-clique).\\
  \end{tabular}

 \caption{The  graph $ C_3 $  and  representations without bends and with 1 bend.} \label{fig:trianguloepgRepresentacao}
\end{figure}


The \emph{bend number} of a graph $G$ is the smallest $k$ for which $G$ is a $B_k$-EPG graph. Analogously, the bend number of a class of graphs is the smallest $k$ for which all graphs in the class have a $B_k$-EPG representation. Interval graphs have bend number $0$, trees have bend number $1$, see~\cite{golumbic2009}, and outerplanar graphs have bend number $2$, see~\cite{daniel2014b}. The bend number for the class of planar graphs is still open, but according to \cite{daniel2014b}, it is either $3$ or $4$.

The class of EPG graphs has been studied in several papers, such as \cite{alcon2016, Asinowski2009, cohen2014, golumbic2009, heldt2014,  martin2017,golumbic2019edge}, among others. The investigations regarding EPG graphs frequently approach characterizations concerning the number of bends of the graph representations. Regarding the complexity of recognizing $B_k$-EPG graphs, only the complexity of recognizing a few of these sub-classes of EPG graphs have been determined: $B_0$-EPG graphs can be recognized in polynomial time, since it corresponds to the class of interval graphs, see ~\cite{booth1976}; in contrast, recognizing $B_1$-EPG and $B_2$-EPG graphs are NP-complete problems, see~\cite{heldt2014} and \cite{martin2017}, respectively. 
Also, note that the paths in a $B_1$-EPG representation have one of the following shapes: $\llcorner$, $\lrcorner$, $\ulcorner$ and $\urcorner$. \cite{cameron2016edge} showed that for each $S\subset \{\llcorner, \lrcorner, \ulcorner, \urcorner\}$, it is NP-complete to determine if a given graph $G$ has a $B_1$-EPG representation using only paths with shape in $S$.

A  collection $C$ of sets satisfies the Helly property when every sub-collection of $C$ that is pairwise intersecting has at least one common element. 
The study of the Helly property is useful in diverse areas of science. We can enumerate applications in semantics, code theory, computational biology, database, image processing, graph theory, optimization, and linear programming, see \cite{dourado2009}.

The Helly property can also be applied to the $B_k$-EPG representation problem, where each path is considered a set of edges. A graph $G$ has a  Helly-$B_k$-EPG representation if there is a $B_k$-EPG representation of $G$ where each path has at most $k$ bends, and this representation satisfies the Helly property. Figure~\ref{fig:envelopeRepresentacoes}(a) presents two $B_1$-EPG representations of a graph with five vertices.  Figure~\ref{fig:envelopeRepresentacoes}(b)   illustrates 3 pairwise intersecting paths ($P_{v_1}, P_{v_2}, P_{v_5}$), containing a common edge, so it is a Helly-$B_1$-EPG representation. In Figure~\ref{fig:envelopeRepresentacoes}(c), although the three paths are pairwise intersecting, there is no common edge in all three paths, and therefore they do not satisfy the Helly property.

The Helly property related to EPG representations of graphs has been studied in~\cite{golumbic2009} and~\cite{golumbic2013}. 

Let $\cal {F}$ be a family of subsets of some universal set $U$, and $h\geq 2$ be an integer.  Say that $\cal{F}$ is $h$-{\it intersecting} when every group of $h$ sets of $\cal {F}$ intersect. The {\it core} of $\cal {F}$, denoted by $core(\cal F)$, is the intersection of all sets of $\cal {F}$. The family $\cal{F}$ is $h$-{\it Helly} when every $h$-intersecting subfamily $\cal{F'}$ of $\cal{F}$ satisfies $core(\cal{F'}) \neq \emptyset$, see e.g. \cite{D76}. On the other hand, if for every subfamily $\cal{F'}$ of $\cal{F}$, there are $h$ subsets whose core equals the core of  $\cal {F'}$, then $\cal {F}$ is said to be {\it strong} $h$-{\it Helly}.
Note that the Helly property that we will consider in this paper is precisely the property of being 2-Helly. 

The  {\it Helly number} of the family $\cal{F}$ is the least integer $h$, such that $\cal{F}$ is $h$-Helly. Similarly, the {\it strong Helly number} of $\cal{F}$ is the least $h$, for which  $\cal{F}$ is strong $h$-Helly. It also follows that the strong Helly number of $\cal{F}$ is at least equal to its Helly number. In~\cite{golumbic2009} and~\cite{golumbic2013}, they have determined the strong Helly number of $B_1$-EPG graphs. 

\begin{figure}[h]
  \centering
  \begin{tabular}{ p{4cm} p{5cm} p{5cm} }
    \centering \includegraphics[width=3cm]{./img/envelope.png} & \includegraphics[width=4cm]{./img/envelopeHellyGradeTransparente.png} & \includegraphics[width=3.8cm]{./img/envelopeNaoHellyGrade.png}
    \\
    \footnotesize \centering (a) A graph with  5 vertices. & \footnotesize(b) A $B_1$-EPG representation that satisfies the Helly property & \footnotesize (c) A $B_1$-EPG representation that does not satisfy the Helly property  \\

  \end{tabular}
\caption{A graph with 5 vertices and two of its single bend representation, the first is Helly and the second does not satisfy the Helly property.} \label{fig:envelopeRepresentacoes}
\end{figure}


Next, we describe some terminology and notation.

The term \emph{grid} is used to denote the Euclidean space of integer orthogonal coordinates. Each pair of integer coordinates corresponds to a \emph{point} (or vertex) of the grid. The \emph{size} of a grid is its number of points. The term \emph{edge of the grid} will be used to denote a pair of vertices that are at a distance one in the grid. Two edges $e_1$ and $e_2$ are \emph{consecutive edges} when they share exactly one point of the grid.
 A (simple) path in the grid is as a sequence of distinct edges $e_1, e_2, \leq, e_m$,  where consecutive edges are adjacent, i.e., contain a common vertex, whereas non-consecutive edges are not adjacent.  In this context, two paths only intersect if they have at least a common edge. The first and last edges of a path are called \emph{extremity edges}.
  
The \emph{direction of an edge} is vertical when the first coordinates of its vertices are equal, and is horizontal when the second coordinates are equal. A \emph {bend} in a path is a pair of consecutive edges $ e_1, e_2 $ of that path, such that the directions of $ e_1$ and $ e_2$ are different. When two edges $ e_1$ and $e_2 $ form a bend, they are called \emph { bend edges}. A \emph {segment} is a set of consecutive edges with no bends. %is a path with no bends.
Two paths are said to be \emph{edge-intersecting}, or simply  \emph{intersecting} if they share at least one edge. Throughout the paper, any time we say that two paths intersect, we mean that they edge-intersect. If every path in a representation of a graph $G$ has at most $k$ bends, we say that this graph $G$ has a \emph{$B_k$-EPG} representation. When $k = 1$ we say that this is a \emph{single bend} representation.

\medskip

In this chapter, we study the Helly-$B_k$-EPG graphs. First, we show that every graph admits an EPG representation that is Helly, and present a characterization of Helly-$B_1$-EPG representations. Besides, we relate Helly-$B_1$-EPG graphs with L-shaped graphs, a natural family of subclasses of $B_1$-EPG. Finally, we prove that recognizing Helly-$B_k$-EPG graphs is in NP, for every fixed $k$. Besides, we show that recognizing Helly-$B_1$-EPG graphs is NP-complete, and it remains NP-complete even when restricted to 2-apex and 3-degenerate graphs.

The rest of the chapter is organized as follows. In Section~\ref{sec:prelim}, we present some preliminary results, we show that every graph is a Helly-EPG graph, present a characterization of Helly-$B_1$-EPG representations, and relate Helly-$B_1$ EPG with L-shaped graphs. In Section~\ref{sec:NPpert}, we discuss the NP-membership of {\sc Helly-$B_k$ EPG Recognition}. In Section~\ref{sec:sectionDispositivoClausula}, we present the NP-completeness of recognizing Helly-$B_1$-EPG graphs. Finally, in the last section the reader can find the complete paper accepted to  journal Discrete Mathematics \& Theoretical Computer Science (DMTCS) that contains the set of proofs that have been omitted from the text.


\section{Preliminaries} \label{sec:prelim}

Before leaving for more laborious results to obtain, let us first notice a simple result. We can observe that when we do not restrict the number of bends of each path, we can show that any graph can be represented as an EPG graph.

This study starts with the following lemma.

\begin{lemma}[\citet{golumbic2009}] \label{lem:todoGrafoEpg}
 Every graph is an EPG graph.
 \end{lemma}
 
 Moreover, the same applies to EPG-Helly graphs. 
 
In an equivalent way, it is also possible to show that every graph has an EPG representation that satisfies the Helly property. An algorithm that performs this construction is presented in the Lemma~\ref{lem:todoGrafoEpgHelly}.

 \begin{lemma}\label{lem:todoGrafoEpgHelly}
 Every graph is a Helly-EPG graph.
 \end{lemma}

\begin{proof}
Let $G$ be a graph with $n$ vertices $v_1, v_2, \dots, v_n$ and $\mu$ maximal cliques $C_1, C_2, \dots , C_{\mu }$. We construct a Helly-EPG representation of $G$ using a $\mu +1\times \mu +1$ grid $Q$. 
%The rows and columns correspond to each maximal cliques and are numbered $1, 2, \dots , \mu$. 
Each maximal clique $C_i$ of $G$ is mapped to an edge of $Q$ as follow: 
\begin{itemize}
    \item if $i$ is even then the maximal clique $C_i$ is mapped to the edge in column $i$ between rows $i$ and $i+1$;
    \item if $i$ is odd then the maximal clique $C_i$ is mapped to the edge in row $i$ between columns $i$ and $i+1$.
\end{itemize}

The following describes a descendant-stair-shaped construction for the paths.
  
Let $v_l \in V(G)$ and $C_i$ be the first maximal clique containing $v_l$ according to the increasing order of their indices. If $i$ is even (resp. odd) the path $P_l$ starts in column $i$ (resp. in row $i$), in the point $(i,i)$. Then $P_l$ extends to at least the point $(i+1, i)$ (resp. $(i, i+1)$) proceeding to the until the row (resp. column) corresponding to next maximal clique of the sequence containing $v_l$, we say $C_{j}$.
At this point, we bend $P_l$, which goes to the point $(j,j)$ and repeat the process previously described. 
%
Figure~\ref{fig:gradeDemonstracao} shows the Helly-EPG representation of the octahedral graph $O_3$, according to the construction previously described.

By construction, each path travels only rows and columns corresponding with maximal cliques containing its respective vertex. And, every path crosses the edges of the grid to which your maximal cliques were mapped. Thus, the previously described construction results in an EPG representation of $G$, which is Helly since every set ${\mathcal P}$ of paths representing a maximal clique has at least one edge in its core.
\end{proof}
 
 \begin{figure}[htb]	
\center%6.3
\includegraphics[width=8cm]{./img/grade3.png}
%clausulaGadgetGFCompletaSBPO
\caption{Representação do caminho $P_2$ correspondendo ao vértice $v_2$ contido nas cliques maximais $C_2, C_4$ e $C_5$}
\label{fig:gradeDemonstracao}
\end{figure}


 The provided construction in Lemma~\ref{lem:todoGrafoEpgHelly} can be modified to represent an monotonic row-ascendant EPG representation how in~\cite{golumbic2009} and~\cite{golumbic2013}. To do this, it is enough to change the orientation of the y-axis such that it grows from bottom to up.

%   The Helly property related to EPG representations of graphs has been studied in~\citet{golumbic2009,golumbic2013}.

\begin{definition}
The \emph{Helly-bend number} of a graph $G$, denoted by $b_H(G)$, is the smallest $k$ for which $G$ is a Helly-$B_k$-EPG graph. Also, the bend number of a graph class ${\mathcal C}$ is the smallest $k$ for which all graphs in ${\mathcal C}$ have a $B_k$-EPG representation.
\end{definition}
 
\begin{corollary}\label{cor:maxCliques}
For every graph $G$ containing $\mu$ maximal cliques, it holds that $b_H(G)\leq \mu -1$. 
\end{corollary}


The proof of Corollary~\ref{cor:maxCliques}  is immediate by the Lemma~\ref{lem:todoGrafoEpgHelly}.

Next, we examine the $B_1$-EPG representations of a few graphs that we employ in our constructions.

\medskip

Given an EPG representation of a graph $G$, for any grid edge $e$, the set of paths containing $e$ is a clique in $G$; such a clique is called an edge-clique. A claw in a grid consists of three grid edges meeting at a grid point. The set of paths that contain two of the three edges of a claw is a clique; such a clique is called a claw-clique, see~\cite{golumbic2009}. Fig.~\ref{fig:trianguloepgRepresentacao} illustrates an edge-clique and a claw-clique.

\begin{lemma}[\citet{golumbic2009}]\label{edge-claw-clique} 
Consider a $B_1$-EPG representation of a graph $G$. Every clique in $G$ corresponds to either an edge-clique or a claw-clique.
\end{lemma}

Next, we present a characterization of Helly-$B_1$-EPG representations.

\begin{lemma}\label{caracterization}
A $B_1$-EPG representation of a graph $G$ is Helly if and only if each clique of $G$ is represented by an edge-clique, i.e., it does not contain any claw-clique.
\end{lemma}

The Lemma~\ref{caracterization} is useful to prove that we can check if a given $B_1$-EPG representation is Helly. Just check each clique. 

Now, we consider EPG representations of $C_4$.

\begin{definition} \label{defi:tortasFrame}
Let $ Q $ be a grid and let $ (a_1, b),$ $(a_2, b),$ $(a_3, b),$ $(a_4, b)$ be a 4-star as depicted in Figure~\ref{fig:piesInGrid2}(a). Let $ \mathcal{P} = \{P_1, \dots , P_4\}$ be a collection of distinct paths each containing exactly two edges of the $4$-star.
\begin{itemize}
\item A \emph{true pie} is a representation where each $P_i$ of $ \mathcal{P} $ forms a bend in $b$.

\item A \emph {false pie} is a representation where two of the paths $P_i$ do not contain bends, while the remaining two do not share an edge. 
\end{itemize}
\end{definition}

Fig.~\ref{fig:piesInGrid2} illustrates true pie and false pie representations of a $C_4$.

\begin{figure}[htb]
  \centering
%segundo bloco de figuras
  \begin{tabular}{c c c c c }
    \includegraphics[width=3.5cm]{./img/disposicaoTortaGrid3}    
    & &\includegraphics[width=3.5cm]{./img/truePieGrid} 
    & &
 \includegraphics[width=3.5cm]{./img/falsePieGrid} \\%[\abovecaptionskip]
    {\footnotesize (a) 4-star in grid}  & &  {\footnotesize (b) True pie} & & {\footnotesize (c) False pie} %\label{fig:frame}
  \end{tabular}
  \caption{$B_{1}$-EPG representation of the induced cycle of size 4 as pies with emphasis in center $b$.}\label{fig:piesInGrid2}
\end{figure} 


\begin{definition} \label{defi:tortasFrame2}
 Consider a rectangle of any size with 4 corners at points $ (x_1, y_1);$ $(x_2, y_1);$ $(x_2, y_2);$ $(x_1, y_2) $, positioned as in  Fig.~\ref{fig:frameInGrid}(a). 
 \begin{itemize}
 \item A \emph{frame} is a representation containing 4 paths $\mathcal{P} =  \{ P_1, \dots, P_4\} $, each having a bend in a different corner of a rectangle, and such that the  sub-paths $ P_1 \cap P_2, P_1 \cap P_3, P_2 \cap P_4, P_3 \cap P_4 $ share at least one edge. While $P_1 \cap P_4 $ and $ P_2 \cap P_3$ are empty sets.
 
 \item A square-frame is a frame where $P_1$, $P_2$, $P_3$ and $P_4$ have respectively point of bend $ (x_1, y_1),$ $(x_2, y_1),$ $(x_1, y_2)$ and $(x_2, y_2)$, and are of the shape $\llcorner$, $\lrcorner$, $\ulcorner$ and $\urcorner$.  (see Fig.\ref{fig:frameInGrid})
 \end{itemize}
\end{definition}

Fig.~\ref{fig:frameInGrid} illustrates some frame representations of a $C_4$.





\begin{figure}[htb]
  \centering
  \begin{tabular}{c c c c c }
    \includegraphics[width=3.5cm]{./img/dispositionFrameInGrid}    
    & &
   \includegraphics[width=3.5cm]{./img/frame2} 
     & &
   \includegraphics[width=3.5cm]{./img/square2} \\%[\abovecaptionskip]
   {\footnotesize (a) Coordinates of bends of a frame}  
   & & {\footnotesize (b) An example of a frame} 
   & & {\footnotesize (c) A square-frame} %\label{fig:frame}
  \end{tabular}
  \caption{$B_{1}$-EPG representation of the induced cycle of size 4 as frame}\label{fig:frameInGrid}
\end{figure} 


\begin{lemma}[\citet{golumbic2009}]\label{lem:representacaoC4}
Every  $C_4$ that is an induced subgraph of a graph $ G $ corresponds, in any representation, to a true pie, a false pie, or a frame.
\end{lemma}

The following is a claim of~\cite{heldt2014} which a reasoning can be found in~\cite{Asinowski2009}.

\begin{lemma}[\citet{daniel2014b} and \citet{Asinowski2009}]\label{fact:k24facts}
In every single bend representation of a $K_{2,4}$, the path representing each vertex of the largest part has its bend in a false pie.
\end{lemma} %fac

By creating four $K_{2,4}$ and identifying a vertex of the largest part of each one to a distinct vertex of a $C_4$, we construct the graph called bat graph (see Fig~\ref{fig:grafoQ}). Regarding to such a graph, the following holds.

\begin{figure}[htb]
  \centering
  \begin{tabular}{c c c c c }
    \includegraphics[width=5.5cm]{./img/Qexemplo}    
    & &
   \includegraphics[width=8cm]{./img/representationQ}
  \end{tabular}
  \caption{A bat graph $G$ and a Helly-$B_1$-EPG representation of $G$.}\label{fig:grafoQ}
\end{figure} 

\begin{corollary}\label{batgraph}
In every single bend representation of the bat graph, $G$ presented in Fig.~\ref{fig:grafoQ}, the $C_4$ that is a transversal of all $K_{2,4}$ is represented by a square-frame.
\end{corollary}


\begin{figure}[htb]
  \centering
  \begin{tabular}{c c c c c }
   \includegraphics[width=5cm]{./img/representationQ2}
  \end{tabular}
  \caption{Helly-$B_1$-EPG representation of a $K_{2,4}$.}\label{fig:grafoQ2}
\end{figure} 

Figure~\ref{fig:grafoQ2} is an $B_1$-EPG representation for a $K_{2,4}$ graph. We know that any representation of a $K_{2,4}$ has the same shape for vertices in largest set, i.e. bending in a false pie. This fact is fundamental for the construction of a $B_1$-EPG representation of the bat graph, see Figure~\ref{batgraph}. When  we positioned a path $P_i$ corresponding to a vertex of the largest part of $K_{2,4}$ in construction of the bat graph, then we can state that the bend edge of each path $P_i$ does not be used by any non-member of this $K_{2,4}$. Thus each $P_i$ has to intersect another members of this $C_4$ by an extremity edge. Thus, we conclude that each path representing a vertex of the $C_4$ (transversal to all $K_{2,4}$) has its bend in a false pie and this $C_4$ is represented by a square-frame.

\begin{definition}
A $B_k$-EPG representation is \emph{minimal} 
when its set of edges does not properly contain another $B_k$-EPG representation. 
\end{definition}

The \textit{octahedral} graph is the graph containing 6 vertices and 12 edges, depicted  in Figure~\ref{fig:octaedro}(a). Next, we consider representations of the octahedral graph.
 

The next lemma follows directly from the discussion presented in~\cite{heldt2014}.

\begin{lemma}\label{lem:octaedronaohelly}
Every minimal $B_1$-EPG representation of the octahedral graph $O_3$ has the same shape.
\end{lemma}

 
\begin{figure}[h]
  \centering
  
%segundo bloco de figuras
  \begin{tabular}{@{}c@{} p{1.5cm} @{}c@{} }
   \centering \includegraphics[width=2.5cm]{./img/octaedro.png} & &\includegraphics[width=4cm]{./img/representacaoOctaedro.png}  \\[\abovecaptionskip]
    \footnotesize \centering (a) O grafo octaedro $O_3$   & &  \footnotesize(b) Representação $B_1$-EPG do grafo $O_3$
  \end{tabular}

 \caption{O grafo octaedro $O_3$ e sua representação $B_1$-EPG}\label{fig:octaedro}
\end{figure}

The true pie presented in Figure~\ref{fig:octaedro}(b) is composite by paths corresponding to the vertex set $\{b, B, c, C\}$ of the Figure~\ref{fig:octaedro}(a). But in another $B_1$-EPG representation other paths corresponding to a different set of vertices could form the true pie, in any case, yet the shape is maintained. 



\section{Subclasses of $B_1$-EPG graphs}


By Lemma~\ref{lem:octaedronaohelly}, every minimal $B_1$-EPG representation of the octahedral graph $O_3$ has the same shape, as depicted in Fig.~\ref{fig:octaedro}(b). 
Since in any representation of the graph $O_3$ there is always a triple of paths that do not satisfy the Helly property, paths $P_{a}, P_{b} $ and $P_{c}$ in the case of Fig.~\ref{fig:octaedro}(b), it holds that $O_3 \notin$ Helly-$B_1$ EPG, which implies that the class of Helly-$B_1$-EPG graphs is a proper subclass of $B_1$-EPG.

It is easy to see that any $B_0$-EPG representation is Helly. Thus, $B_0$-EPG and Helly-$B_0$-EPG graphs coincide.  Hence, Helly-$B_0$ EPG can be recognized in polynomial time, see \cite{booth1976}.

In a $B_1$-EPG representation of a graph, the paths can be of the following four shapes: $\llcorner$, $\lrcorner$, $\ulcorner$ and $\urcorner$. \citet{cameron2016edge} studied $B_1$-EPG graphs whose paths on the grid belong to a proper subset of the four shapes. If $S$ is a subset of $\{\llcorner, \lrcorner, \ulcorner, \urcorner\}$, then $[S]$ denotes the class of graphs that can be represented by paths whose shapes belong to $S$, where zero-bend paths are considered to be degenerate $\llcorner$'s. They consider the natural subclasses of $B_1$-EPG: $[\llcorner], [\llcorner, \ulcorner], [\llcorner, \urcorner]$ and $[\llcorner, \ulcorner, \urcorner]$, all other subsets are isomorphic
to these up to 90 degree rotation. \cite{cameron2016edge}  showed that recognizing each of these classes is NP-complete.

The following shows how these classes relate to the class of Helly-$B_1$-EPG graphs.
%%%%%%%%%%%%%%

%\begin{figure}[htb]	
\center%6.3
\includegraphics[width=3.5cm]{./img/diagramaClassesEPG.png}
\caption{Diagrama hierárquico de algumas classes  EPG}
\label{fig:diagramaEPG}
\end{figure}
\begin{figure}[H]	
\center%6.3
\includegraphics[width=8.5cm]{./img/classes} %2
\caption{Hierarchical diagram of some EPG classes.}
\label{fig:diagramaEPG}
\end{figure}

\begin{theorem}\label{theo:HellyLShaped}
$[\llcorner]\subsetneq [\llcorner, \urcorner]\subsetneq$~Helly-$B_1$ EPG, and Helly-$B_1$ EPG is incomparable with $[\llcorner, \ulcorner]$ and $[\llcorner, \ulcorner, \urcorner]$.
\end{theorem}

The demonstration of Theorem~\ref{theo:HellyLShaped} use statements of \cite{cameron2016edge} about $L$-shaped paths, it also use the bat graph representation (see Fig.~\ref{fig:grafoQ}) and Corollary~\ref{batgraph} and results known to Helly graphs. 

Figure~\ref{fig:diagramaEPG} depicts example of graphs of the classes $B_0$-EPG, $[\llcorner]$, $[\llcorner, \urcorner]$, Helly-$B_1$ EPG, and $B_1$-EPG that distinguish these classes.

It is known that recognizing $[\llcorner]$, $[\llcorner, \urcorner]$, and $B_1$-EPG are NP-complete while recognizing $B_0$-EPG and EPG graphs can be done in polynomial time (c.f.~\cite{booth1976}, \cite{heldt2014}, and \cite{cameron2016edge}).

Although the classes $ B_1$-EPG and Helly-$B_1$ EPG are distinct, the same is not true for the classes $B_0$-EPG and Helly-$B_0 $ EPG, because any $B_0$-EPG representation is Helly.  The $B_0$-EPG class corresponds exactly to class of interval graphs, see~\citet{booth1976}. Throughout the text we will show that each class of Helly-$B_k$ EPG is contained in the corresponding non-Helly class for all $ k \geq 1$, i.e. Helly-$B_k$ EPG $ \subsetneq$ $B_k$-EPG.

In following sections we show that it is NP-complete to recognize Helly-$B_1$-EPG graphs.


\section{Membership in $\mathcal{NP}$} \label{sec:NPpert}

%In this paper we are interested in characterizing the complexity of the $B_1$-EPG-Helly recognition problem, whose formal definition is presented next:
In this section, we will show that the (Helly-)$B_k$-EPG graph recognition problem belongs to $\mathcal{NP}$, when $k$ is polynomial in size with respect to $ |V(G)|$. The problem can be formally described as follows.

\begin{table}[h!]
\centering
%\caption{My caption}
%\label{my-label}
\begin{tabular}{ll}
\hline \hline
\multicolumn{2}{c}{\sc Reconhecimento $B_k$-EPG-Helly}                         \\ \hline \hline 
\emph{Entrada}: & Um grafo $G$, e um inteiro $k$.\\
 & \\
\emph{Objetivo}: & \begin{tabular}[c]{@{}p{12.5cm}}
Determinar se existe um conjunto de caminhos $\mathcal{P} = \{P_1, P_2, \ldots, P_n\}$, com até $k$-dobras,  em uma grade  $ Q $ 
tal que: \\
\ \ $\bullet$ $v_i, v_j\in V(G)$ são adjacentes se e somente se  $P_i,P_j$ compartilham uma aresta em $Q$; \\
\ \ $\bullet$ $\mathcal{P}$ satisfaz à propriedade Helly.
\end{tabular} \\ \hline
\end{tabular}
\end{table}


A (positive) certificate for the {\sc Helly-$B_k$ EPG recognition} consists of a grid $Q$, a set $\mathcal{P}$ of $k$-bend paths on $Q$, which is in one-to-one correspondence with the vertex set $V(G)$ of $G$, such that, for each pair of distinct paths $P_i, P_j\in \mathcal{P}, P_i\cap P_j \neq \emptyset $ if and only if the corresponding vertices are adjacent in $G$. Furthermore, $\mathcal{P}$ satisfies the Helly property.


The following are key concepts that make it easier to control the size of an EPG representation. A \emph{relevant edge} of a path in a $B_k$-EPG representation is either an extremity edge or a bend edge of the path. Note that each path with at most $k$ bends can have up to $2(k + 1)$ relevant edges, and any $B_k$-EPG representation contains at most $2|\mathcal{P}|(k + 1)$ distinct relevant edges. 


To show that there is a non-deterministic polynomial-time algorithm for {\sc Helly-$B_k$ EPG recognition}, it is enough to consider as certificate a  $B_k$-EPG representation $R$ containing a collection $\mathcal{P}$ of paths, $|\mathcal{P}| = |V(G)|$, such that  each path $P_i \in \mathcal{P}$ is given by its set of relevant edges along with the relevant edges, that intersects $P_i$, of each path $P_j$ intersecting $P_i$, where $P_j \in \mathcal{P}$.  The relevant edges for each path are given in the order that they appear in the path, to make straightforward checking that the edges correspond to a unique path with at most $k$ bends.  This representation is also handy for checking that the paths form an intersection model for $G$.

To verify in polynomial time that the input is a positive certificate for the problem, we must assert the following:

\begin{enumerate}%[label=(\roman*)]
\item[(i)] The sequence of relevant edges of a path $P_i\in \mathcal{P}$ determines $P_i$ in polynomial time; \label{it:bullet1}

\item[(ii)] Two paths $P_i, P_j \in \mathcal{P}$ intersect if and only if they intersect in some relevant edge; \label{it:bullet2}

\item[(iii)] The set $\mathcal{P}$ of relevant edges satisfies the Helly property.  \label{it:bullet3}
\end{enumerate}


%The following lemma shows condition~\ref{it:bullet1} holds.

The following lemma states that condition~(i) holds. 



\begin{lemma}\label{lem:verify1}
Each path $P_i$ can be uniquely determined in polynomial time by the sequence of its relevant edges.
\end{lemma}
\begin{proof}
%It is easy to verify that~\ref{it:bullet1} is true, 
Consider the sequence of relevant edges of some path $P_i\in \mathcal{P}$. Start from an extremity edge of $P_i$. Let $t$ be the row (column) containing the last considered relevant edge. The next relevant edge $e'$ in the sequence, must be also contained in row (column) $t$. If $e'$ is an extremity edge, the process is finished, and the path has been determined. It contains all edges between the considered relevant edges in the sequence. Otherwise, if $e'$ is a bend edge, the next relevant edge is the second bend edge $e''$ of this same bend, which is contained in some column (row) $t'$. The process continues until the second extremity edge of $P_i$ is located. 

With the above procedure, we can determine in $\mathcal{O}(k\cdot |V(G)|)$ time, whether path $P_i$ contains any given edge of the grid $Q$. Therefore, the sequence of relevant edges of $P_i$ uniquely determines $P_i$.
 \end{proof}

Next, we assert property~(ii).

\begin{lemma}\label{lem:relevantEdges}
Let $\mathcal{P}$ be the set of paths in a $B_k$-EPG representation of $G$, and let $P_1, P_2\in \mathcal{P}$. Then $P_1$, $P_2$ are intersecting paths if and only if their intersection contains at least one relevant edge.
\end{lemma}

\begin{proof}
 %If $P_1, P_2$ contain a common relevant edge there is nothing to prove. Otherwise,
Assume that $P_1, P_2$ are intersecting, and we show they contain a common relevant edge. Without loss of generality, suppose $P_1, P_2$ intersect at row \textit{i} of the grid, in the  $B_k$-EPG representation $R$. The following are the possible cases that may occur:

\begin{itemize}
\item \textbf{Case 1:} Neither $P_1$ nor $P_2$ contain bends in row \textit{i}. 

Then $P_1$ and $ P_2$  are entirely contained in row \textit{i}. Since they intersect, either $P_1, P_2$  overlap, or one of the paths contains the other. In any of these situations, they intersect in a common extremity edge, which is a relevant edge.

\item \textbf{Case 2:} $P_1$ does not contain bends in \textit{i}, but $ P_2$ does.

If some bend edge of $P_2$ also belongs to $P_1$, then $P_1, P_2$  intersect in  a relevant edge. Otherwise, since $P_1, P_2$  intersect, the only possibility is that the intersection contains an extremity edge of $P_1$ or $ P_2$. Hence the paths intersect in a relevant edge.  

\item \textbf{Case 3:} Both $P_1$,  $P_2$ contain bends in \textit{i}

Again, if the intersection occurs in some bend edge of $P_1$  or $P_2$, the lemma follows. Otherwise, the same situation as above must occur: $P_1, P_2$  must intersect in an extremity edge.
 
\end{itemize}
In any of the cases, $P_1$ and $P_2$ intersect in some relevant edge.
 \end{proof}
 
 
The two previous lemmas let us check that a certificate is an actual $B_k$-EPG representation of a given graph $G$.  The next lemma says we can also verify in polynomial time that the representation encoded in the certificate is a Helly representation. Fortunately, we do not need to check every subset of intersecting paths of the representation to make sure they have a common intersection. 

% \begin{enumerate}[label=\roman*]
% \setcounter{enumi}{2}
% \item The set $\mathcal{P}$ of relevant edges satisfies the Helly property. The following lemma shows the above condition.
% \end{enumerate}

\begin{lemma}\label{lem:verify3}
%Let $\mathcal{P}$ be the set of relevant edges of a $B_k$-EPG representation $R$ of a graph $G$. We can verify that $R$ is Helly representation in polynomial time.
Let $\mathcal{P}$ be a collection of paths encoded as a sequence of relevant edges that constitute a  $B_k$-EPG representation of a graph $G$. We can verify in polynomial time if $\mathcal{P}$ has the Helly property.
\end{lemma}
%\setcounter{proof}{2}

\begin{proof}
%Let $\mathcal{T}$ be the set of relevant edges of $R$. We consider each triple $T_i$ of edges of $\mathcal{T}$. Let $\mathcal{P}_i$ be the set of paths of $R$ containing at least two the relevant edges of $T_i$. By Gilmore's Theorem~\cite{bergeDuchet1975} the paths of $\mathcal{P}_i$  contain a common edge if only if  $R$ is a Helly representation.  By Lemma~\ref{lem:relevantEdges} they contain a common relevant edge. Since there is a polynomial number of relevant edges, we can identify such a common edge, in polynomial time, and confirm that $R$ is in fact Helly. Since the number of triples is also polynomial, the lemma follows.
Let $T$ be the set of relevant edges of $\mathcal{P}$. Consider each triple $T_i$ of edges of $T$ . Let $P_i$ be the set of paths of $\mathcal{P}$ containing at least two of the edges in the triple  $T_i$. By Gilmore's Theorem, see \cite{bergeDuchet1975}, $\mathcal{P}$ has the Helly property if an only if the subset of paths $P_i$  corresponding to each triple  $T_i$  has a non-empty intersection.  By Lemma~\ref{lem:relevantEdges}, it suffices to examine the intersections on relevant edges. Therefore a polynomial algorithm for checking if $\mathcal{P}$ has the Helly property could examine each of the subsets $P_i$, and for each relevant edge $e$ of a path in $P_i$, to compute the number of paths in $P_i$ that contain $e$. Then  $\mathcal{P}$ has the Helly property if and only if for every  $P_i$,  there exists some relevant edge that is present in all paths in $P_i$,  yielding a non-empty intersection.
 \end{proof}

\begin{corollary}\label{cor:comumAtodos}
Let ${\mathcal P'}$ be a set a pairwise intersecting paths in a Helly-$B_k$-EPG representation of a graph $G$. Then the intersection of all paths of  ${\mathcal P'}$ contains at least one relevant edge.
\end{corollary}

Note that the property described in Corollary~\ref{cor:comumAtodos} is a consequence of Gilmore's Theorem, see~\cite{bergeDuchet1975}, and it applies only to representations that satisfy Helly's property.

From Corollary~\ref{cor:comumAtodos}, the following theorem concerning the Helly-bend number of a graph holds.

\begin{theorem}\label{teo:lowerboundCliques}
For every graph $G$ containing $n$ vertices and $\mu$ maximal cliques, it holds that $$\frac{\mu}{2n}-1\leq b_H(G)\leq \mu -1.$$ 
\end{theorem}
\begin{proof}
The upper bound follows from Corollary~\ref{cor:maxCliques}.
For the lower bound first notice that each path with at most $k$ bends can have up to $2(k + 1)$ relevant edges, and any $B_k$-EPG representation with a set of paths $\mathcal{P}$ contains at most $2|\mathcal{P}|(k + 1)$ distinct relevant edges. Now, let $G$ be a graph with $n$ vertices, $\mu$ maximal cliques, and $b_H(G)=k$. From Corollary~\ref{cor:comumAtodos}, it follows that in a Helly-$B_k$-EPG representation of $G$ every maximal clique of $G$ contains at least one relevant edge. By maximality, two distinct maximal cliques cannot share the same edge-clique. Thus, in a Helly-$B_k$-EPG representation of $G$ every maximal clique of $G$ contains at least one distinct relevant edge, which implies that $\mu\leq 2n(k+1)$, so $\frac{\mu}{2n}-1\leq b_H(G)$.
\end{proof}


\begin{lemma}\label{lem:gridPolinomial}
Let $G$ be a (Helly-)$B_k$-EPG graph. Then $G$ admits a (Helly-)$B_k$-EPG representation on a grid of size at most $4n(k+1) \times 4n(k+1)$.
\end{lemma}
\begin{proof}
Let $R$ be a $B_k$-EPG representation of a graph $G$ on a grid $Q$ with the smallest possible size.
Let $\mathcal{P}$ be the set of paths of $R$. Note that $|\mathcal{P}|=n$.
A counting argument shows that there are at most $2|\mathcal{P}|(k+1)$ relevant edges in $R$. 
 If $Q$ has a pair of consecutive columns $c_i,c_{i+1}$ neither of which contains relevant edges of $R$, and such that there is no relevant edge crossing from $c_i$ to $c_{i+1}$, then we can contract each edge crossing from $c_i$ to $c_{i+1}$ into single vertices so as to obtain a new  $B_k$-EPG representation of $G$ on a smaller grid, which is a contradiction. An analogous argument can be applied to pairs of consecutive rows of the grid.
 Therefore the grid $Q$ is such that each pair of consecutive columns and consecutive rows of $Q$  has at least one relevant edge of $R$ or contains a relevant edge crossing it.  
  Since $Q$ is the smallest possible grid for representing $G$, then the first row and the first column of $Q$ must contain at least one point belonging to some relevant edge of $R$. 
Thus, if $G$ is $B_k$-EPG then it admits a $B_k$-EPG representation on a grid of size at most $4|\mathcal{P}|(k+1) \times 4|\mathcal{P}|(k+1)$.
Besides, by Corollary~\ref{cor:comumAtodos}, it holds that the contraction operation previously described preserves the Helly property, if any. Hence, letting $R$ be a Helly-$B_k$-EPG representation of a graph $G$ on a grid $Q$ with the smallest possible size it holds that $Q$ has size at most $4|\mathcal{P}|(k+1) \times 4|\mathcal{P}|(k+1)$.\end{proof}

Given a graph $G$ with $n$ vertices and an EPG representation $R$, it is easy to check in polynomial time with respect to $n +|R|$ whether $R$ is a $B_k$-EPG representation of $G$. By Lemma~\ref{lem:gridPolinomial}, if $G$ is a $B_k$-EPG graph then there is a positive certificate (an EPG representation) $R$ of polynomial size  with respect to $k+n$ to the question ``$G\in B_k$-EPG?''. Therefore, Corollary~\ref{BkNP} holds.

\begin{corollary}\label{BkNP}
Given a graph $G$ and an integer $k\geq 0$, the problem of determining whether $G$ is a $B_k$-EPG graph is in NP, whenever $k$ is bounded by a polynomial function of $|V(G)|$.
\end{corollary}

At this point, we are ready to demonstrate the NP-membership of {\sc Helly-$B_k$ EPG recognition}.

\medskip

\begin{theorem}\label{teo:nppertinencia}
{\sc Helly-$B_k$ EPG recognition} is in NP.
\end{theorem}
\begin{proof}
By Lemma~\ref{lem:gridPolinomial} and the fact that $k$ is bounded by a polynomial function of $|V(G)|$, it follows that the collection $\mathcal{P}$ can be encoded through its relevant edges with $n^{\mathcal{O}(1)}$ bits.

Finally, by Lemmas~\ref{lem:verify1}, \ref{lem:relevantEdges} and \ref{lem:verify3}, it follows that one can verify in polynomial-time in the size of $G$ whether $\mathcal{P}$ is a family of paths encoded as a sequence of relevant edges that constitute a Helly-$B_k$-EPG representation of a graph $G$.
\end{proof}

\section{$NP$-Hardness}\label{sec:sectionDispositivoClausula}

Now we will prove that  {\sc Helly-$B_1$ EPG recognition} is NP-complete. For this proof, we follow the basic strategy described in the prior hardness proof of~\cite{heldt2014}. We set up a reduction from {\sc Positive (1 in 3)-3SAT} defined  as follows:

\begin{table}[h!]
\centering
%\caption{My caption}
%\label{my-label}
\begin{tabular}{ll}
\hline \hline
\multicolumn{2}{c}{\sc Positive (1 in 3)-3SAT}                                \\ \hline \hline 
\emph{Entrada}: & \begin{tabular}[c]{@{}p{12.5cm}@{}} Um conjunto $X$ de variáveis positivas; uma coleção $\mathcal{C}=\{C_1,C_2,\ldots,C_m\}$ de cláusulas sobre  $X$ tal que para cada $C_i\in \mathcal{C}$, $|C_i|= 3$.
\end{tabular} \\
 &  \\
\emph{Objetivo}:  & \begin{tabular}[c]{@{}l@{}} %@{}l@{}
Determinar se existe uma atribuição de valores para as variáveis em $ X $\\ de modo que toda cláusula em  $\mathcal{C}$ tem exatamente um literal verdadeiro.
\end{tabular} \\ \hline
\end{tabular}
\end{table}

{\sc Positive (1 in 3)-3SAT } is a well-known NP-complete problem (see \cite{johnson1979}, problem [L04], page 259). Also, it remains NP-complete when the incidence graph of the input CNF (Conjunctive Normal Form) formula is planar, see~\cite{mulzer2008minimum}.

Given a formula $F$ that is an instance of {\sc Positive (1 in 3)-3SAT} we will present a polynomial-time construction of a graph $ G_F$ such that $ G_F \in$ Helly-$B_1$ EPG if and only if $ F $ is satisfiable. This graph will contain an induced subgraph $ G_{C_i}$ with 12 vertices (called \emph {clause gadget}) for every clause $C_i \in \mathcal{C}$, and an induced subgraph (\emph {variable gadget}) for each variable $ x_j$, containing a special vertex  $ v_j$, plus a \emph{base gadget}  with 55 additional vertices.

 
We will use a graph $H$ isomorphic to the graph presented in Figure~\ref{fig:gadgetBase}, as a gadget to perform the proof. For each clause $C_i$ of $F$ of the target problem, we will have a \emph{clause gadget} isomorphic to $H$, denoted by $G_{c_i}$.

\begin{figure}[htb]	
\center%6.3
\includegraphics[width=5cm]{./img/gadgetBase.png}
\caption{O grafo dispositivo parcial $H$}
\label{fig:gadgetBase}
\end{figure} 


The reduction of a formula $F$ from  {\sc Positive (1 in 3)-3SAT}  to a particular graph $G_F$ (where $G_F$ has a Helly-$B_1$-EPG representation if only if $F$ is satisfiable) is given below.

\begin{definition}\label{sec:reducao}
Let $F$ be a CNF-formula with variable set $\mathcal{X}$ and clause set $\mathcal{C}$ with no negative literals, in which every clause has exactly three literals. The graph $G_F$ is constructed as follows:

\begin{enumerate}
\item For each clause $C_i \in \mathcal{C}$ create a  \textit{clause gadget} $G_{C_i}$, isomorphic to  graph $H$;

\item For each variable $x_{j}\in \mathcal{X}$ create a \emph{variable vertex} $v_{j}$ that is adjacent to the vertex $a$, $e,$ or $h$ of $G_{C_i}$, when $x_{j}$ is the first, second or third variable in $C_i$, respectively;

\item For each variable vertex $v_{j}$, construct a \emph{variable gadget} formed by adding two copies of $H$, $H_1$ and $H_2$, and making $v_j$ adjacent to the vertices of the triangles $(a, b, c)$ in  $H_1$ and $H_2$.

 %where $v_{j}$ is  adjacent to all vertices of the triangle (a,b,c);%; (c,e,g); (g,f,h); or (b,d,f)) of each $H_1$ and $H_2$; 

%\item The  subgraph induced by \emph{variable vertex}  $v_{j}$, and also $V(H_1)$ and $V(H_2)$ will be called \emph{variable gadget}; 

\item Create a vertex $V$, that will be used as a vertical reference of the construction, and add an edge from $V$ to each vertex $d$ of a clause gadget;%$d \in V(G_c)$;

\item Create a bipartite graph $K_{2,4}$ with a particular vertex $T$ in the largest stable set. This vertex is nominated \emph{true vertex}. Vertex $T$ is adjacent to all $v_{j}$ and also to $V$;

\item Create two  graphs isomorphic to $H$, $G_{B1}$ and $G_{B2}$. The vertex $T$ is connected to each vertex of the triangle (a,b,c) in $G_{B1}$ and $G_{B2}$;


\item Create two graphs isomorphic  to $H$, $G_{B3}$ and $G_{B4}$. The vertex $V$ is connected to each vertex of the triangle (a,b,c) in $G_{B3}$ and $G_{B4}$;

\item The  subgraph induced by the set of vertices $\{V(K_{2,4}) \cup  \{T, V\} \cup V(G_{B1}) \cup V(G_{B2}) \cup V(G_{B3}) \cup V(G_{B4})\}$ will be referred to as the  \emph{base gadget}. 
\end{enumerate}
\end{definition}


Figure~\ref{fig:exemploGrafoGF} illustrates how this construction works on a small formula. 


\begin{figure}[htb]	
\center%6.3
\includegraphics[width=6.5cm]{./img/exemploGrafoGFSBPO4.png}
\caption{The  graph $G_{F}$ corresponding to formula $F=(x_1+ x_2+ x_3) \cdot  (x_2+ x_3+ x_4 )\cdot  (x_3 + x_1 + x_4 )$.}
\label{fig:exemploGrafoGF}
\end{figure}

\begin{lemma}\label{lem:ida}
Given a satisfiable instance $F$ of {\sc Positive (1 in 3)-3SAT}, the graph $G_F$ constructed from $F$ according to Definition~\ref{sec:reducao} admits a Helly-$B_1$-EPG representation.
\end{lemma}


Using base, variables and clause gadgets constructed in an appropriate way, it is possible to demonstrate that we can obtain a Helly-$B_1$-EPG representation from the input formula. Figure~\ref{fig:gadgetFormulaCompletaPies} depicts a Helly-$B_1$-EPG representation for the gadget built in Figure~\ref{fig:exemploGrafoGF}. For more proof details see the paper is last section of this chapter.


\begin{figure}[htb]	
\center%6.3
\includegraphics[width=10cm]{./img/formulaFGCompletaPies.png}
%clausulaGadgetGFCompletaSBPO
\caption{Representação de dobra simples de $G_F$}
\label{fig:gadgetFormulaCompletaPies}
\end{figure}

Now, we consider the converse. Let $R$ be a Helly-$B_1$-EPG representation of $G_F$, then we will get the original formula.


\begin{lemma}\label{lem:volta}
If a graph $G_F$, constructed according to Definition~\ref{sec:reducao}, admits a Helly-$B_1$-EPG representation, then the associated CNF-formula $F$ is a yes-instance of {\sc Positive (1 in 3)-3sat}.
\end{lemma}

To obtain the formula $F$ associated with the  representation $R$ of $G_F$, a few steps are necessary: 1 - identify the  clauses gadgets, so you can know how many clauses make up the formula; 2 - identify the variable gadgets, that way you can know which are the variables that make up each clause; 3 - check which variables and clauses gadgets have intersecting paths, so it is possible to identify which variables are in each clause; 4 - check the direction of the intersections among each variable gadget with the base gadget, this makes it possible to identify the value (True/False) of each variable.  This takes us to the next theorem.

\begin{theorem}
{\sc Helly-$B_1$ EPG recognition} is NP-complete.
\end{theorem}
\begin{proof} %\textbf{Proof}.
By Theorem~\ref{teo:nppertinencia}, Lemma~\ref{lem:ida}, Lemma~\ref{lem:volta}.
 \end{proof}



We say that a $k$-apex graph is a graph that can be made planar by the removal of $k$ vertices. A $d$-degenerate graph is a graph in which every subgraph has a vertex of degree at most $d$. Recall that {\sc Positive (1 in 3)-3SAT} remains NP-complete when the incidence graph of the input formula is planar, see~\cite{mulzer2008minimum}. Thus, the following corollary holds.

\begin{corollary}\label{coro:2apexAnd3degenerate}
{\sc Helly-$B_1$ EPG recognition} is NP-complete on $2$-apex and $3$-degenerate graphs.
\end{corollary}

To prove that $G_F$ is 3-degenerate, we apply the $d$-degenerate graphs recognition algorithm. Now, to prove that $G_F$ is $2$-appex when the incidence graph of the input formula is planar, we need to recall that {\sc Positive (1 in 3)-3SAT} remains NP-complete when the incidence graph of $F$ is planar, see~\cite{mulzer2008minimum}. Therefore, is enough to demonstrate that there is a planar arrangement between the intersections of the variable gadgets and clause gadgets, then from that one can construct a graph $G_F$ such that the removal of $V$ and $T$ results into a planar graph. The complete proof is in paper in the last section of this chapter.


\section{Concluding Remarks}

In this chapter, we show that every graph admits a Helly-EPG representation, and $\frac{\mu}{2n}-1\leq b_H(G)\leq \mu -1$. Besides, we relate Helly-$B_1$-EPG graphs with L-shaped graphs, a natural family of subclasses of $B_1$-EPG. Also, we prove that recognizing (Helly-)$B_k$-EPG graphs is in $\mathcal{NP}$, for every fixed $k$. Finally, we show that recognizing Helly-$B_1$-EPG graphs is NP-complete, and it remains NP-complete even when restricted to 2-apex and 3-degenerate graphs.

Now, let $r$ be a positive integer and let $K_{2r}^-$ be the cocktail-party graph, i.e., a complete graph on $2r$ vertices with a perfect matching removed. Since $K_{2r}^-$ has $2^r$ maximal cliques, by Theorem~\ref{teo:lowerboundCliques} follows that $\frac{2^r}{4r}-1\leq b_H(K_{2r}^-)$. This implies that, for each $k$, the graph $K_{2(k+5)}^-$ is not a Helly-$B_k$-EPG graph. Therefore, as \cite{martin2017} showed that every cocktail-party graph is in $B_2$-EPG, we conclude the following.

\begin{lemma}
Helly-$B_k$-EPG $\subsetneq B_k$-EPG for each $k>0$.
\end{lemma}

The previous lemma suggests asking about the complexity of recognizing Helly-$B_k$-EPG graphs for each $k>1$. Also, it seems interesting to present characterizations for Helly-$B_k$-EPG representations similar to Lemma~\ref{caracterization} (especially for $k=2$) as well as considering the $h$-Helly-$B_k$ EPG graphs. Regarding L-shaped graphs, it also seems interesting to analyse the classes Helly-$[\llcorner, \ulcorner]$ and Helly-$[\llcorner, \ulcorner, \urcorner]$ (recall Thereom~\ref{theo:HellyLShaped}).




\section{Article accepted for publication in the journal  Discrete Mathematics \& Theoretical Computer Science (DMTCS)} 
\includepdf[pages=-]{./includes/include-pdf-files/dmtcsTanilson.pdf}
  \chapter{The Helly and Strong Helly numbers fo graphs $B_k$-EPG and $B_k$-VPG}
\label{cap:iv}

\begin{flushright}
\begin{minipage}[t][0cm][b]{0.47\textwidth}
\emph{
%A Matemática é o alfabeto com o qual Deus escreveu o Universo.
Falta algo para completar esta demonstração, mas não tenho tempo.}
\end{minipage}

\rule[0cm]{7cm}{0.03cm}%{largura}{espessura}

Évariste Galois
\end{flushright}

Definir Helly number e strong Helly number.

A pergunta natural a ser feita é sobre as relações entre estes parâmetros e entre
eles e a propriedade Helly. Alguns resultados são imediatos. Por exemplo, como
a noção de número Helly forte generaliza a de número Helly temos que o número
Helly de qualquer família é menor ou igual do que seu número Helly forte.

O estudo de grafos EPG foi introduzido por  Golumbic, Lypshteyn e Stern (2009) e consiste dos grafos de intersecção de conjuntos de caminhos sobre uma grade ortogonal, cujas intersecções são tomadas considerando as arestas dos caminhos. Se as intersecções dos caminhos consideram os vértices e não as arestas, a classe de grafos resultante é chamada de grafos VPG. Tal classe foi introduzida em 2011 \cite{asinowski2011string} e \cite{asinowski2012}. Nesse capítulo estudaremos dois parâmetros em ambas classes de grafos EPG e VPG. Os parâmetros que serão estudados são nomeadamente o número de Helly e o número de Helly forte.

\section{Discussão Inicial}

Seja  $\cal {F}$ uma família de conjuntos de algum conjunto universal $U$, e $h$ um número inteiro tal que $h\geq 1$. Podemos dizer que $\cal{F}$ é $h$-{\it intersectante} quando todos  $h$ subconjuntos de $\cal {F}$ intersectam-se. Chamamos de {\it core} de $\cal {F}$ a intersecção de todos conjuntos de $\cal {F}$, e denotamos por $core(\cal F)$. 

A família $\cal{F}$ é $h$-{\it Helly} quando toda subfamília $h$-intersectante $\cal{F'}$ satisfaz $core(\cal{F'}) \neq \emptyset$, ver mais em \cite{duchet1978propriete}. Por outro lado, se para toda subfamília $\cal{F'}$ de $\cal{F}$, existem $h$ subconjuntos cujo core é igual ao core de  $\cal {F'}$, então $\cal {F}$ é dito ser  $h$-{\it Helly} {\it forte}. Claramente, se $\cal {F}$ é $h$-Helly então ele também é $h'$-Helly, para $h' \geq h$. Similarmente, se ${\cal F}$ é $h$-Helly forte então ele também é $h'$-Helly forte, para $h' \geq h$. 

Finalmente, o   {\it número de Helly} da família  $\cal{F}$ é o menor inteiro $h$, tal que $\cal{F}$ é  $h$-Helly. Similarmente, o {\it número de Helly forte} de  $\cal{F}$ é o menor $h$, para o qual  $\cal{F}$ é  $h$-Helly forte. Também segue que o número de Helly forte de $\cal{F}$ é no mínimo igual ao seu número de Helly.


Uma  {\it classe} $\cal {C}$ de famílias $\cal {F}$  de subconjuntos de algum conjunto universal $U$ é uma  subcoleção das famílias $\cal {F}$ de $U$. Dizemos que  $\cal C$ é uma {\it classe hereditária}, quando ela é fechada sob inclusão, i.e. se um grafo $G$ pertence a uma classe $C$ então todo subgrafo induzido de $G$ também pertence a $C$. O {\it número de Helly}  de uma classe  $\cal{C}$ de famílias $\cal{F}$ de subconjuntos é o maior número de Helly entre todas as famílias de $\cal {F}$. Similarmente, o {\it número de Helly forte} de uma classe  $\cal {C}$ é o maior número de Helly forte das famílias de $\cal {C}$.

Se $\cal F$ é uma família de subconjuntos e $\cal C$ uma classe de famílias, denotamos por $H(\cal F)$ e por 
$H(\cal C)$,  o número de  Helly de $\cal F$ e $\cal C$, respectivamente, enquanto  $sH({\cal F})$ e $sH({\cal C})$  representam os números de  Helly forte de $\cal F$ e $\cal C$.


Nesse capítulo, nos preocupamos com famílias de subconjuntos $\cal{F}$ de caminhos de arestas e vértices em uma grade. No primeiro contexto, consideramos que cada caminho $P_i$  consiste de uma sequência de arestas consecutivas na grade ortogonal, que forma o caminho, e chamaremos essas de   {\it representações EPG}. Dessa forma, segue que dois caminhos intersectam-se se e somente se eles contém no mínimo uma aresta da grade em comum. Aos grafos que  correspondem às representações EPG denotaremos por {\it grafos EPG}. 
No segundo contexto, um caminho é visto como uma sequência de vértices consecutivos, e dois caminhos intersectam-se se eles contém um vértice comum. Analogamente aos anteriores esses são chamados de {\it representações VPG} e {\it grafos VPG}. 

Cada aresta possui uma direção associada na grade, a qual pode ser horizontal ou vertical. Uma  {\it dobra} no caminho é um par de arestas consecutivas que possuem direções distintas.  Um {\it segmento} de um caminho é uma sequência de arestas consecutivas do caminho, sem dobras. Dizemos que o caminho $P_i$ é um  $B_k$-{\it path} se ele contém $k$ dobras. Dizemos que $\cal {F}$ é uma família de $B_k$-caminhos, ou simplesmente  uma $B_k$-família, se cada caminho de $\cal {F}$ contém no máximo $k$ dobras. 

 Nesse capítulo, resolvemos completamente o problema de determinar ambos o número de Helly e o número de Helly forte, para ambos contextos de grafos $B_k$-EPG e $B_k$-VPG. Determinamos o número de Helly em grafos $B_k$-EPG e $B_k$-VPG, para cada valor de $k$.

Para grafos EPG, o número de Helly de $B_0$-families é bem conhecido e é igual a 2, uma vez que  grafos $B_0$-EPG coincidem com grafos de intervalo. Também é simples concluir que o número de Helly forte dos grafos $B_0$-EPG é também igual a 2. Para $k = 1$,   provamos que ambos o número de Helly e número de Helly forte da classe de $B_1$-families são iguais a 3. Para a classe de  $B_2$-families, provamos que esses dois parâmetros são iguais a 4. Além disso o número de Helly e número de Helly forte para $B_3$-families é igual a 8, e finalmente esses parâmetros são ilimitados para  $k \geq 4$. 
Quanto aos grafos VPG, é simples concluir que o número de Helly de grafos $B_0$-VPG é igual a 2, e provamos que grafos $B_1$-VPG possuem número de Helly  4, grafos $B_2$-VPG possuem número de Helly  6, grafos $B_3$-VPG possuem número de Helly 12, enquanto o número de Helly para grafos $B_4$-VPG novamente é ilimitado.

Finalmente, o número de Helly forte é igual ao número de Helly nos grafos  $B_k$-EPG, para cada $k$. O mesmo vale para grafos $B_k$-VPG.

Com relação aos resultados existentes, 
Golumbic, Lipshteyn  e Stern \cite{golumbic2009} já tem mostrado que o número de Helly forte para grafos $B_1$-EPG é igual a 3, e para grafos $B_1$-VPG é igual a  4. Empregando técnicas de prova diferentes das utilizadas neste trabalho. Veja  \cite{golumbic2019edge}, Teorema 11.13, abaixo:
\begin{theorem}\label{thm:golumbic2019edge}{\cite{golumbic2019edge}}
Seja $P$ uma coleção de caminhos de dobra simples sobre uma grade. Se cada 2 caminhos em  $P$ compartilham no mínimo uma aresta da grade, então $P$ possui número de Helly forte igual a 3. Caso contrário, $P$ possui número de Helly forte igual a 4. 
\end{theorem}
Nenhum outro resultado relacionado ao número de Helly forte, ou resultados relacionados ao número de Helly de grafos $B_k$-EPG foi notado ter sido reportado na literatura levantada. Quanto a outras classes, Golumbic e Jamison  tem determinado o número de Helly forte dos caminhos de intersecção de arestas sobre uma árvore em~\cite{golumbic1985}. Finalmente, Asinowski, Cohen, Golumbic, Limouzy, Lipshteyn e Stern tem reportado que o número de Helly forte de grafos $B_0$-VPG é igual a 2 \cite{asinowski2011string}.  
Alguns resultados relacionados estão listados a seguir. Decidir se um dado hipergrafo é  $k$-Helly pode ser feito em tempo polinomial para um  $k$ fixo,  empregando a caracterização proposta por Berge e Duchet \cite{bergeDuchet1975}. Para algum $k$ arbitrário, o problema é  \cal{co-NP}-completo \cite{dourado2009}. Para ver mais problemas correspondendo à propriedade $k$-Helly forte e exemplos sugerimos a leitura de~\cite{dourado2008strong,dourado2009}.

Este capítulo está organizado como listado a seguir. A seção~\ref{sec:preliminares4}, contém algumas proposições preliminares e notações adicionais utilizadas neste escrito. A seção~\ref{sec:Helly-number}  descreve os resultados para o número de Helly de grafos $B_k$-EPG, enquanto a seção~\ref{sec:helly-vpg} contém resultados desse parâmetro para grafos $B_k$-VPG. O número de Helly forte é considerado na seção~\ref{sec:helly-forte}. Considerações finais são efetuadas na seção~\ref{sec:finalRemarks4}.

\section{Preliminares}\label{sec:preliminares4}

O seguinte teorema caracteriza famílias de subconjuntos $h$-Helly.


\begin{theorem}\label{thm:BD}(\cite{bergeDuchet1975}):
Uma família $\cal{F}$ de subconjuntos do conjunto universal  $U$ é $h$-Helly se e somente se para todo subconjunto   $U' \subseteq U$, $|U'|= h+1$,  a subfamília  $\cal{F'}$ de $\cal{F}$,  formada pelos subconjuntos contendo no mínimo  $h$ dos $h+1$ elementos de $U'$, tem um core não vazio. 
\end{theorem}

O próximo teorema é central para os nossos resultados.

\begin{theorem}\label{thm:minimal}Seja ${\cal C}$ uma classe hereditária de famílias ${\cal F}$ de subconjuntos do conjunto universal $U$, cujo número de Helly $H({\cal C})$ é igual a $h$. Então, existe uma família ${\cal F'} \in {\cal C}$ com exatamente $h$ subconjuntos, satisfazendo as seguintes condições: 

Para cada subconjunto  $P_i \in \cal {F'}$, existe exatamente um elemento distinto $u_i \in U$, tal que \\
$$u_i \not \in P_i,$$ 
mas $u_i$ está contido em todos subconjuntos 
$$P_j \in {\cal F'} \setminus P_i.$$
\end{theorem}
 

Proof: 
Seja ${\cal C}$ uma classe de famílias  ${\cal F}$ de subconjuntos $P$, cada subconjunto formado pelos elementos  $u \in U$, tal que o número de Helly $H({\cal C})$  é igual a $h$. Então cada família   ${\cal F} \in {\cal C}$ satisfaz $H({\cal F}) \leq h$. Considere uma família ${\cal F'} \in {\cal C}$  cujo número de Helly é  exatamente $h$, e contendo exatamente $h$ subconjuntos. Essa família deve existir uma vez que  ${\cal C}$ é uma classe hereditária. Além disso $H({\cal F'}) = h$, $\cal F'$ é $h$-intersectante, e portanto $(h-1)$-intersectante. Ademais, ${\cal F'}$ não é $(h-1)$-Helly. Aplicando o   Teorema~\ref{thm:BD}, podemos concluir que existem   $h$ elementos $U' = \{u_1, \ldots, u_h\} \subset U$, tal que cada conjunto de ${\cal F'}$ contem no mínimo $h-1$ elementos de $U'$. Uma vez que $H({\cal F'}) > h-1$, $core({\cal F'}) = \emptyset$ e além disso não existe elemento comum entre os conjuntos de $\cal F'$. Em particular, uma vez que cada conjunto $P_i \in {\cal F'}$ contem no mínimo  $h-1$ elementos de $U'$, e $core(\cal F') = \emptyset$, podemos escolher   $h$ subconjuntos $P_i$, em que cada um deles deixa de possuir um elemento distinto  $u_i \in U'$. Então para cada subconjunto  $P_i \in \cal F$, existe algum elemento $u_i \not \in P_i$, mas $u_i \in P_j$, para todos $P_j \in \cal F'$, $j \neq i$. \qed

Seja $\cal{ F'}$ como descrito no teorema anterior. É simples concluir que ao remover qualquer subconjunto de $\cal {F'}$ este torna-se $(h-1)$-Helly.  Todavia podemos chamar $\cal {F'}$ de uma {\it família minimal não}-$(h-1)$-{\it Helly}. Além do mais, o elemento $u_i \not \in P_i$, contido em todos os subconjuntos $P_j \in {\cal{F'}} \setminus P_i$, exceto $P_i$, é o {\it $h$-não-representativo} de $P_i$.  

Empregaremos as famílias de subconjuntos minimais citadas anteriormente, aplicadas à $B_k$-caminhos em uma grade. Note que $B_k$-caminhos em uma grade formam uma classe hereditária.

\section{O Número de Helly de Grafos $B_k$-EPG}\label{sec:Helly-number}

Nessa seção determinaremos o número de Helly das classes de grafos $B_1$-EPG, $B_2$-EPG e $B_3$-EPG, e mostraremos que para os grafos $B_k$-EPG, $k \geq 4$, o número de Helly é ilimitado. Provaremos os seguintes resultados.

\begin{theorem}\label{thm:Helly-EPG}
O número de Helly de grafos $B_k$-EPG satisfaz:
\begin{enumerate}[nosep,label=\emph{(\roman*)}]
\item  $H(B_1$-EPG) = 3 
\item $H(B_2$-EPG)  = 4 
\item $H(B_3$-EPG)  = 8 
\item $H(B_k$-EPG) é ilimitado, para 
$k \geq 4$.
\end{enumerate}

\end{theorem}

A prova consiste em determinar limites inferiores e limites superiores justos, como mostrado nas próximas subseções.

\subsection{Limites Inferiores}

Primeiro, descreveremos limites inferiores para o parâmetro número de Helly, como função do número de dobras $k$.

\begin{lema}\label{claim:lower-Bk-EPG} 
Os seguintes são limites inferiores para os grafos  $B_k$-EPG.
\begin{enumerate}[nosep,label=\emph{(\roman*)}]
\item   $H(B_1$-$EPG) \geq 3$ 
\item $H(B_2$-$EPG) \geq 4$ 
\item $H(B_3$-$EPG) \geq 8$ 
\item $H(B_k$-$EPG )$ é ilimitado para  $k \geq 4$.
\end{enumerate}
\end{lema}

\proof:

Para cada valor de  $k$, exibiremos uma $B_k$-família de caminhos em arestas na grade, tendo o número de dobras requerido, e cujo número de Helly é no mínimo o valor correspondente declarado. Nos referimos ao par de coordenadas dos pontos da grade (arestas da grade), de forma a descrever os caminhos.

\begin{figure}[!h]
\begin{center}
% \begin{tikzpicture}[line cap=round,line join=round,>=triangle 45,x=3.7mm,y=3.7mm]
% \draw [color=cqcqcq,, xstep=0.37cm,ystep=0.37cm] (-7,-1.4) grid (22.16,6.8);
% \clip(-5.7,-2.5) rectangle (25,9.6);
% \draw (-5,-1.3) node[anchor=north west] {(a)};
% \draw (2.5,-1.3) node[anchor=north west] {(b)};
% \draw (14.5,-1.3) node[anchor=north west] {(c)};
% \draw [line width=2pt] (4,0)-- (4,2)-- (6,2)-- (6,0);
% \draw [line width=2pt] (3,2)-- (1,2)-- (1,0)-- (3,0);
% %\draw (-0.3,3.7) node[anchor=north west] {0};
% %\draw (-0.3,5.1) node[anchor=north west] {1};
% %\draw (-0.3,6.1) node[anchor=north west] {2};
% \draw [line width=2pt] (1,5)-- (3,5)-- (3,3)-- (1,3);
% \draw [line width=2pt] (4,5)-- (4,3)-- (6,3)-- (6,5);
% %\draw (8.7,0.7) node[anchor=north west] {0};
% %\draw (8.7,2.1) node[anchor=north west] {1};
% %\draw (8.7,3.1) node[anchor=north west] {2};
% %\draw (8.7,3.7) node[anchor=north west] {0};
% %\draw (8.7,5.1) node[anchor=north west] {1};
% %\draw (8.7,6.1) node[anchor=north west] {2};
% \draw [line width=2pt] (10,5)-- (12,5)-- (12,3)-- (10,3)-- (10,4);
% \draw [line width=2pt] (14,5)-- (13,5)-- (13,3)-- (15,3)-- (15,5);
% \draw [line width=2pt] (19,3)-- (19,5)-- (21,5)-- (21,3)-- (20,3);
% \draw [line width=2pt] (18,4)-- (18,5)-- (16,5)-- (16,3)-- (18,3);
% \draw [line width=2pt] (10,1)-- (10,2)-- (12,2)-- (12,0)-- (10,0);
% \draw [line width=2pt] (13,2)-- (13,0)-- (15,0)-- (15,2)-- (14,2);
% \draw [line width=2pt] (20,0)-- (19,0)-- (19,2)-- (21,2)-- (21,0);
% \draw [line width=2pt] (18,2)-- (16,2)-- (16,0)-- (18,0)-- (18,1);
% \draw [line width=2pt] (-6,2.25)-- (-4.25,2.25)-- (-4.25,4);
% \draw [line width=2pt] (-3.75,4)-- (-3.75,2.25)-- (-2,2.25);
% \draw [line width=2pt] (-6,1.75)-- (-2,1.75);
% \end{tikzpicture}
\includegraphics[width=12cm]{./img/b1epgSub.pdf}
\end{center}
\caption{Subfamílias Minimais não-Helly para $B_1$, $B_2$ e $B_3$ -families.}
\label{fig:b1b2b3families}
\end{figure}

Para $k=1$, seja $\cal{F}$ uma família de três caminhos de 1-dobra que são mutuamente intersectantes mas que não possuem aresta em comum, como ilustrado na Figura~\ref{fig:b1b2b3families}$(a)$. 
%For $k=1$, let $\cal{F}$ be the família %of three 1-dobra caminhos, $P_1: (0,0),(0,1),(1,1)$; $P_2: (1,1), (1,0),(0.2)$; and $P_3: (0,0),(0,2)$.  See Figure   $1a$. 
Então $\cal{F}$ é uma família de três caminhos  2-intersectante e $B_1$-EPG, tendo um core vazio. Portanto, $H(B_1$-EPG$) \geq 3$. 
Além disso, pela remoção de qualquer dos caminhos o core de $\cal{F}$ torna-se não-vazio. Todavia $\cal{F}$ é uma família de caminhos minimal não 2-Helly.

Seja $S$ o 4-ciclo, formado pelos 4 segmentos de pares de arestas, com dobras nos seguintes pontos da grade $(0,0),(0,2),(2,2),(2,0)$, respectivamente.
 Para $k= 2$,  considere $\cal{F}$ ser a família de quarto caminhos de 2-dobras formada quando removemos exatamente um par de arestas que formam os segmentos de cada lado do 4-ciclo, como ilustrado na  Figura~\ref{fig:b1b2b3families}$ (b)$.
Segue que $\cal{F}$ é 3-intersectante e seus caminhos não possuem nenhuma aresta comum. Consequentemente $H(B_2$-EPG)$ \geq 4$.

Para $k=3$, considere a família $\cal{F}$ de oito caminhos de  3-dobras, respectivamente, que podem ser obtidos de $S$. O 4-ciclo $S$ contem exatamente  8 arestas da grade. A  família $\cal{F}$ consiste de oito caminhos $P_i$, $1 \leq i \leq 8$, obtidos pela remoção de $S$, exatamente uma dessas oito arestas distintas, como ilustrado na Figura~\ref{fig:b1b2b3families} $ (c)$. Consequentemente, $\cal{F}$ é 7-intersectante, mas $core({\cal{F}}) = \emptyset$. Portanto, $H(B_3$-EPG)$\geq 8$.

Finalmente, para $k = 4$, seja $\cal{F}$ a família de $n$ $B_4$-caminhos $P_i$, descrita como segue:

\begin{itemize}
    \item $P_1$ é formado pelos segmentos: \\ $(0,0),(0,1),(1,1),(1,0),(n,0)$; 

     \item para $2 \leq i \leq n-1$, $P_i$ contem os segmentos: \\
     $(0,0),(0,i-1),(i-1,1),(i,1),(i,0),(n,0)$;
     
     \item  $P_n$ é formado pelos seguintes segmentos: \\ $(0,0),(n-1,0),(n-1,1),(n-1,0).$
     
\end{itemize}     

Observe que $\cal{F}$ é $(n-1)$-intersectante, enquanto $core({\cal{F}})=\emptyset$. Veja a Figura~\ref{fig:figurab4}. Dessa forma $H(B_4$-EPG) é ilimitado. Claramente o mesmo vale para $k >4$. \qed  



\begin{figure}[!h]
\begin{center}
% \begin{tikzpicture}[line cap=round,line join=round,>=triangle 45,x=3.7mm,y=3.7mm]
% \draw [color=cqcqcq,, xstep=0.74cm,ystep=0.74cm] (-7,-1.0) grid (22.8,14.8);
% \clip(-6.7,-1.8) rectangle (27,14.6);
% \draw [line width=2pt] (-6,0)-- (-6,2)-- (-4,2)-- (-4,0)-- (22,0);

% \draw [line width=2pt] (-6,4)-- (-4,4)-- (-4,6)-- (-2,6)-- (-2,4)-- (22,4);

% %\draw [line width=2pt] (-6,6)-- (-2,6)-- (-2,8)-- (0,8)-- (-0,6)-- (22,6);

% \draw [line width=2pt] (-6,8)-- (10,8)-- (10,10)-- (12,10)-- (12,8)-- (22,8);

% \draw [line width=2pt] (-6,12)-- (20,12)-- (20,14)-- (22,14)-- (22,12);


% \draw (6.5,6.5) node[anchor=north west] {.};
% \draw (6.5,7) node[anchor=north west] {.};
% \draw (6.5,7.5) node[anchor=north west] {.};

% \draw (6.5,10.5) node[anchor=north west] {.};
% \draw (6.5,11) node[anchor=north west] {.};
% \draw (6.5,11.5) node[anchor=north west] {.};

% \draw (-5.65,0.4) node[anchor=north west] {1};
% \draw (-3.65,3.4) node[anchor=north west] {2};

% %\draw (-1.65,6.4) node[anchor=north west] {3};

% \draw (10.35,8.4) node[anchor=north west] {i};

% \draw (20.35,12.4) node[anchor=north west] {n};

% \end{tikzpicture}
\includegraphics[width=12.5cm]{./img/b4epg.pdf}
\end{center}
\caption{$B_4$ possui número de Helly ilimitado.}\label{fig:figurab4}
\end{figure}

A seguir, nos preocupamos em encontrar limites superiores para grafos $H(B_k$-EPG).

\subsection{Limites Superiores}\label{subsec-upper}

De forma a obter um limite superior justo para o número de Helly, em termos do número de dobras, introduzimos abaixo mais algumas notações e lemas.

Dizemos que um conjunto de arestas da grade é {\it co-linear} se todas arestas do conjunto pertencem a uma mesma linha da grade, horizontal ou vertical. O conjunto de arestas é chamado {\it paralelo} se todas as suas arestas estiverem sobre linhas paralelas da grade mas nenhuma delas for co-linear.


\begin{lemma}
\label{lemma:3colin}
Seja $\cal {F}$ uma família minimal não-$(h-1)$-Helly de caminhos sobre uma grade contendo três arestas co-lineares não representativas. Então $\cal{F}$ deve conter caminhos com no mínimo quatro dobras.
\end{lemma}

\proof
Seja  $u_i$ a aresta do meio entre as três arestas  co-lineares não representativas. Ela corresponde ao caminho $P_i$ de $\cal {F'}$, não contendo $u_i$.
 Então $P_i$ deve passar pelas outras duas arestas mas ele não deve incluir a aresta do meio. Portanto, o caminho $P_i$ deve sair da linha comum da grade, contendo essas três arestas representativas, e retornar à mesma linha, assim requerendo no mínimo quatro dobras.
\qed


\begin{lemma}
\label{lemma:3par}
Seja $\cal{F}$ uma família minimal não-$(h-1)$-Helly de caminhos sobre uma grade, contendo três arestas paralelas, e tendo número de Helly $H(\cal{F})$   $\geq 4$. Então $\cal{F}$ deve conter caminhos com no mínimo quatro dobras. 
\end{lemma}

\proof
Uma vez que $H(\cal{F}) $ $\geq 4$ e $\cal{F}$ é uma $(h-1)$-família minimal, segue que $\cal{F}$ deve conter no mínimo quatro caminhos, $P_1,P_2,P_3,P_4$. Sem perda de generalidade, sejam $u_1,u_2,u_3$ as arestas não-representativas dos caminhos $P_1,P_2,P_3$ que são paralelas. Então $P_4$ deve passar por todas três arestas paralelas não-representativas $u_1,u_2,u_3$, assim requerendo no mínimo quatro dobras. 
\qed


\begin{lemma} \label{lemma:Lwit}
Seja $\cal{F}$ uma família minimal não-$(h-1)$-Helly de caminhos sobre uma grade com  número de Helly $H({\cal F}) \geq 4$. Se  ${\cal F}$ contem três arestas  não-representativas que se encontram sobre um mesmo $B_1$-caminho  $P_1$ de $\cal{F}$, então $\cal {F}$ deve possuir algum caminho com no mínimo três dobras. \end{lemma}

\proof
Visto que $\cal{F}$ é uma  $(h-1)$-família minimal tendo  número de Helly $\geq 4$, ela contem no mínimo quatro caminhos. Sem perda de generalidade, sejam  $u_1, u_2, u_3$ as três arestas não-representativas contidas em $P_4$ e tal que  $u_2$ encontra-se entre $u_1$ e $u_3$ em $P_4$. Então o caminho $P_2$ deve conter $u_1$ e $u_3$, mas evite $u_2$, assim requerendo no mínimo três dobras.  
\qed

A seguir são apresentados os limites superiores justos para os números de Helly de $B_k$-caminhos em arestas, para $k = 1,2,3$.

\begin{lema}\label{claim:upper-B1}
$H(B_1$-$EPG) \leq 3.$
\end{lema}
 
\proof
Assuma por contradição que o número de Helly de famílias de caminhos $B_1$-EPG é $h > 3$. Nesse caso, considere uma família minimal não-$(h-1)$-Helly $\cal F$ de caminhos $B_1$-EPG. Então $\cal F$ contem no mínimo  $h$ caminhos.  
Qualquer caminho $P_1 \in \cal{F}$ deve conter $h-1$ arestas não-representativas  correspondendo aos $h-1$ distintos caminhos de $\cal F$, diferentes de $P_i$. Uma vez que $h-1 \geq 3$, $P_1$  contem no mínimo três arestas  distintas não-representativas $u_2, u_3, u_4 \in P_1$, com $u_3$ encontrando-se entre $u_2$ e $u_4$ no caminho.  Se $P_1$ não possui dobras, $u_2,u_3,u_4$ são co-lineares. Pelo Lema~\ref{lemma:3colin},  o caminho $P_3 \in \cal{F}$ deve conter no mínimo quatro dobras. Se $P_1$ possui exatamente uma dobra, segue do Lema~\ref{lemma:Lwit} que $P_3$ possui três dobras. Em qualquer situação, uma contradição surge, implicando que $H({\cal F}) \leq 3$.
\qed

\begin{lema}\label{claim:upper-B2}
$H(B_2$-EPG$) \leq 4.$
\end{lema}

\proof
Assuma por contradição que o  número de Helly de famílias de caminhos  $B_2$-EPG é $h > 4$. Nesse caso, considere uma família minimal não-$(h-1)$-Helly  $\cal F$ de caminhos $B_2$-EPG. A família  $\cal F$ deve conter no mínimo $h \geq 5$ caminhos distintos, cada um deles correspondendo a uma aresta distinta  não-representativa. Selecione arbitrariamente 5 dessas arestas não-representativas.

Pelos Lemas~\ref{lemma:3colin} e ~\ref{lemma:3par} quaisquer dessas três arestas escolhidas não podem ser  co-lineares nem paralelas. Por consequência, no mínimo uma das 5 arestas não-representativas escolhidas deve estar em uma direção que é diferente da direção da maioria das outras arestas escolhidas. Chamemos a direção dessa aresta de vertical, e a direção da maioria das outras arestas de horizontal. Considere um caminho $P_1$ da família $\cal F$ que passa por essa aresta vertical. 
O caminho $P_1$ contem no mínimo quatro das arestas   não-representativas escolhidas, e no mínimo uma delas é vertical. Uma vez que $P_1$ possui no máximo duas dobras então ele deve ter no máximo três segmentos. Uma vez que temos três segmentos e quatro arestas não-representativas que $P_1$ deve conter, pelo princípio da casa dos pombos, um desses segmentos deve ter duas arestas não-representativas. Se esse par de arestas está em um segmento horizontal de $P_1$, então esse par de arestas, juntamente com a aresta vertical estão em dois segmentos consecutivos desse caminho, formando um $B_1$-subcaminho em $\cal F$. Então o Lema~\ref{lemma:Lwit} implica que algum caminho de $\cal F$ deve possuir no mínimo três dobras.  Caso contrário, as duas arestas seriam verticais. Mas as outras devem estar na  horizontal, e novamente temos no mínimo três arestas em um par de segmentos consecutivos formando um subcaminho em  $\cal F$ tendo uma dobra. Novamente, o  Lema ~\ref{lemma:Lwit} implica que algum caminho deve possuir no mínimo três dobras.
\qed

\begin{lema}\label{claim:upper-B3}
$H(B_3$-EPG$) \leq 8.$
\end{lema}

\proof
Assuma por contradição que o número de Helly de caminhos  $B_3$-EPG é $h > 8$. Nesse caso, considere uma família minimal não-$(h-1)$-Helly $\cal F$ de caminhos $B_3$-EPG. Então $\cal F$ contem no mínimo $h$  arestas não-representativas distintas,  correspondendo a  $h$ subconjuntos distintos. Pelo Lema~\ref{lemma:3par} já que podemos ter no máximo três dobras em qualquer caminho, então essas $h$ arestas  não-representativas devem encontrar-se em no máximo duas linhas vertical e horizontal da grade. Por esse motivo uma dessas quatro linhas possíveis deve conter pelo menos três arestas não-representativas distintas. Pelo  Lema~\ref{lemma:3colin},  isso implica na existência de um caminho com quatro dobras.\qed

Isso completa a prova do Teorema~\ref{thm:Helly-EPG}. 


\section{Número de Helly of $B_k$-VPG Graphs}\label{sec:helly-vpg}

Nessa seção determinaremos o número de Helly de grafos $B_k$-VPG. Para isso, provamos os seguintes resultados.
\begin{theorem}\label{thm:Bk-VPG}
O número de Helly para grafos $B_k$-VPG satisfaz:
\begin{enumerate}
\item $H(B_1$-VPG) = 4
\item $H(B_2$-VPG) = 6
\item $H(B_3$-VPG) = 12
\item $H(B_4$-VPG) é ilimitado.
\end{enumerate}
\end{theorem}

Novamente, a demonstração do teorema é efetuada pela apresentação de limites superiores e inferiores justos.

\subsection{Limites Inferiores}

Primeiro, apresentamos um conjunto de instâncias com caminhos que possuem no máximo $k$ dobras para delimitar um limite inferior para cada $k=1,2,3$.

A Figura~\ref{VPG:lower-B1} ilustra um conjunto de  quatro $B_1$-caminhos de um grafo $G$, em uma grade $2 \times 2$, tal que cada caminho cobre exatamente três vértices de  $G$, e evita exatamente um  dos  vértices. 

\begin{figure}[!h]
    \centering
    \includegraphics[width=3cm]{./img/lower-bound-B1-VPG.pdf}
    \caption{Limite inferior para grafos $B_1$-VPG.}
    \label{VPG:lower-B1}
\end{figure}

A Figura~\ref{VPG:lower-B2} ilustra um conjunto de seis $B_2$-caminhos de um grafo $G$, em uma grade  $2 \times 3$, tal que cada caminho cobre cinco vértices de $G$, e evita exatamente um.


\begin{figure}[!h]
    \centering
    \includegraphics[width=8cm]{./img/lower-bound-B2-VPG.pdf}
    \caption{Limite inferior para grafos $B_2$-$VPG$.}
    \label{VPG:lower-B2}
\end{figure}


A Figura~\ref{VPG:lower-B3} ilustra um conjunto de doze $B_3$-caminhos de um grafo $G$, em uma grade, de perímetro 12, tal que cada caminho cobre  onze vértices de $G$, evitando um deles.

\begin{figure}[!h]
    \centering
    \includegraphics[width=12cm]{./img/lower-bound-B3-VPG.pdf}
    \caption{Limite inferior para grafos $B_3$-VPG.}
    \label{VPG:lower-B3}
\end{figure}

A Figura~\ref{VPG:lower-B4} ilustra um conjunto de $n$ $B_4$-caminhos de um grafo $G$ com $n$-vértices, em uma grade que possui perímetro $n$,  tal que cada caminho cobre exatamente   $n-1$  vértices do perímetro da grade, e evitando um deles. 

\begin{figure}[!h]
    \centering
    \includegraphics[width=12cm]{./img/lower-bound-B4-VPG.pdf}
    \caption{Limite inferior para grafos $B_4$-VPG.}
    \label{VPG:lower-B4}
\end{figure}

Aplicando o Teorema~\ref{thm:minimal}, podemos então concluir que o número de vértices de cada um dos grafos descritos anteriormente são limites inferiores para as classes correspondentes. Então, podemos estabelecer os seguintes limites.

\begin{lema}\label{claim:VPG-lower}
Os seguintes são limites inferiores para $H(B_k$-VPG).
\begin{enumerate}
\item $H(B_1$-VPG) $\geq 4$
\item $H(B_2$-VPG) $\geq 6$
\item $H(B_3$-VPG) $\geq 12$
\item $H(B_4$-VPG) é ilimitado.
\end{enumerate}
\end{lema}

\subsection{Limites Superiores}

A seguir, provamos limites superiores para o número de Helly de grafos $B_k$-VPG. Os seguintes lemas são empregados.

\begin{lemma}\label{column-sizes}
Seja $\cal F$ uma família minimal não-$(h-1)$-Helly de caminhos, para algum $h$, contendo $k \in \{3,4,5\}$ distintos pontos co-lineares não-representativos da grade. Então $\cal F$ contem um caminho com pelo menos $k-1$ dobras.
\end{lemma}

\proof Para $k \in \{3,5\}$, o caminho que evita o ponto do meio possui pelo menos $k-1$ dobras; enquanto para $k = 4$ o caminho evitando um dos pontos do meio também possui essa mesma propriedade.
\qed

\begin{lemma}\label{column-number}
Seja $\cal F$ uma família minimal não-$(h-1)$-Helly de caminhos, sobre uma grade contendo $k < h$ pontos distintos mutuamente não-co-lineares não-representativos. Então $\cal F$ deve conter um caminho com no mínimo $k-1$ dobras.
\end{lemma}   

\proof Uma vez que $k < h$, $\cal F$ deve conter um caminho que visite todos esses $k$ pontos mutuamente não-co-lineares. Esse caminho exige pelo menos uma dobra, entre dois pontos consecutivos não-co-lineares. Portanto $\cal F$ contem um caminho com pelo menos $k-1$ dobras. \qed \\

Também empregamos alguns conceitos e notações adicionais, descritos abaixo.

Seja $\cal F$ uma família minimal não-$(h-1)$-Helly  de $B_{k-1}$-caminhos sobre uma grade $Q$. Pelo Teorema~\ref{thm:minimal},  podemos escolher $h$ caminhos $P_i \in {\cal F}$, cada um deles associado a um ponto $p_i$ distinto não-representativo da grade, tal que $P_i$ evita $p_i$, mas contem todos os outros  $h-1$ pontos distintos não-representativos  $p_j \in P_J$, para cada $j \neq i$. Denote por $P_N$, $|P_N|=h$, o subconjunto de pontos da grade $Q$, restritos ao conjunto escolhido de distintos pontos  não-representativo $p_i$. Pelos Lemas~\ref{column-sizes} e \ref{column-number}, os pontos da grade de $P_N$ estão contidos em no máximo $k$ colunas (linhas), e cada  coluna (linha) contem no máximo $k$ pontos de $P_N$. Consequentemente, as cardinalidades dos pontos de $P_N$, contidas na coluna (linha) de $Q$,  forma uma partição  do inteiro $h$, em no máximo $k$ partes, tal que cada  parte possui tamanho no máximo  $k$. Chamemos tal partição como uma {\it partição viável de $h$, relativa a $P_N$}. Portanto, cada ponto não-representativo $p_i \in P_N$ contribui com uma unidade para alguma parte da partição, que é então referido como a parte da partição  {\it correspondendo} a $p_i$.    

O seguinte lema descreve condições suficientes para um inteiro $h$ ser um limite superior do número de Helly.

\begin{lemma}\label{upper-bound} Seja $\cal F$ uma família minimal não-$(h-1)$-Helly de $B_{k-1}$-caminhos sobre uma grade $Q$, e $P_N$ o conjunto de pontos não-representativos de $Q$. Sejam $k,h$ inteiros, $1 \leq k \leq 3$ e $k < h$. As seguintes condições implicam que $H(B_k$-VPG) $\leq h$  
\begin{itemize}
    \item[(i)] Não existe partição viável de $h+1$, relativa a $P_N$, ou 
    \item[(ii)] Para qualquer possível partição viável, e para qualquer disposição dos pontos da grade de $P_N$ em $Q$, existe algum ponto não-representativo  $p_i \in P_N$, tal que não existe caminho em $Q$, tendo no máximo  $k$ dobras, contendo todos pontos de $P_N$, exceto $p_i$.    
\end{itemize}
\end{lemma}
\proof A prova de (i) segue diretamente dos Lemas~\ref{column-sizes} e \ref{column-number}, enquanto a prova de (ii) é uma consequência do Teorema~\ref{thm:minimal}.  \qed \\

Os seguintes são limites superiores para o número de Helly de grafos $B_k$-VPG, para cada $k$, $1 \leq k \leq 3$, obtido pela aplicação do Lema~\ref{upper-bound}.      
 
\begin{lema}\label{claim:upper-B1-VPG}
$H(B_1$-VPG) $\leq  4$.
\end{lema}

\proof Não há partição do número inteiro 5, em 2 partes, em que cada parte possua no máximo 2 elementos. Consequentemente, o resultado segue do Lema~\ref{upper-bound}~(i). \qed

\begin{lema}\label{claim:upper-B2-VPG}
$H(B_2$-VPG)  $\leq  6$.
\end{lema}

\proof Assuma o contrário. Então $H(B_2$-VPG) $\geq  7$, seja $\cal F$ uma família minimal não-$6$-Helly de $B_2$-caminhos, e  $P_N$ seja o conjunto de pontos  não-representativos de $\cal F$ in $Q$. Existem duas partições viáveis do inteiro 7, em três partes, cada uma delas de tamanho no máximo 3, nomeadamente $(3,3,1)$ e $(3,2,2)$. Em qualquer desses casos  é sempre possível selecionar algum ponto $p_i \in P_N$, pertencendo a uma parte da partição de tamanho 3, tal que um caminho em  $\cal F$  que evite $p_i$ e cubra os outros  6 pontos não-representativos, deve conter pelo menos  3 dobras.  Então pelo Lema~\ref{upper-bound}, de fato $H(B_2$-VPG)  $\leq  6$. \qed


\begin{lema}\label{claim:upper-B3-VPG}
$H(B_3$-VPG) $\leq  12$.
\end{lema}

\proof Assuma por contradição que $H(B_3$-VPG) $\geq  12$. Seja $\cal F$ uma família minimal não-$12$-Helly de $B_3$-caminhos, e $P_N$ seja o conjunto de pontos não-representativos de $\cal F$ em $Q$. Existem três possíveis partições do inteiro 13, em quatro partes, tal que cada parte tenha tamanho máximo 4, nomeadamente $(4,4,4,1)$, $(4,4,3,2)$ e $(4,3,3,3)$. Nesse caso, selecione $p_i \in P_N$ para ser um ponto  não-representativo, correspondendo a uma parte de tamanho $4$ da partição.  O caminho de ${\cal F}$, que evita $p_i$ deve cobrir os outros 12 pontos não-representativos. Esses pontos estão localizados em 4 distintas colunas, de cardinalidades 4,4,3,1, 4,3,3,2, ou 3,3,3,3, considerando as 3 possíveis partições, respectivamente. Esse caminho deve conter no mínimo 4 dobras, uma contradição. Então pelo Lema~\ref{upper-bound}, $H(B_3$-VPG) $\leq  12$.    \qed  

Dos limites inferiores e superiores descritos nas subseções anteriores, obtemos os resultados para o  número de Helly de grafos $B_k$-VPG, completando a prova do Teorema~\ref{thm:Bk-VPG}.

\section{Número de Helly forte}\label{sec:helly-forte}

Nesta seção, consideramos uma maneira de determinar o número de Helly forte de grafos  $B_k$-EPG.

Iniciamos por descrever um teorema similar ao Teorema~\ref{thm:minimal}.

\begin{theorem}\label{thm:minimal-strong}

Seja ${\cal C}$ uma classe de famílias hereditárias  $\cal F$ de subconjuntos do conjunto universal $U$, cujo número de Helly forte  $sH({\cal C})$ é igual a $h$. Então existe uma família ${\cal F'} \in {\cal C}$ com exatamente $h$ subconjuntos satisfazendo a seguinte condição: 

Para cada subconjunto $P_i \in \cal {F'}$, há exatamente um elemento distinto $u_i \in U$, tal que 
$$u_i \not \in P_i,$$ 
mas $u_i$ está condido em todos subconjuntos 
$$P_j \in {\cal F'} \setminus P_i.$$
\end{theorem}

\proof O número de Helly forte  de ${\cal C}$ é $h$ e não $h - 1$, de modo que deva existir alguma família ${\cal F} \in {\cal C}$ cujo  número de Helly forte  é exatamente $h$, i.e. $\cal F$  contem $h$ subconjuntos $P_i$ cuja intersecção é igual ao core($\cal F'$) mas é tal que nenhum de seus  $h-1$ subconjuntos possui a mesma intersecção. Em particular, seja $\cal F'$ a família contendo exatamente os  $h$ subconjuntos $P_i$ descritos anteriormente. Essa  família deve existir, uma vez que $\cal C$ é hereditária. Então cada  $P_i$ não contem pelo menos um elemento $u_i$ na intersecção dos  $h-1$ subconjuntos restantes $P_j$, $j \ne i$, 
uma vez que a intersecção desses $h-1$ subconjuntos não deve ser igual ao core($\cal F'$).  \qed

Mais uma vez, se considerarmos a  família $\cal F'$ descrita no teorema anterior, então é simples concluir que a remoção de qualquer subconjunto de $\cal {F'}$ a torna $(h-1)$-Helly-forte.  Então denotamos $\cal {F'}$ como uma família {\it minimal} não-$(h-1)$-Helly-forte. Além disso, o elemento $u_i \not \in P_i$, contido em todos subconjuntos $P_j \in {\cal{F'}} \setminus P_i$, exceto $P_i$, é o {\it $h$ não-representativo} de $P_i$.  

Como antes, empregaremos os subconjuntos de famílias minimais já estabelecidos, aplicados a caminhos em uma grade.

De fato, provaremos que o número de Helly forte  de grafos $B_k$-EPG  coincide com o número de Helly, para cada valor de  $k$ correspondente. Similarmente, para grafos $B_k$-VPG. Para $k=0$, é  simples mostrar que se um conjunto de intervalos  $\cal I$ em uma linha mutuamente intersectam-se então existem dois intervalos de $\cal I$ cuja intersecção é igual à intersecção de todos os intervalos de $\cal I$. Consequentemente o $k$-número de Helly forte  de grafos $B_0$-EPG é igual a 2. 
Similarmente, o mesmo vale para grafos $B_0$-VPG. 
Relembrando que o número de Helly forte é pelo menos igual ao número de Helly de uma família, de modo que os limites inferiores apresentados  no Lema~\ref{claim:lower-Bk-EPG} também são válidos para o  número de Helly forte . As provas para o número de Helly forte s para $k \geq 1$ são similares às já descritas na Seção~\ref{sec:Helly-number}.  



\section{Considerações finais}\label{sec:finalRemarks4}

Nesse capítulo, determinamos o  número de Helly e o número de Helly forte  de grafos $B_k$-EPG e de grafos $B_k$-VPG, para $k \geq 0$. Os limites inferiores foram determinados pela apresentação de instâncias de conjuntos que definiam uma família de caminhos dentro de cada configuração. Por outro lado, os limites superiores foram determinados por abordagens matemáticas/geométricas que levaram em consideração fatos como as posições relativas de uma sequência de arestas (vértices) dos caminhos ou as possíveis subdivisões em partições de cada um de seus respectivos subconjuntos.

A Tabela \ref{tab:Helly-Strong-Helly} sumariza os resultados obtidos.
 
\Large 

\begin{table}[htb]
    \centering
    \begin{tabular}{c|c|c}
    \cline{1-3} $k$  & $B_k$-EPG & $B_k$-VPG \\
    \cline{1-3} 0 & 2 & 2 \\
    \cline{1-3} 1 & 3 & 4 \\
    \cline{1-3} 2 & 4 & 6 \\
    \cline{1-3} 3 & 8 & 12 \\
    \cline{1-3} $\geq 4$ & ilimitado & ilimitado \\
    \cline{1-3} 
    \end{tabular}
    \caption{Número de Helly e número de Helly forte para grafos $B_k$-EPG e $B_k$-VPG.}
    \label{tab:Helly-Strong-Helly}
\end{table}

\normalsize


Levantamos duas questões para serem investigadas, em relação aos resultados apresentados nesse capítulo.

\begin{enumerate}
\item Dado um {\it grafo específico}  EPG ou VPG, a  questão seria de formular um algoritmo capaz de determinar seu número de Helly e seu número de Helly forte. Ver o trabalho de~\cite{dourado2008improved}, para exemplos de tais algoritmos aplicados a grafos em geral.

\item Os valores do número de Helly e do número de Helly forte que foram determinados nesse capítulo coincidem em todos os casos. Claramente, em geral esse não é o caso. Deixamos uma questão em aberto, a de encontrar as condições para as quais essa igualdade ocorre.
\end{enumerate}


% \begin{thebibliography}{99}
% \bibitem{asinowski2011string}
% A. Asinowski, E. Cohen, M. C. Golumbic, V. Limouzy, M. Lipshteyn and M. Stern. Electronic Notes in Discrete Mathematics 37 (2011), pp. 141-146. 

% \bibitem{asinowski2012}  A. Asinowski, E. Cohen, M. C. Golumbic, V. Limouzy, M. Lipshteyn, and M. Stern,
% Vertex intersection graphs of caminhos on a grid, Journal of Graph Algorithms and
% Applications, 16 (2012) pp. 129-150.

% \bibitem{bergeDuchet1975}
% C. Berge and P. Duchet. A generalization of Gilmore’s theorem, {\emph in} M. Fiedler,
% editor, Recent Advances in Graph Theory, Acad. Praha, Prague, 1975, pp. 49-55

% \bibitem{duchet1978propriete}
% P. Duchet. Propriet\'e de Helly et probl\`emes de repr\'esentations. In Colloquium
% International CNRS 260, Probl\'emes Combinatoires et Th\'eorie de Graphs, Orsay, France, 1976, pp. 117-118

%  \bibitem{dourado2008improved}
%  M. C. Dourado, M. C. Lin, F. Protti, and J. L. Szwarcfiter. Improved algorithms
%  for recognizing $p$-Helly and hereditary $p$-Helly hypergraphs. Information Processing Letters 108  (2008), pp. 257-250.

% \bibitem{dourado2008strong}
% M. C. Dourado, F. Protti, and J. L. Szwarcfiter. On the strong $p$-Helly property.
% Discrete Applied Mathematics, 156 (2008), pp. 1053–1057

% \bibitem{dourado2009}
% M. C. Dourado, F. Protti and J. L. Szwarcfiter,
% Complexity aspects of the Helly
% property: graphs and hypergraphs. Electronic Journal on Combinatorics, Dynamic Surveys 17, 2009 

% \bibitem{golumbic1985}
% M. C. Golumbic and R. E. Jamison,
% The edge intersection graphs of caminhos in a tree,
% Journal of Combinatorial Theory B 38 (1985), pp. 8-22.

% \bibitem{golumbic2009}
% M. C. Golumbic, M. Lipshteyn and  M. Stern,
% Edge intersection graphs of single dobra caminhos on a grid, Networks 54 (2009), pp. 130-138.

% \bibitem{golumbic2013}
% M. C. Golumbic, M. Lipshteyn and  M. Stern,
% Single dobra caminhos on a grid have número de Helly forte  4,
% Networks (2013), 161-163

%\bibitem{}
% M. C. Golumbic and G. Morgenstern,
% Edge intersection graphs of caminhos
% on a grid, {\emph in} ``50 Years of Combinatorics, Graph Theory and Computing'', F.~Chung, R.~Graham, F.~Hoffman, L.~Hogben, R.~Mullin, D.~West, eds, CRC Press, 2019, pp. 193-209. 
  \chapter{Relationship among  $B_1$-EPG, EPT and VPT graph classes}
\label{cap:v}

\begin{flushright}
\begin{minipage}[t][0cm][b]{0.47\textwidth}
\emph{
A Matemática não mente. Mente quem faz mau uso dela.}
\end{minipage}

\rule[0cm]{7cm}{0.03cm}%{largura}{espessura}

Albert Einstein
\end{flushright}

This chapter presents as a main result that every Chordal $B_1$-EPG graph is simultaneously in the VPT and EPT graph classes. In particular, we describe structures that are always present in graphs that do not support a Helly-$B_1$-EPG representation and thus we define some sets of subgraphs that delimit Helly subfamilies. 
 In addition, this chapter also features characterizations for some non-trivial graph families that are properly contained in Helly-$B_1$ EPG, namely these families are composed of the Bipartite, Blocks, Cactus and Line of Bipartite  graphs.


\section{Discussão Inicial}

Modelos baseados em intersecção de caminhos podem ser considerados basicamente sob dois pontos de vista, o primeiro considera as intersecções em vértices e o segundo em arestas. Casos onde os caminhos são hospedados em uma árvore aparecem primeiro na literatura, veja por exemplo~\cite{gavril1978recognition, golumbic1985edge, golumbic1985}.  Representações usando caminhos sobre uma grade foram considerados mais tarde, ver~\cite{golumbic2009,golumbic2013, golumbic2013intersection}. A seguir detalharemos os modelos de intersecção que serão estudados neste capítulo. %More details on each intersection model will be given in the following text.

 Seja $P$ uma família de caminhos sobre uma árvore hospedeira $T$. Dois tipos de grafos de intersecção do par  $<P,T>$ são definidos, denotaremos esses como grafos VPT e EPT.
Os \textit{Grafos de Intersecção de Arestas} de $P$, EPT(P), possuem  vértices que correspondem aos membros de $P$, e dois vértices são  adjacentes em EPT(P) se e somente se os caminhos correspondentes em $P$ compartilham pelo menos uma aresta em $T$. Similarmente, os \textit{Grafos de Intersecção de Vértices} de $P$, VPT(P), possuem vértices que correspondem aos membros de  $P$, e dois vértices são adjacentes em VPT(P) se e somente se os caminhos correspondentes em  $P$ compartilham pelo menos um vértice em $T$. Em qualquer dos modelos considerados, sempre existe uma função de bijeção entre as intersecções dos caminhos e as adjacências dos vértices correspondentes. 


Grafos VPT e EPT definem famílias incomparáveis de grafos. Entretanto, quando o grau máximo da árvore hospedeira é restrito a três, então a família dos grafos VPT coincide com a família dos grafos EPT~ \cite{golumbic1985edge% \cite{alcon2010necessary
}. Também é conhecido que qualquer grafo Cordal EPT é também um grafo VPT (see~\cite{syslo1985triangulated}). Recordando também que outro resultado conhecido na literatura é o de que os grafos Cordais são os grafos de intersecção em vértices de subárvores de uma árvore, ver~\cite{gavril1974intersection}.


\textit{Grafos de Intersecção de Caminhos sobre uma Grade} são chamados de \textit{grafos EPG}. 

Em \cite{golumbic2009}, o autor provou que todos os grafos possuem uma representação EPG, e iniciou o estudo das subclasses definidas pela limitação do número de vezes que qualquer caminho utilizado na representação pode dobrar. Grafos admitindo uma representação onde caminhos possuem no máximo  $k$ mudanças de direção (dobras) foram chamados de grafos $B_k$-EPG. 
 Em particular, quando os caminhos possuem no máximo uma dobra então temos os grafos \textit{ $B_1$-EPG}, também conhecidos como grafos EPG de  \textit{dobra simples}.

Uma questão pertinente no contexto de grafos de intersecção de caminhos é como segue: dadas duas classes de de grafos de intersecção de caminhos,  a primeira cujo hospedeiro é uma árvore e a segunda cujo hospedeiro é uma grade, existe uma intersecção ou um relacionamento de continência entre essas classes? O que sabemos sobre isso?

Nesse capítulo iremos explorar os grafos  $B_1$-EPG, em particular os grafos diamante-livre e os grafos Cordais. Trabalharemos principalmente sobre questões a respeito da relação de contenção entre as classes de grafos VPT, EPT e $B_1$-EPG.

Uma coleção de conjuntos satisfaz a   \textit{propriedade Helly} quando toda subcoleção que é mutuamente intersectante possui pelo menos um elemento em comum. Quando essa propriedade é satisfeita pelo conjunto de vértices (ou arestas) dos caminhos utilizados na representação, então temos uma representação Helly. Grafos   $B_1$-EPG-Helly foram estudados em~\cite{bornstein2019}, que mostraram que todo grafo admite uma representação Helly e também que o problema de reconhecer grafos $B_1$-EPG-Helly é $NP$-completo.  

É conhecido que nem todo grafo $B_1$-EPG admite uma representação  $B_1$-EPG-Helly. Estamos interessados em determinar os subgrafos que são grafos
$B_1$-EPG porém não admitem uma representação  Helly $B_1$-EPG. Nesse capítulo, descrevemos estruturas que estarão presentes em qualquer desses subgrafos, além do mais, também apresentamos novas subclasses de grafos  $B_1$-EPG-Helly. Ademais, delimitamos novas subclasses $B_1$-EPG-Helly e damos alguns  conjuntos de subgrafos que delimitam subfamílias Helly.
\section{Definições e Resultados Técnicos}

O \textit{conjunto de vértices} e o \textit{conjunto de arestas} de um grafo  $G$ são denotados por $V(G)$ e $E(G)$, respectivamente.  Dado um vértice  $v\in V(G)$,  $N(v)$ e $N[v]$ representam a   \textit{vizinhança} aberta e fechada de  $v$ em $G$, respectivamente. 
Para um subconjunto  $S \subseteq V(G)$,  $G[S]$ é o subgrafo de $G$ induzido por $S$.
 Se $\mathcal{F}$ é uma família qualquer de grafos, dizemos que $G$ é  \textit{$\mathcal{F}$-livre} se $G$ não possui subgrafo induzido isomorfo a um membro de $\mathcal{F}$.
 Um \textit{ciclo},  denotado por $C_n$, é uma sequência de vértices distintos  $v_1, \dots , v_n, v_1$  onde $v_i \neq v_j$ for $i \neq j$ e $(v_i, v_i + 1) \in E(G)$, tal que $n \geq 3$. Uma \textit{corda} é uma aresta que está entre dois vértices não-consecutivos em uma sequência de vértices de um ciclo. Um \textit{ciclo induzido}  ou \textit{ciclo sem corda} é um ciclo que não possui corda, nesse capítulo todo ciclo induzido será chamado simplesmente de  \textit{ciclo}. Um grafo  $G$ formado por um ciclo induzido $H$ mais um único vértice universal  $v$ conectado a todos vértices de $H$ é chamado  \textit{grafo roda}. Se o grafo roda possui $n$ vértices, ele é denotado por $n$-roda. 

O grafo $k$\textit{-sol} $S_k$, $k \geq 3$, consiste de 
$2k$ vértices, um conjunto independente $X = \{x_1, \dots, x_k\}$ e uma clique $Y = \{y_1, \dots, y_k\}$, e um conjunto de arestas $E_1 \cup E_2$, onde $E_ 1=\{ (x_1,y_1); (y_1, x_2); (x_2, y_2); (y_2, x_3); \dots , (x_k, y_k); (y_k, x_1) \}$ forma o ciclo externo e $E_2= \{(y_i, y_j) |i\neq j\}$ forma uma clique interna.

Um grafo é $ B_k$-EPG se ele admite uma representação EPG em que cada caminho possui no máximo $k$ dobras. Quando $ k = 1 $ dizemos que essa é uma representação \emph{ EPG de dobra simples} ou simplesmente uma representação  $B_1$-EPG. 
Uma \textit{clique} é um conjunto de vértices mutuamente adjacentes, enquanto
um \textit{conjunto independente} é um conjunto de vértices não mutuamente adjacentes entre si.
 Dada uma representação EPG de um grafo  $G$, identificaremos cada vértice  $v$ de $G$ com o correspondente caminho  $P_{v}$ da grade, utilizado na representação. Adequadamente, por exemplo, diremos que um vértices de  $G$ cobre ou contem alguma aresta da grade (significando que o correspondente caminho faz isso), ou que o conjunto de caminhos da representação induz um subgrafo de $G$ (significando que o correspondente conjunto de vértices faz isso, na verdade). 

Em uma representação $B_1$-EPG, uma clique $K$ é dita ser uma \textit{clique-aresta} se todos os vértices de $K$ compartilham pelo menos uma aresta comum da grade (ver Figura~\ref{fig:cliquesRepresentation}(a)).
 Uma \textit{garra da grade} é um conjunto de três arestas da grade, incidentes no mesmo ponto da grade, que é chamado o \textit{centro da garra}. As duas arestas da garra que tem a mesma direção formam a  \textit{ base da garra}. Se $K$ não é uma clique-aresta, então existe uma garra da grade   (e somente uma) tal que os vértices de  $K$ são aqueles que contem exatamente duas das três arestas dessa garra; tal clique é chamada de   \textit{clique-garra} \cite{golumbic2009} (ver Figura~\ref{fig:cliquesRepresentation}(b)).

    

\begin{figure}[h]
  \centering
  \begin{tabular}{  p{4cm} p{0.7cm} p{4cm} }
    %\centering
    \includegraphics[width=4.5cm]{img/edge-clique.png} & &
    \includegraphics[width=3.5cm]{img/claw-clique.png}%b1EpgTransparenteGrade2
    \\
    \footnotesize %\centering 
    (a)  \footnotesize Representação de uma  clique como clique-aresta. && \footnotesize (b) Representação de uma  clique como clique-garra.\\
  \end{tabular}

 \caption{Exemplos de representações de clique.} \label{fig:cliquesRepresentation}
\end{figure}    

Note que se três vértices induzem uma clique-garra, então exatamente dois deles dobram no centro da garra correspondente na grade, e o terceiro contem a base da garra.
Além disso, qualquer outro vértice adjacente aos três deve conter duas das arestas dessa garra, então o seguinte lema é válido.

\begin{lema}\label{lem:cliquesMaximais}
Se três vértices estão juntos em mais que uma  clique maximal do grafo $G$, então em qualquer representação $B_1$-EPG de $G$ esses três vértices não formam uma  clique-garra.
\end{lema}

 
\begin{figure}[ht]
  \centering
  \begin{tabular}{  p{5cm} p{0.7cm} p{5cm} }
    %\centering
    \includegraphics[width=3.5cm]{img/lemaClaw2Maximais} & &
    \includegraphics[width=5.5cm]{img/claw2}
    \\
    \footnotesize %\centering 
    (a)  \footnotesize Example of two maximal cliques sharing vertices. && \footnotesize (b) Representation  of a claw-clique in grid.\\
  \end{tabular}

 \caption{Vertices represented by a claw are present in a unique maximal clique.} \label{fig:lemaClaw2Maximais}
\end{figure}

% \begin{figure}[ht]
%   \centering
%   \begin{tabular}{  p{5cm} p{0.7cm} p{5cm} }
%     %\centering
%     \includegraphics[width=3.5cm]{img/lemaClaw2Maximais} & &
%     \includegraphics[width=5.5cm]{img/claw2}
%     \\
%     \footnotesize %\centering 
%     (a)  \footnotesize Examplo de duas cliques maximais compartilhando vértices. && \footnotesize (b) Representação de uma clique-garra na grade.\\
%   \end{tabular}

%  \caption{Vértices representados por uma garra  estão presentes em uma única clique maximal.} \label{fig:lemaClaw2Maximais}
% \end{figure}

Em Asinowski et al. \cite{ries2009} foi provado o seguinte lema para grafos $C_4$-livre.

\begin{lema} \cite{ries2009} \label{lem:lemaBRies2009}
Seja $G$ um grafo $B_1$-EPG. Se $G$ é $C_4$-livre, então existe uma representação $B_1$-EPG de $G$ tal que toda clique-garra maximal $K$ é representada sobre uma  garra da grade cuja base é coberta unicamente por vértices de $K$.
\end{lema}


Temos obtido o seguinte resultado, similar para grafos diamante-livre. Um \textit{diamante} é um grafo  $G$ com conjunto de vértices  $V(G) = \{a, b, c, d\}$ e conjunto de arestas $E(G)=\{ab, ac,bc, bd,cd\}$ (ver Figura~\ref{fig:diamond}). %A graph is diamante-livre if it does not contain a diamante as induced subgraph.

 \begin{figure}[htb]	
 \center%6.3
 \includegraphics[width=2.2cm]{./img/diamond.png}
 \caption{Diamond graph.}
\label{fig:diamond}
\end{figure}  
 


\begin{lema}\label{lem:b1epgDiamondFree}
Seja $G$ um grafo $B_1$-EPG. Se $G$ é diamante-livre, então em qualquer representação $B_1$-EPG de $G$,  toda  clique-garra maximal $K$ é representada sobre uma garra da grade cujas arestas são cobertas somente por  vértices de $K$.
\end{lema}

\begin{proof}Seja $K$ uma  clique maximal que é uma clique-garra em uma dada representação $B_1$-EPG de $G$. Então existem três  vértices de $K$ que induzem uma  clique-garra $K'$ sobre a mesma garra da grade que $K$. Assuma, de forma a derivar uma contradição, que um vértice $v\notin K$ cobre alguma aresta da garra. Claramente, $v$ deve cobrir somente uma das arestas. Portanto $v$ e os vértices de $K'$ induzem um  diamante, uma contradição.
\end{proof}


% \begin{defi} \label{defi:tortasFrame}

Seja $ Q $ uma grade e sejam $ (a_1, b),$ $(a_2, b),$ $(a_3, b),$ $(a_4, b)$ uma $4$-estrela centrada em $b$ como ilustrado na Figura~\ref{fig:piesInGrid}(a). Seja $ \mathcal{P} = \{P_1, \dots , P_4\}$ uma coleção de quatro caminhos cada um contendo um diferente par de arestas da $4$-estrela.
%exactly two edges of the $4$-estrela:
Seguindo \cite{golumbic2009}, dizemos que esses quatro caminhos formam:

\begin{itemize}
\item uma \emph{torta verdadeira}  quando cada um possui uma dobra em  $b$, Figura~\ref{fig:piesInGrid}(b); e
\item uma \emph {torta falsa} quando exatamente dois dos caminhos dobram em  $b$ e eles não compartilham aresta da $4$-estrela, Figura~\ref{fig:piesInGrid}(c). %contain bends, while the remaining two do not share an edge. 

\begin{figure}[htb]
  \centering
%segundo bloco de figuras
  \begin{tabular}{c c c c c }
    \includegraphics[width=3.5cm]{img/disposicaoTortaGrid3.pdf}    
    & &\includegraphics[width=3.5cm]{img/truePieGrid} 
    & &
 \includegraphics[width=3.5cm]{img/falsePieGrid} \\%[\abovecaptionskip]
    {\footnotesize (a) 4-estrela em uma grade.}  & &  {\footnotesize (b) Torta verdadeira.} & & {\footnotesize (c) Torta falsa.} 
  \end{tabular}
  \caption{Representação $B_{1}$-EPG do ciclo induzido de tamanho  4 como tortas, com ênfase no centro $b$.}\label{fig:piesInGrid}
\end{figure} 

%\vspace{-0.5cm}
\end{itemize}
% \end{defi}

Claramente se quatro caminhos de uma representação $B_1$-EPG de $G$ formam uma torta, então os vértices correspondentes induzem um $4$-ciclo em $G$. % The converse implication is also true (see~\cite{golumbic2009}). 
O seguinte resultado pode ser facilmente provado. Dizemos que um conjunto de caminhos forma uma garra quando cada par de arestas da  garra está coberto por algum dos caminhos.

\begin{lema}\label{lem:twogarraNotSameCenterInCordal}
Em qualquer representação $B_1$-EPG de um grafo $G$, um conjunto de caminhos formando duas diferentes garras centradas no mesmo ponto da grade contem quatro caminhos formando também uma torta verdadeira ou uma torta falsa. Portanto, em qualquer representação $B_1$-EPG de um grafo Cordal $G$, não existem duas  clique-garras maximal  $G$ centradas no mesmo ponto da  grade.
\end{lema}

\begin{lema}\label{lem:3cliquesNotgarra}
Seja $G$ um grafo cujo conjunto de vértice pode ser particionado em uma clique não trivial $K$ e um conjunto independente $I=\{w_1,w_2,w_3\}$, tal que cada vértice de $K$  é adjacente a cada vértice de $I$. Então, em qualquer representação $B_1$-EPG de $G$,  pelo menos uma das cliques  $K_i = K \cup \{w_i\}$, com $1 \leq i \leq 3$,  é uma clique-aresta. 
\end{lema}

\begin{proof}
Assuma, de forma a derivar uma contradição, que as três cliques são cliques-garra. Pelo Lema~\ref{lem:twogarraNotSameCenterInCordal}, eles possuem diferentes centros. Digamos que os centros dessas cliques-garra sejam os pontos $q_1, q_2, q_3$ da grade, respectivamente. Uma vez que pelo menos dois caminhos possuem uma dobra no centro de uma garra, para cada $i\in\{1,2,3\}$,   deve existir um vértice
  $v_i$ de $K$ tal que o caminho correspondente $P_{v_i}$ dobra no ponto $q_i$ da grade.  Note que cada um dos três caminhos $P_{v_i}$ devem conter os três pontos da grade,  $q_1$, $q_2$ e $q_3$. Para provar que isso não é possível, consideraremos, sem perda de generalidade, dois casos.
  No primeiro,  $q_1$ está entre $q_2$ e $q_3$ em $P_{v_1}$. Então, $P_{v_3}$ não pode dobrar em $q_3$ e conter $q_1$ e $q_2$.   No segundo caso,
  $q_2$ está entre $q_1$ e $q_3$ em $P_{v_1}$. Dessa forma, $P_{v_2}$ não pode dobrar em  $q_2$ e conter $q_1$ e $q_3$; assim a prova está completa.
\end{proof}

Três vértices $u, v, w$ de um grafo $G$ formam uma \textit{ tripla asteroidal } (AT) de  $G$ se para todo par deles existe um caminho conectando esses dois  vértices e tal que o caminho evita a vizinhança do vértice remanescente~\cite{Asinowski2009}. Um grafo sem uma tripla asteroidal é chamado \textit{AT-livre}. 

\begin{lema}
[\cite{ries2009}] \label{l:AT-livre} Seja $v$ qualquer vértice de um grafo $B_1$-EPG  $G$. Então $G[N(v)]$ é AT-livre.
\end{lema}

Seja $C$ qualquer subconjunto dos vértices de um grafo $G$. O \textit{grafo branch} $B(G|C)$, ver~\cite{golumbic2009}, de $G$ sobre $C$ possui um conjunto de vértices, $V(B)$, consistindo de todos os vértices de $G$ que não pertencem a $C$ mas são adjacentes a algum membro de  $C$, i.e. $V(B) = N(C) - C$. As adjacência em $B(G|C)$ são definidas como segue: unimos dois vértices $x$ e $y$ por uma aresta em  $E(B)$ se e somente se em $G$ ocorre:
\begin{enumerate}
    \item  $x$ e $y$ são não adjacentes;
    \item $x$ e $y$ possuem um vizinho comum $u \in C$;
    \item os conjuntos $N(x) \cap C$ e $N(y) \cap C$ são incomparáveis, i.e. existem vizinhos privados $w, z \in C$ tal que $w$ é adjacente a $x$ mas não a $y$, e $z$ é adjacente a $y$ mas não é adjacente a $x$; dizemos que $x$ e $y$ são incomparáveis por vizinhança.
\end{enumerate}

Um grafo $G$ é \textit{k-colorável} se seus vértices podem ser coloridos com no máximo $k$ cores de forma que vértices adjacentes não compartilhem da mesma cor.

\begin{lema}[~\cite{golumbic2009}] \label{l:branch} Seja $C$ qualquer clique maximal de um grafo $B_1$-EPG $G$. Então, o grafo branch $B(G|C)$ é $\{P_6, \, C_n \hbox{ para }  n\geq 4\}$-livre.
\end{lema}





\section{Subclasses de Grafos $B_1$-EPG-Helly}

Nessa seção, delimitaremos algumas subclasses de grafos  $B_1$-EPG que admitem uma representação $B_1$-EPG-Helly. É conhecido que as classes de grafos $B_1$-EPG e $B_1$-EPG-Helly são classes hereditárias, assim elas podem ser reconhecidas por um conjunto de estruturas proibidas. 
Em ambos os casos, encontrar a lista de subgrafos induzidos proibidos que delimita essas classes é um desafiante problema em aberto. 
Dando um passo em direção a solucionar esses problemas, essa seção apresenta algumas estruturas onde pelo menos uma delas estará necessariamente presente em qualquer grafo $B_1$-EPG que não admita uma representação $B_1$-EPG-Helly. Além disso, mostramos que as já bem conhecidas famílias de grafos Bloco, Cactus e Linha de Bipartido estão totalmente contidas na classe dos grafos  $B_1$-EPG-Helly.


Sejam $S_{3}, S_{3'}, S_{3''}$ e $ C_{4}$ os grafos ilustrados na  Figura~\ref{fig:proibidos}. 


\begin{theorem}
\label{lem:CordalDiamondFree}
Seja $G$ um grafo $B_1$-EPG. Se $G$ é  $\{S_{3}, S_{3'}, S_{3''}, C_{4}\}$-livre então $G$ é um grafo $B_1$-EPG-Helly.
\end{theorem}

\begin{proof}
Se $G$ não é um grafo $B_1$-EPG-Helly, então em qualquer representação $B_1$-EPG de $G$, existe pelo menos uma  clique que é representada como clique-garra e não como  clique-aresta. Considere qualquer representação $B_1$-EPG  de $G$  e seja $K$ uma clique maximal  que é representada como uma clique-garra. Assuma, sem perda de generalidade,  que $K$ está sobre uma garra da grade com base $[x_0, x_2]\times\{y_0\}$ e centro $C = (x_1, y_0)$. Denote por   $\mathcal{P}_K$ o conjunto de caminhos correspondendo aos  vértices de $K$. Pelo Lema~\ref{lem:lemaBRies2009}, os segmentos $[x_0, x_2]\times\{y_0\}$ da grade são cobertos somente por vértices de $K$. % because $G$ is $C_4$-livre


 Para todo ${\displaystyle \lrcorner}$-caminho %$P_v \in \mathcal{P}_K$ 
 (resp. ${\displaystyle \llcorner}$-caminho 
% $P_{v'} \in \mathcal{P}_K$
 ) pertencendo a $\mathcal{P}_K$, fazemos o seguinte: se o caminho não intersecta nenhum caminho $P_t \notin\mathcal{P}_K$ sobre a  coluna $x_1$, então deletamos seus segmento vertical e adicionamos o segmento  $[x_1, x_2]\times\{y_0\}$ (resp. $[x_0, x_1]\times\{y_0\}$) à grade. Se depois dessa transformação não existe mais  ${\displaystyle \lrcorner}$-caminhos (resp. ${\displaystyle \llcorner}$-caminhos) em $\mathcal{P}_K$, então temos efetuado a correção corretamente e obtemos uma clique-aresta. Assim podemos assumir que todo  ${\displaystyle \lrcorner}$-caminho   e todo ${\displaystyle \llcorner}$-caminho  em $ \mathcal{P}_K$ intersecta algum caminho $P_t \notin \mathcal{P}_K$ sobre a  coluna $x_1$ (note que podemos assumir que este é o mesmo  caminho $P_t$ para todos os vértices vértices). 
 
 Agora, se nenhum dos  ${\displaystyle \lrcorner}$-caminhos pertencendo a $\mathcal{P}_K$ intersecta um caminho que não está em  $ \mathcal{P}_K$ sobre a linha $y_0$, então podemos substituir a parte  horizontal desses caminhos pelo segmento $[x_1,x_2]\times \{y_0\}$, obtendo uma representação por clique-aresta da clique $K$. Assim, podemos assumir que existe pelo menos um  ${\displaystyle \lrcorner}$-caminho $P_{v} \in \mathcal{P}_K$ intersectando algum caminho  $P_{t'} \notin \mathcal{P}_K$ sobre a linha $y_0$. Analogamente, existe no mínimo um  ${\displaystyle \llcorner}$-caminho $P_{v'} \in \mathcal{P}_K$ intersectando algum caminho $P_{t''} \notin K$ sobre a linha $y_0$, como ilustrado na Figura~\ref{fig:clawGrid}. Note que o vértice $t'$ não pode ser adjacente a qualquer dos vértices $t$, $v'$ ou $t''$; e, além disso, o vértice $t''$ não pode ser adjacente a $t$,  ou $v$.
 
 Finalmente,   uma vez que $K$ é clique-garra,  existe um caminho $P_u \in \mathcal{P}_K$ cobrindo a base da garra. Dependendo das adjacências possíveis entre  $u$ e $t'$ ou   $t''$, um dos grafos  $S_{3}$, $S_{3'}$ ou $S_{3''}$ é obtido.
\end{proof}



\begin{figure}[h]
  \centering
  \begin{tabular}{  c p{0.7cm} c}
    %\centering
    \includegraphics[width=5.5cm]{img/clawGrid} & &
    \includegraphics[width=3.5cm]{img/clawInduced.png}
    \\
    \footnotesize %\centering 
    (a)  \footnotesize Claw with paths. && \footnotesize (b) Subgraph induced by paths.\\
  \end{tabular}

 \caption{Reconstruction of the intersection model.}
 \label{fig:clawGrid}
\end{figure} 

 


% \begin{figure}[h]
%   \centering
%   \begin{tabular}{  c p{0.7cm} c}
%     %\centering
%     \includegraphics[width=5.5cm]{img/clawGrid} & &
%     \includegraphics[width=3.5cm]{img/clawInduced.png}
%     \\
%     \footnotesize %\centering 
%     (a)  \footnotesize Clique-garra com caminhos adicionais. && \footnotesize (b) Subgrafo induzido pelos caminhos.\\
%   \end{tabular}

%  \caption{Reconstrução do modelo de intersecção.}
%  \label{fig:clawGrid}
% \end{figure} 

 



Note  que qualquer grafo toro-livre é $\{S_{3}, S_{3'}, S_{3''}\}$-livre, assim nosso resultado anterior implica no Lema 5 de~\cite{ries2009}.


\begin{figure}[h]
  \centering
  \begin{tabular}{  c p{0.7cm} c }
    \centering
    \includegraphics[width=4cm]{img/s3.png} & &
    \includegraphics[width=4cm]{img/s3-1.png}
    \\
    \footnotesize \centering 
    (a)  \footnotesize Grafo $S_3$. &&  \footnotesize (b) Grafo $S_{3'}$. \\
    
    %---------------------
      \centering 
      \includegraphics[width=4cm]{img/s3-2.png} & &
    \includegraphics[width=3cm]{img/c4.png}
    \\
    \footnotesize \centering 
    (c)  \footnotesize Grafo $S_{3''}$. && \footnotesize (b) Grafo $C_{4}$.\\
  \end{tabular}

 \caption{Grafos do enunciado do  Teorema~\ref{lem:chordalDiamondFree}.}
 \label{fig:proibidos}
\end{figure} 



%--------------------------


O próximo teorema possui como consequência a identificação de algumas classes de grafos onde a existência de uma  representação $B_1$-EPG implica na existência de uma representação  $B_1$-EPG-Helly.


\begin{theorem} \label{lem:b1DiamondFree}
 Se $G$ é um grafo $B_1$-EPG e diamante-livre então $G$ é um grafo $B_1$-EPG-Helly.
 \end{theorem}

\begin{proof}
Se $G$ não é um grafo $B_1$-EPG-Helly, então em cada representação $B_1$-EPG de $G$, existe pelo menos uma clique que é representada como clique-garra e não como clique-aresta.  Considere qualquer representação $B_1$-EPG de $G$  e seja $K$ uma clique maximal que é representada como uma clique-garra. Assuma, sem perda de generalidade, que $K$ está sobre uma garra da grade com base $[x_0, x_2]\times\{y_0\}$ e centro $C = (x_1, y_0)$. Denote por  $\mathcal{P}_K$ o conjunto de caminhos correspondendo aos vértices de $K$. 
 Pelo Lema~\ref{lem:b1epgDiamondFree},  %(see~\cite{ries2009})
%no path $P_w$ for $w\notin K$ covers 
o segmento da grade $[x_0, x_2]\times\{y_0\}$ é coberto somente por vértices de $K$. % because $G$ is $C_4$-livre
 Para todo  ${\displaystyle \lrcorner}$-caminho %$P_v \in \mathcal{P}_K$ 
 (resp. ${\displaystyle \llcorner}$-caminho 
% $P_{v'} \in \mathcal{P}_K$
 ) pertencente a  $\mathcal{P}_K$, fazemos o seguinte: se %$P_v$ (resp. $P_{v'}$)
 o caminho não intersecta qualquer caminho $P_t \notin\mathcal{P}_K$ sobre a coluna $x_1$, então deletamos seu segmento vertical e adicionamos na grade o segmento $[x_1, x_2]\times\{y_0\}$ (resp. $[x_0, x_1]\times\{y_0\}$). Se depois dessa transformação não existir mais nenhum ${\displaystyle \lrcorner}$-caminhos (resp. ${\displaystyle \llcorner}$-caminhos) em $\mathcal{P}_K$, então terminamos, uma vez que temos obtido uma clique-aresta. Agora, podemos assumir que todo  ${\displaystyle \lrcorner}$-caminho   e todo ${\displaystyle \llcorner}$-caminho em $ \mathcal{P}_K$ intersecta algum caminho $P_t \notin \mathcal{P}_K$   sobre a coluna $x_1$ (note que podemos assumir que é o mesmo caminho $P_t$ para todos os vértices). Uma vez que $K$ é clique-garra, existe um caminho $P_u \in \mathcal{P}_K$ cobrindo a base da garra. Assim, $G[v, v', u, t]$ induz um diamante, uma contradição.
\end{proof}  

Um \textit{conjunto independente} de vértices é um conjunto de vértices não dois a dois adjacentes.
Um grafo $G$ é dito ser \textit{Bipartido} se seu conjunto de vértices pode ser particionado em dois conjuntos independentes distintos.
 Existem grafos Bipartidos que não são grafos $B_1$-EPG, por exemplo o grafo $K_{2,5}$ e $K_{3,3}$ (veja explicação em~\cite{cohen2014}). Claramente, uma vez que grafos Bipartidos são triângulo-livre, qualquer representação $B_1$-EPG de um grafo Bipartite é também uma representação $B_1$-EPG-Helly.
 Um resultado similar (mas um pouco mais fraco) é obtido como corolário do teorema anterior.


\begin{corollary}
Se $G$ é um grafo Bipartido $B_1$-EPG então $G$ é um grafo $B_1$-EPG-Helly.
\end{corollary}

\begin{proof}
Os grafos Bipartidos são diamante-livre, assim pelo Teorema~\ref{lem:b1DiamondFree} esses grafos são grafos $B_1$-EPG-Helly.
\end{proof}

Um \textit{grafo Bloco } ou \textit{grafo Árvore Clique} é um tipo de grafo em que toda componente biconexa (bloco) é uma clique.

\begin{corollary}\label{lem:cdf}
 Grafos Bloco são $B_1$-EPG-Helly.
\end{corollary}

\begin{proof}
Grafos Bloco são conhecidos por corresponderem exatamente à classe de grafos Cordal diamante-livre, assim pelo Teorema 19 de \cite{ries2009}, todos grafos Bloco são grafos  $B_1$-EPG. Segue do Teorema~\ref{lem:b1DiamondFree} que todos grafos Bloco estão na classe de grafos $B_1$-EPG-Helly. 
\end{proof} 

Um \textit{grafo Cactus} (algumas vezes chamado também de Árvore Cactus) é um grafo conexo em que quaisquer dois ciclos possuem no máximo um vértice em comum. Equivalentemente, ele é um grafo conexo em que toda aresta pertence a no máximo um ciclo, ou (para um cactus não trivial) em que cada bloco (subgrafo maximal sem vértice de corte) é uma aresta ou um ciclo. A família de grafos em que cada componente é um Cactus é fechada sob operações menores de grafos. Essa família de grafos pode ser caracterizada por um único subgrafo proibido menor, o grafo diamante.
 
\begin{corollary}
Grafos Cactus são  $B_1$-EPG-Helly.
\end{corollary}
\begin{proof}
Em~\cite{cela2019monotonic} foi provado que todo grafo Cactus graph é um grafo  $B_1$-EPG monotônico 
(existe uma representação $B_1$-EPG onde todos os caminhos são ascendentes em linhas e colunas). 
Assim, grafos Cactus são grafos $B_1$-EPG. 

Uma vez que Cactus são diamante-livre, pelo Teorema~\ref{lem:b1DiamondFree}, a prova segue verdadeira.
\end{proof}

Dado um grafo $G$, seu \textit{grafo Linha} $L(G)$ é um grafo tal que cada vértice de $L(G)$ representa uma aresta de  $G$ e dois  vértices de $L(G)$ são adjacentes se e somente se suas arestas correspondentes compartilham um vértice comum (i.e. elas ``são incidentes'' em um mesmo vértice) em $G$.  
Um grafo $G$ é um \textit{grafo Linha de um grafo Bipartido} (ou simplesmente \textit{Linha de Bipartido}) se e somente se ele não contem  garra, nem ciclo ímpar, e nem diamante como subgrafo induzido, \cite{harary1974line}.



\begin{corollary}\label{coro:lineOfBipartido}
 Grafos Linha de Bipartido são $B_1$-EPG-Helly. 
\end{corollary}

\begin{proof}
Grafos Linha de Bipartido foram provados ser $B_1$-EPG in~\cite{golumbic2018edge}. Uma vez que eles são diamante-livre, a prova segue diretamente pelo Teorema~\ref{lem:b1DiamondFree}.
\end{proof}

O diagrama da Figura~\ref{fig:diagram}
ilustra o relacionamento de continência entre as classes de grafos estudadas até agora neste trabalho.  
Listamos na Figura~\ref{fig:exemplosDiagram} exemplos de grafos em cada região numerada do diagrama. Os números de cada item abaixo correspondem às respectivas regiões com mesmo número do diagrama ilustrado na Figura~\ref{fig:diagram}.

%This numbers correspond with the respective number item and in some cases we make a brief explanation.

 \begin{figure}[htb]	
 \center%6.3
 \includegraphics[width=8cm]{./img/diagram.pdf}
 \caption{Diagrama de algumas classes de grafos.}
\label{fig:diagram}
\end{figure}  
 

\begin{enumerate}[label=(\arabic*)]
    \item Grafos $B_1$-EPG  - $B_1$-EPG-Helly, ilustrado na Figura~\ref{fig:exemplosDiagram}(a), grafo $E_1$;%1
    
    \item Grafos Linha de Bipartido  - Cactus - Bloco - Bipartido, ilustrado na Figura~\ref{fig:exemplosDiagram}(b), grafo $E_2$;%2
    \item Grafos $B_1$-EPG-Helly - Linha de Bipartido - Bloco - Cactus - Bipartido, ilustrado na Figura~\ref{fig:exemplosDiagram}(c), grafo $E_3$;%3
    \item Grafos Bloco $\cap$ Linha de Bipartido - Cactus - Bipartido, ilustrado na Figura~\ref{fig:exemplosDiagram}(d), grafo $E_4$;%4
    \item Grafos Bloco $\cap$ Linha de Bipartido $\cap$  Cactus - Bipartido, ilustrado na Figura~\ref{fig:exemplosDiagram}(e), grafo $E_5$;%5
    \item Grafos Cactus $\cap$ Linha de Bipartido - Bloco - Bipartido. Essa intersecção é vazia. Seja $G$ um grafo que é um Cactus e Linha de Bipartido então $G$ é $\{$garra, ciclo ímpar, diamante$\}$-livre. Mas $G$ não é um grafo Bipartido, então $G$ possui um ciclo ímpar, o que é um absurdo considerando a hipótese de que $G$ é Linha de Bipartido; %Dessa forma $G$ possui no mínimo um triângulo ou ciclo ímpar  $C_n, n\geq 4$, e $G$ também é um grafo conexo. Mas dado um ciclo $C_n, n\geq 4$, ao adicionar um terceiro vértice adjacente a qualquer vértice desse ciclo então isso induz uma garra, 
    %absurd with the hypothesis of $G$ is Linha de Bipartido;%6
    \item Grafos Bipartido $\cap$ Linha de Bipartido  - Cactus - grafos Bloco, ilustrado na Figura~\ref{fig:exemplosDiagram}(f), grafo $E_7$;%7
    \item Grafos Bipartido $\cap$ Linha de Bipartido $\cap$  Cactus - grafos Bloco, ilustrado na Figura~\ref{fig:exemplosDiagram}(g), grafo $E_8$;%8
    \item Grafos Bipartido $\cap$ Linha de Bipartido $\cap$  Cactus $\cap$ grafos Bloco, ilustrado na Figura~\ref{fig:exemplosDiagram}(h), grafo $E_9$;%9
  \item Grafos Bipartido $\cap$  Cactus $\cap$ Bloco - Grafos Linha de Bipartido, ilustrado na Figura~\ref{fig:exemplosDiagram}(i), grafo $E_{10}$;%10
    \item Grafos Bipartido  $\cap$  Cactus - Bloco -  Grafos Linha de Bipartido, ilustrado na Figura~\ref{fig:exemplosDiagram}(j), grafo $E_{11}$;%11
     \item Grafos Bipartido $\cap$ $B_1$-EPG-Helly - Cactus - Bloco -  Grafos Linha de Bipartido, ilustrado na Figura~\ref{fig:exemplosDiagram}(k), grafo $E_{12}$;%12
      \item Grafos Bipartido - $B_1$-EPG, ilustrado na Figura~\ref{fig:exemplosDiagram}(l), grafo $E_{13}$;%13
      \item Grafos Bloco - Bipartido - Linha de Bipartido  - Cactus, ilustrado na Figura~\ref{fig:exemplosDiagram}(m), grafo $E_{14}$;%14
 
      \item Grafos Bloco $\cap$  Cactus -  Linha de Bipartido - Bipartido, ilustrado na Figura~\ref{fig:exemplosDiagram}(n), grafo $E_{15}$;%15
      \item Grafos Cactus - Bloco -  Linha de Bipartido - Bipartido, ilustrado na Figura~\ref{fig:exemplosDiagram}(o), grafo $E_{16}$, os ciclos ímpares $C_{2n+1},n\geq 2$;%16
      \item Grafos Helly EPG - $B_1$-EPG  - Bipartido, ilustrado na Figura~\ref{fig:exemplosDiagram}(p), grafo  $E_{17}$;%17
\end{enumerate}

 \begin{figure}[htb]	
 
   \centering
  \begin{tabular}{  c c c c  c}
    %\centering
    \includegraphics[width=2cm]{img/octaedroNoLabel.png} 
    & 
    \includegraphics[width=1.5cm]{img/ex3.png} 
    & 
    \includegraphics[width=2cm]{img/diamondNoLabel.png} 
    & 
    \includegraphics[width=1.5cm]{img/k4.png} 
    & 
    \includegraphics[width=2cm]{img/k3.png} 
    \\
    \footnotesize 
    (a)  \footnotesize Graph $E_1$. 
    & 
    \footnotesize (b) Graph $E_2$.
    & 
    \footnotesize (c) Graph $E_3$.
    & 
    \footnotesize (d) Graph $E_4$.
    & 
    \footnotesize (e) Graph $E_5$.
    \\%%Segunda linha
        \includegraphics[width=2.5cm]{img/2c4.png} 
    & 
    \includegraphics[width=1.5cm]{img/c4e.png} 
    & 
    \includegraphics[width=1.8cm]{img/k2.png} 
    & 
    \includegraphics[width=1cm]{img/e10.png} 
    & 
    \includegraphics[width=1.8cm]{img/e11.png} 
    \\ %%Segundo Bloco legendas
    \footnotesize 
    (f)  \footnotesize Graph $E_7$. 
    & 
    \footnotesize (g) Graph $E_8$.
    & 
    \footnotesize (h) Graph $E_9$.
    & 
    \footnotesize (i) Graph $E_{10}$.
    & 
    \footnotesize (j) Graph $E_{11}$.
    %%Terceira linha de imagens
    \\%%Terceira linha
        \includegraphics[width=2.5cm]{img/e12.png} 
    & 
    \includegraphics[width=2cm]{img/k25.png} 
    & 
    \includegraphics[width=2cm]{img/e14.png} 
    & 
    \includegraphics[width=1.8cm]{img/e15.png} 
    & 
    \includegraphics[width=1.8cm]{img/c2n+1.png} 
    \\ %%Terceiro Bloco legendas
    \footnotesize 
    (k)  \footnotesize Graph $E_{12}$. 
    & 
    \footnotesize (l) Graph $E_{13}$.
    & 
    \footnotesize (m) Graph  $E_{14}$.
    & 
    \footnotesize (n) Graph $E_{15}$.
    & 
    \footnotesize (o)  Graph $E_{16}$,  $C_{2n+1},n\geq2$.
    \\
    &&\includegraphics[width=2.5cm]{img/4sunNoLabel.png}&&
    \\
    &&\footnotesize (p)  Graph $E_{17}$.&&
    
    %\multicolumn{3}{c}{ \footnotesize (c) Another partial single bend representation of $H$ } \\
  \end{tabular}
 \caption{The set of instances for Venn Diagram of the graph classes of this chapter.}
 %, see  more in~\cite{leveque2009characterizing,tondato2009grafos}
 \label{fig:exemplosDiagram}
\end{figure}  
 



















%  \begin{figure}[htb]	
 
%   \centering
%   \begin{tabular}{  c c c c  c}
%     %\centering
%     \includegraphics[width=1.7cm]{img/octaedro2.png} 
%     & 
%     \includegraphics[width=1.5cm]{img/ex3.png} 
%     & 
%     \includegraphics[width=2cm]{img/diamondNoLabel.png} 
%     & 
%     \includegraphics[width=1.5cm]{img/k4.png} 
%     & 
%     \includegraphics[width=2cm]{img/k3.png} 
%     \\
%     \footnotesize 
%     (a)  \footnotesize Grafo $E_1$. 
%     & 
%     \footnotesize (b) Grafo $E_2$.
%     & 
%     \footnotesize (c) Grafo $E_3$.
%     & 
%     \footnotesize (d) Grafo $E_4$.
%     & 
%     \footnotesize (e) Grafo $E_5$.
%     \\%%Segunda linha
%         \includegraphics[width=2.5cm]{img/2c4.png} 
%     & 
%     \includegraphics[width=1.5cm]{img/c4e.png} 
%     & 
%     \includegraphics[width=1.8cm]{img/k2.png} 
%     & 
%     \includegraphics[width=1cm]{img/e10.png} 
%     & 
%     \includegraphics[width=1.8cm]{img/e11.png} 
%     \\ %%Segundo Bloco legendas
%     \footnotesize 
%     (f)  \footnotesize Grafo $E_7$. 
%     & 
%     \footnotesize (g) Grafo $E_8$.
%     & 
%     \footnotesize (h) Grafo $E_9$.
%     & 
%     \footnotesize (i) Grafo $E_{10}$.
%     & 
%     \footnotesize (j) Grafo $E_{11}$.
%     %%Terceira linha de imagens
%     \\%%Terceira linha
%         \includegraphics[width=2.5cm]{img/e12.png} 
%     & 
%     \includegraphics[width=2cm]{img/k25.png} 
%     & 
%     \includegraphics[width=2cm]{img/e14.png} 
%     & 
%     \includegraphics[width=1.8cm]{img/e15.png} 
%     & 
%     \includegraphics[width=1.8cm]{img/c2n+1.png} 
%     \\ %%Terceiro Bloco legendas
%     \footnotesize 
%     (k)  \footnotesize Grafo $E_{12}$. 
%     & 
%     \footnotesize (l) Grafo $E_{13}$.
%     & 
%     \footnotesize (m) Grafo  $E_{14}$.
%     & 
%     \footnotesize (n) Grafo $E_{15}$.
%     & 
%     \footnotesize (o)  Grafo $E_{16}$,  $C_{2n+1},n\geq2$.
%     \\
%     &&\includegraphics[width=2.5cm]{img/4sunNoLabel.png}&&
%     \\
%     &&\footnotesize (p)  Grafo $E_{17}$.&&
    
%     %\multicolumn{3}{c}{ \footnotesize (c) Another partial single bend representation of $H$ } \\
%   \end{tabular}
%  \caption{O conjunto de instâncias para o diagrama de Venn das classes de grafos estudadas até aqui.}
%  %, see  more in~\cite{leveque2009characterizing,tondato2009grafos}
%  \label{fig:exemplosDiagram}
% \end{figure}  
 

\section{Relacionamento de Contenção entre Grafos Cordal $B_1$-EPG, VPT e EPT}


 Qualquer grafo que admite uma representação $B_1$-EPG  cujos caminhos não cobrem todas as arestas de um polígono da grade  (i.e. o grafo subjacente da grade é uma árvore) é também um grafo EPT: a mesma representação é ao mesmo tempo uma representação $B_1$-EPG e $EPT$.
Todavia, é fácil verificar que o grafo subjacente da grade de qualquer representação $B_1$-EPG de um ciclo $C_n$ com  $n\geq 5$ não é uma árvore,
%has a non Cordal subjacent grade subgraph 
apesar de $C_n$ ser um grafo  EPT. Nosso objetivo de médio prazo é compreender os grafos $B_1$-EPG que são também grafos  EPT. Quando uma representação $B_1$-EPG pode ser reorganizada de forma a se tornar uma representação EPT? Nessa seção, responderemos essa questão para os grafos   Cordal $B_1$-EPG e de fato provamos que todo grafo Cordal $B_1$-EPG é EPT. Antes de conseguir esse resultado tivemos algumas tentativas sem sucesso para demonstrar que para um dado grafo $G$ com uma representação $B_1$-EPG cujos caminhos cobrissem todas arestas de algum polígono da  grade, e tentando mostrar que se nenhum dos caminhos pudesse ser modificado de forma a evitar uma aresta do polígono, então   $G$ teria algum ciclo sem corda (i.e. $G$ não seria Cordal).  Nossa surpresa foi que a única forma que encontramos de demonstrar o principal Teorema~\ref{teo:b1epgept} foi através dos grafos $VPT$.

 A partir de agora, provaremos o seguinte teorema.

\begin{theorem}\label{teo:CordalB1inVPT}
Cordal $B_1$-EPG $\subsetneq$ VPT. 
\end{theorem}

Em~L{\'e}v{\^e}que et al. \cite{leveque2009characterizing} apud \cite{alcon2015characterizing} os grafos VPT foram caracterizados por uma família minimal de subgrafos induzidos proibidos,
os quais estão representados na
Figura~\ref{fig:16proibidos} mais o ciclo induzido  $C_n$ para $n\geq 4$. Portanto, de forma a provar que os grafos Cordal $B_1$-EPG estão em VPT é suficiente mostrar que nenhum dos grafos na Figura~\ref{fig:16proibidos} 
é $B_1$-EPG. Utilizaremos esta abordagem. %The following lemmas are developed with that objective.   

Primeiro, note que em cada um dos grafos $F_{1}, F_{2}, F_{3}, F_{4}$ e $F_{5}$ ( Figuras~\ref{fig:16proibidos}(a), (b), (c), (d), (e), respectivamente), a vizinhança do  vértice universal (o vértice que está destacado e maior que os outros, nas respectivas Figuras) contem uma  tripla asteroidal. Portanto, pelo Lema~\ref{l:AT-livre}, esses grafos não estão em  $B_1$-EPG.

Agora, em cada um dos grafos $F_{11}, F_{12}, F_{13}, F_{14}$, $F_{15}$ e $F_{16}$  (Figuras~\ref{fig:16proibidos}(k), (l), (m), (n), (o), (p), respectivamente), seja $C$ a clique maximal em destaque negrito. É fácil checar que, em todos os casos, o grafo branch $B(G|C)$ contem um ciclo induzido $C_n$, para algum  $n\geq 4$, ou um caminho induzido $P_6$; assim, pelo  Lema~\ref{l:branch}, os grafos $F_{11}, F_{12}, F_{13}, F_{14}$, $F_{15}$ e $F_{16}$ não estão em $B_1$-EPG.



Um \textit{satélite} de uma clique $K$ é um vértice $v$ tal que $B_v=N(v)\cap K$ é um
subconjunto próprio não vazio de  $K$. O conjunto $B_v$ é chamado de \textit{base} de $v$ e  ela é dito ser \textit{minimal} se nenhuma outra base de um satélite de $K$ está propriamente contida em $B_v$, ver~\cite{alcon2010necessary}.

 Seja $I=[q_1,q_2]$ o intervalo da grade definido pela intersecção $\displaystyle \cap_{v\in K}P_v$, onde $K$
é uma clique-aresta de um grafo $G$. Para qualquer $v\in K$, pela remoção do intervalo $(q_1,q_2)$, o caminho $P_v$
é dividido em duas \textit{partes disjuntas}: \textit{parte 1}  contendo  $q_1$, e  \textit{parte 2}  contendo $q_2$.
Se $w$  é um satélite de $K$ adjacente a $v$, então $P_w\cap P_v$ está contido na parte 1 ou na parte 2 de $P_v$. Diremos que  $P_w$ intersecta $P_v$ no lado 1 ou no lado 2 se ele intersecta na parte 1 ou na parte 2, respectivamente.  Note  que se $w$  também é adjacente a outro vértice $v'$ de $K$, então  $P_w$ intersecta $P_v$ e $P_{v'}$ sobre o mesmo lado de $K$. Isso nos permite dividir os satélites de $K$ em dois \textit{grupos disjuntos}, os que estão do  \textit{lado 1} de $K$ e os que estão do \textit{lado 2}.

%%%%%%%%%%%%%%%%%%%%%%%%%%%
\begin{fac} \label{f:between}Sejam $e_1$, $e_2$ e $e_3$  três arestas distintas de um caminho com uma dobra $P$, e assuma que $e_2$ está entre  $e_1$ e $e_3$ sobre $P$. Se $P_1$ e $P_3$ são caminhos com uma dobra tal que: $P_1$ contem $e_1$, $P_3$ contem $e_3$, e  $P_1$ e $P_3$ intersectam-se em no mínimo uma aresta, então $P_1$ ou $P_3$ contem $e_2$.
\end{fac}
%%%%%%%%%%%%%%%%%%%%%%%%%%%%%%

\begin{lema}\label{coro:3Cliques1EdgeClique}
Seja $G$ um grafo cujo conjunto de vértice pode ser particionado em uma  clique $K$ e um conjunto independente $I=\{w_1,w_2,w_3\}$,  tal que cada vértice de $K$ é adjacente a cada vértice de $I$.  Seja $K_i$ cada clique maximal   $K_i = K \cup w_i$, com $1 \leq i \leq 3$.
Em qualquer representação $B_1$-EPG de $G$, uma dessas cliques, digamos $K_2$, é uma clique-aresta, e seus dois  satélites $w_1$ e $w_3$ estão em lados diferentes.
\end{lema}

\begin{proof}
%%%%%%%%%%%%%%%%%%%%%%%%%%%%%%%
Seja $v$ qualquer vértice de $K$. Para $i\in \{1,2,3\}$, seja $I_{v,i}$ o subcaminho de  $P_v$
definido para ser $P_v\cap P_{w_i}$ (relembrando que consideramos o caminho como um conjunto de arestas). Claramente, os três subcaminhos são mutuamente aresta-disjuntos. Assim, sem perda de generalidade, por simetria, podemos assumir que 
$I_{v,2}$ está entre $I_{v,1}$ e $I_{v,3}$. Afirmamos que a clique $K_2$ é uma clique-aresta.
De fato, se ela não o é, % then $P_4$ contains no edge of the subpath $I_{1,4}$. And,
pelo Lema~\ref{lem:3cliquesNotgarra}, podemos assumir, também sem perda de generalidade, que $K_1$ é uma clique-aresta, o que implica que existe uma aresta  de $I_{v,1}$ coberta por todos os vértices de $K_1$.  Seja $q$ o centro da garra correspondendo à clique $K_2$; claramente $q$ é um ponto de $I_{v,2 }$. Uma vez que todos os vértices de $K$ devem conter $q$, porém nem todos eles podem cobrir a mesma aresta de $I_{v,2}$, we have that $q$ must be the extreme of  $I_{v,2}$ closest to  $I_{v,1}$. Therefore, if we let $e_1$ and $e_2$ be the two edges of $P_v$ incident on $q$ ($e_2$ the one contained in $I_{v,2}$),
we have that  all the vértices of $K$
contain  $e_1$ and at least one vértice of $K$, say $w$,  does not contain $e_2$. Observe this contradicts Fact \ref{f:between} (let $P_1=P_w$, $P_3=P_{w_3}$ and $e_3$ any edge of $I_{v,3}$. We conclude that $K_2$ is an clique-aresta; since $w_1$ and $w_3$ are satélites of $K_2$ the proof is complete. 

%%%%%%%%%%%%%%%%%%%%%%%%%%%%%%%%%%%%%%%%%%%%%%%
% By Lema~\ref{lem:3cliquesNotgarra} the clique maximal s $K_1, K_2$ and $K_3$ can not be represented simultaneously as clique-garras, thus at least one is clique-aresta, we say $K_2$. Since $G$ is a Cordal graph, then when $K_1$ and $K_3$ are represented as clique-garras they have  distinct centers.
% Given $\mathcal{P}_K$ the set of paths corresponding to the vértices of $K$. 
% Each clique-garra $K_i$ has at least one path $P_k \in \mathcal{P}_K$ that bend in this clique-garra. If all paths $P_k \in \mathcal{P}_K$ bend in some clique-garra then these paths can not bend in other clique-garra, i.e. if all paths $P_k \in \mathcal{P}_K$ bend in some clique-garra others cliques will be clique-aresta and the lemma holds. So, consider $K_1$ and $K_3$ as clique-garras.
% In each clique-garra $K_1, K_3$, either all paths $P_k \in \mathcal{P}_K$ intersect in some segment between center of clique-garra and right or left part of base, or then there is only one point of intersection of all paths (center of this garra, obviously) and there is no a representação $B_1$-EPG to $G$. Consider the first situation,  w.l.g. we say that $K_1$ has horizontal base in interval $(q_1,q_2)$ and that all paths $P_k \in \mathcal{P}_K$ intersect  at right of $q_2$. Then, $K_3$ with base in interval $(q_1'',q_2'')$ has same condition but with intersection of all paths at left of $q_1''$. Now we have a problem, if clique-aresta $K_2$ is at left of the center of the $K_1$ then there is a path $P_k \in \mathcal{P}_K$ that bend in the center of $K_1$ such that this path is not in $K_2$, the same is true if $K_2$ is at right of $K_3$. 
% Therefore, in this construction $K_2$ must be between the center of $K_1$ and $K_3$. 

% Thus there is always an clique-aresta $K_i$ located between two satélites $w_i$. 
\end{proof}

  %\include{ResultadosQuantica}
  \chapter{Concluding Remarks}\label{conclusao}

\begin{flushright}
\begin{minipage}[t][0cm][b]{0.47\textwidth}
\emph{
Se eu vi mais longe, foi por estar sobre ombros de gigantes.}
\end{minipage}

\rule[0cm]{7cm}{0.03cm}%{largura}{espessura}

Isaac Newton
\end{flushright}


In chapter~\ref{cap:capiii}, we show that every graph admits a Helly-EPG representation, in particular is possible modify the demonstration to proof that every graph admits a monotonic Helly-EPG representation, and $\frac{\mu}{2n}-1\leq b_H(G)\leq \mu -1$. Besides, we relate Helly-$B_1$-EPG graphs with L-shaped graphs, a natural family of subclasses of $B_1$-EPG. Also, we prove that recognizing (Helly-)$B_k$-EPG graphs is in $\mathcal{NP}$, for every fixed $k$. Finally, we show that recognizing Helly-$B_1$-EPG graphs is $NP$-complete, and it remains $NP$-complete even when restricted to 2-apex and 3-degenerate graphs. In addition, in at the end of chapter we proof that
Helly-$B_k$-EPG $\subsetneq B_k$-EPG for each $k>0$.

In this way we suggest asking about the complexity of recognizing Helly-$B_k$-EPG graphs for each $k>1$. Also, it seems interesting to present characterizations for Helly-$B_k$-EPG representations similar to Lemma~\ref{caracterization} (especially for $k=2$) as well as considering the $h$-Helly-$B_k$ EPG graphs. Regarding L-shaped graphs, it also seems interesting to analyse the classes Helly-$[\llcorner, \ulcorner]$ and Helly-$[\llcorner, \ulcorner, \urcorner]$ (recall Thereom~\ref{theo:HellyLShaped}).

In chapter~\ref{cap:iv}, we have determined the Helly number and strong Helly number of $B_k$-EPG graphs and $B_k$-VPG graphs, for $k \geq 0$. 

Table \ref{tab:Helly-Strong-Helly2} summarizes the results obtained.
 
\Large 

\begin{table}[htb]
    \centering
    \caption{Helly and Strong Helly Numbers for $B_k$-EPG and $B_k$-VPG Graphs}
    \label{tab:Helly-Strong-Helly2}
    \begin{tabular}{c|c|c}
     \multicolumn{3}{c}{}\\
    \cline{1-3} $k$  & $B_k$-EPG & $B_k$-VPG \\
    \cline{1-3} 0 & 2 & 2 \\
    \cline{1-3} 1 & 3 & 4 \\
    \cline{1-3} 2 & 4 & 6 \\
    \cline{1-3} 3 & 8 & 12 \\
    \cline{1-3} $\geq 4$ & unbounded & unbounded \\
    \cline{1-3} 
    \end{tabular}
\end{table}

\normalsize

We leave two questions to be investigated concerning the presented results.

\begin{enumerate}
\item Given a {\it specific}  EPG or VPG graph, the question is to formulate an algorithm to determine its Helly and strong Helly numbers. See \cite{dourado2008improved}, for instance, for such algorithms, applied to general graphs. 

\item The values of the Helly and strong Helly numbers, which were determined in chapter, coincided in all cases. Clearly, in general, this is not the case. We leave as an open question, to find the conditions for such equality to occur. \end{enumerate}


In chapter~\ref{cap:v},  we have considered graphs of intersection of paths, in particular Chordal $B_1$-EPG, VPT and EPT graphs. We show that graphs $\{S_3, S_{3'},S_{3''},C_4\}$-free and others non-trivial subclasses of  $B_1$-EPG graphs have the Helly property, namely by instance Bipartite, Block, Cactus and Line of Bipartite graphs. 
  
  In addition, combining the results of~\cite{alcon2014recognizing,Asinowski2009, golumbic2009} and some proves  presented in chapter, we demonstrate by  Theorems~\ref{teo:chordalB1inVPT} and~\ref{teo:b1epgept} that Chordal $B_1$-EPG graphs are simultaneously contained in the classes of VPT and EPT graphs.  
 
 
%If on the one hand some few graph classes are known to be properly contained in $B_1$-EPG, for instance the $L$-shaped paths graphs see~\cite{cameron2016edge},  and the recognition time for $B_1$-EPG graphs in general is $NP$-complete. On the other hand, in the course of this section we also present some subclasses of Helly-$B_1$ EPG for which the recognition problem is polynomial.

Asinowski and Ries present in~\cite{ries2009} some characterization for special cases of Split $B_1$-EPG graphs, when the stable set has size three or when the clique has size three. Observe that the graphs $F_2, F_{11}, F_{13}, F_{14}, F_{15}$, given in Figure~\ref{fig:16proibidos}, are Split but we used another strategy to prove that they are not $B_1$-EPG graphs. So one question is pertinent: Can we characterize Split graphs in general based in results of this chapter? 

We would like to know the relationship of another graph subclasses of $B_1$-EPG with EPT and VPT graphs. If given an input graph $G$ that is an instance of Weakly Chordal $B_1$-EPG,  Distance-hereditary $B_1$-EPG or any specific subclass, what is the relationship of $G$ with the class of paths in trees? For those same classes of graphs, what happens when we demand that the representations be Helly-$B_1$ EPG?

In the course of this research, in particular, we studied edge-intersection graphs of paths in a grid such that the paths had at most one bend and the representation has the Helly property for the edges of the paths. The problem of recognizing whether a graph has a  $B_{k}$-EPG representation is an open problem for $k\geq 3$, i.e. given a graph $ G$, which is the smallest $k$ such that $ G $ has a $ B_{k}$-EPG representation? Also, the problem of recognizing whether a graph has a  Helly-$B_{k}$-EPG representation remains an open problem for $ k\geq 2$. The evidence observed in the EPG graph literature and the results obtained in this work makes us conjecture that the problem of recognizing both $B_{k}$-EPG and Helly-$B_{k}$-EPG  are both $NP$-complete problems, but this demonstration is unknown.

The study of the parameters Helly number and Helly strong number for edge-intersection graphs on a grid was mentioned only in~\cite{golumbic2009, golumbic2013}, which studied only the parameter strong Helly number. It is easy to see that the questions related of this parameters arise naturally when studying the property of the intersecting sets having the property of being $k$-Helly, thus, another research proposed as the objective of this work was the study of upper and lower bounds for the parameters Helly number and Helly strong number, both for specific classes of EPG and Helly-EPG graphs and also for VPG and Helly-VPG graphs.

In the work of~\citet{cohen2014}, mentioned in Chapter~\ref{cap:capiii}, the Cographs that are $B_1$-EPG are characterized by a minimal family of forbidden subgraphs. Moreover, when considered in context of this work, we can ask: in relation to characterization, what are the Cographs Helly-$B_1$  EPG? Is its recognition also polynomial and can it be done using your co-tree? Is there a difference among these $B_1$-EPG and Helly-$B_1$  EPG families? In addition to the known results for Cographs, we propose potential research topics as problems of recognition or hardness proof  for specific classes of graphs $B_1$-EPG and Helly-$B_1$ EPG.


Last but not least, the author of this thesis (Tanilson) conducted a research as a sandwich doctorate at the National University of La Plata - UNLP, Argentina, for a period of 1 year (March/2019 until March/2020). The welcome, insertion in the research and work group developed during this period must to be gratefully acknowledged. Conducting this research at UNLP brought benefits to this doctoral thesis and to maturity as a researcher, since from this period two articles emerged submitted to the SBPO and to ?????. To continue these works, we hope to explore the Helly-EPG subfamilies.

% Por último mas não menos importante, o autor dessa tese (Tanilson) realizou uma pesquisa a título de doutorado sanduíche na Universidade Nacional de La Plata - UNLP, Argentina, pelo período de 1 ano (Março/2019 até Março/2020). A acolhida, inserção no grupo de pesquisa e trabalho desenvolvido nesse período não poderiam deixar de ser reconhecidos com gratidão. Conduzir essa pesquisa na UNLP trouxe benefícios para esta tese de doutorado e para o amadurecimento como pesquisador, pois desse período emergiram dois artigos submetidos para o SBPO e para ?????. Para continuidade desses trabalhos esperamos trabalhar em caracterização de subfamílias Helly-EPG.


%No decorrer desse trabalho estudamos grafos de aresta-interseção de caminhos em uma grade tal que os caminhos possuíssem no máximo uma dobra e a representação respeitasse à propriedade Helly para as arestas dos caminhos. Mais especificamente, abordamos o problema de reconhecimento de grafos $B_1$-EPG-Helly. Provamos a $NP$-Completude do problema e mostramos que o problema também se mantém $NP$-completo mesmo para as classes de grafos $2$-apex e $3$-degenerado.

%O problema de reconhecer se um grafo possui uma representação $B_{k}$-EPG é um problema em aberto para $k\geq 3$, i.e. dado um grafo $G$, qual é o menor $k$ tal que $G$ possui uma representação $B_{k}$-EPG? Ainda, o problema de reconhecer se um grafo possui uma representação $B_{k}$-EPG-Helly ainda é um problema em aberto para $k\geq 2$.

%As evidências observadas na literatura de grafos EPG e os resultados obtidos nesse trabalho nos faz conjecturar que o problema de reconhecimento tanto de $B_{k}$-EPG quanto de $B_{k}$-EPG-Helly são ambos problemas $NP$-completos, porém essa demonstração ainda é desconhecida.

%O estudo dos parâmetros número de Helly e número de Helly forte para grafos de intersecção em grade foi citado somente em~\cite{golumbic2009, golumbic2013}, que estudaram somente o parâmetro número de Helly forte.
%É fácil perceber que o estudo de tais parâmetros surge quase que naturalmente quando se estuda a propriedade de os conjuntos intersectantes apresentarem a propriedade de serem $k$-Helly, assim, outra pesquisa proposta como objetivo deste trabalho foi o estudo de limites superiores e inferiores para os parâmetros número de Helly e número de Helly forte, tanto para grafos EPG e EPG-Helly quanto para grafos VPG e VPG-Helly, esperamos enriquecer este escrito em breve com esses resultados. Essa pesquisa ainda se encontra em desenvolvimento, por isso os frutos preliminares não foram agregados a este escrito. Deixamos esses resultados para serem apresentados em trabalhos futuros.

%É interessante perceber que a mudança de perspectiva de estudos para grafos EPG pode produzir resultados diferentes com relação a algum parâmetro. Ao compararmos nossos resultados com os de~\cite{golumbic2013} temos a impressão de que, por exemplo, apenas a mudança na definição de caminho (uma visão considera o caminho como um conjunto de vértices e outra visão considera o caminho como um conjunto de arestas) pode gerar resultados diferentes para os parâmetros número de Helly e número de Helly forte. É intenção de nossas pesquisas futuras investigar o comportamento desses parâmetros tanto em grafos EPG quanto em grafos VPG.

% No trabalho de~\citeauthor{cohen2014}~\cite{cohen2014}, citado no Capítulo~\ref{cap:capiii}, são caracterizados os cografos que são $B_1$-EPG por uma família minimal de subgrafos proibidos. Mas uma pergunta que surge naturalmente, quando pensada em conjunto com o contexto deste trabalho é a seguinte: com relação a essa caracterização, quais são os cografos $B_1$-EPG-Helly? Seu reconhecimento também é polinomial e pode ser feito utilizando sua co-árvore? Existe diferença entre as famílias de cografos $B_1$-EPG e $B_1$-EPG-Helly? 
% Além dos resultados conhecidos para cografos, também são temas potenciais de pesquisas os problemas de reconhecimento ou prova de dificuldade para classes específicas de grafos $B_1$-EPG e $B_1$-EPG-Helly.
 





  \backmatter

  \bibliographystyle{coppe-plain}
  %\bibliographystyle{plain}

  \bibliography{refs}

 % \appendix
 % 

\chapter{Artigo submetido para evento internacional} 
\includepdf[pages=-]{./includes/include-pdf-files/wg2019.pdf}
\label{appendix}
  
  \printindex   % Se não tiver indice remissivo pode remover
\end{document}
