\chapter{Concluding Remarks}\label{conclusao}

\begin{flushright}
\begin{minipage}[t][0cm][b]{0.47\textwidth}
\emph{
Se eu vi mais longe, foi por estar sobre ombros de gigantes.}
\end{minipage}

\rule[0cm]{7cm}{0.03cm}%{largura}{espessura}

Isaac Newton
\end{flushright}



No decorrer desse trabalho estudamos grafos de aresta-interseção de caminhos em uma grade tal que os caminhos possuíssem no máximo uma dobra e a representação respeitasse à propriedade Helly para as arestas dos caminhos. Mais especificamente, abordamos o problema de reconhecimento de grafos $B_1$-EPG-Helly. Provamos a $NP$-Completude do problema e mostramos que o problema também se mantém $NP$-completo mesmo para as classes de grafos $2$-apex e $3$-degenerado.

O problema de reconhecer se um grafo possui uma representação $B_{k}$-EPG é um problema em aberto para $k\geq 3$, i.e. dado um grafo $G$, qual é o menor $k$ tal que $G$ possui uma representação $B_{k}$-EPG? Ainda, o problema de reconhecer se um grafo possui uma representação $B_{k}$-EPG-Helly ainda é um problema em aberto para $k\geq 2$.

As evidências observadas na literatura de grafos EPG e os resultados obtidos nesse trabalho nos faz conjecturar que o problema de reconhecimento tanto de $B_{k}$-EPG quanto de $B_{k}$-EPG-Helly são ambos problemas $NP$-completos, porém essa demonstração ainda é desconhecida.

O estudo dos parâmetros número de Helly e número de Helly forte para grafos de intersecção em grade foi citado somente em~\cite{golumbic2009, golumbic2013}, que estudaram somente o parâmetro número de Helly forte.
É fácil perceber que o estudo de tais parâmetros surge quase que naturalmente quando se estuda a propriedade de os conjuntos intersectantes apresentarem a propriedade de serem $k$-Helly, assim, outra pesquisa proposta como objetivo deste trabalho foi o estudo de limites superiores e inferiores para os parâmetros número de Helly e número de Helly forte, tanto para grafos EPG e EPG-Helly quanto para grafos VPG e VPG-Helly, esperamos enriquecer este escrito em breve com esses resultados. Essa pesquisa ainda se encontra em desenvolvimento, por isso os frutos preliminares não foram agregados a este escrito. Deixamos esses resultados para serem apresentados em trabalhos futuros.

É interessante perceber que a mudança de perspectiva de estudos para grafos EPG pode produzir resultados diferentes com relação a algum parâmetro. Ao compararmos nossos resultados com os de~\cite{golumbic2013} temos a impressão de que, por exemplo, apenas a mudança na definição de caminho (uma visão considera o caminho como um conjunto de vértices e outra visão considera o caminho como um conjunto de arestas) pode gerar resultados diferentes para os parâmetros número de Helly e número de Helly forte. É intenção de nossas pesquisas futuras investigar o comportamento desses parâmetros tanto em grafos EPG quanto em grafos VPG.

No trabalho de~\citeauthor{cohen2014}~\cite{cohen2014}, citado no Capítulo~\ref{cap:capiii}, são caracterizados os cografos que são $B_1$-EPG por uma família minimal de subgrafos proibidos. Mas uma pergunta que surge naturalmente, quando pensada em conjunto com o contexto deste trabalho é a seguinte: com relação a essa caracterização, quais são os cografos $B_1$-EPG-Helly? Seu reconhecimento também é polinomial e pode ser feito utilizando sua co-árvore? Existe diferença entre as famílias de cografos $B_1$-EPG e $B_1$-EPG-Helly? 
Além dos resultados conhecidos para cografos, também são temas potenciais de pesquisas os problemas de reconhecimento ou prova de dificuldade para classes específicas de grafos $B_1$-EPG e $B_1$-EPG-Helly
 

Ainda há previsão para uma pesquisa a título de doutorado sanduíche na Universidade Nacional de La Plata - UNLP, Argentina, pelo período de 1 ano. Com essa oportunidade esperamos ter contato com o grupo de pesquisa de Teoria dos Grafos da Faculdade de Ciências Exatas da UNLP, os quais alguns membros já possuem trabalhos relacionados com a propriedade Helly e em grafos arco-circular, EPR, EPT, clique-Helly, VPT, VPG. Acreditamos que conduzir essa pesquisa na UNLP pode trazer benefícios tanto para a UFRJ quanto para o aluno. Uma vez que Tanilson pode contribuir como mão-de-obra intelectual na UNPL, além de essa ser uma boa oportunidade de amadurecimento como pesquisador do ponto de vista de experiência acadêmica e também no mérito de compartilhar conhecimentos sobre a pesquisa desenvolvida em ambas instituições. 

